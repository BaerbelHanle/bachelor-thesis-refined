% ****************************************************************************************************
% classicthesis-config.tex
% formerly known as loadpackages.sty, classicthesis-ldpkg.sty, and classicthesis-preamble.sty
% Use it at the beginning of your ClassicThesis.tex, or as a LaTeX Preamble
% in your ClassicThesis.{tex,lyx} with % ****************************************************************************************************
% classicthesis-config.tex
% formerly known as loadpackages.sty, classicthesis-ldpkg.sty, and classicthesis-preamble.sty
% Use it at the beginning of your ClassicThesis.tex, or as a LaTeX Preamble
% in your ClassicThesis.{tex,lyx} with % ****************************************************************************************************
% classicthesis-config.tex
% formerly known as loadpackages.sty, classicthesis-ldpkg.sty, and classicthesis-preamble.sty
% Use it at the beginning of your ClassicThesis.tex, or as a LaTeX Preamble
% in your ClassicThesis.{tex,lyx} with % ****************************************************************************************************
% classicthesis-config.tex
% formerly known as loadpackages.sty, classicthesis-ldpkg.sty, and classicthesis-preamble.sty
% Use it at the beginning of your ClassicThesis.tex, or as a LaTeX Preamble
% in your ClassicThesis.{tex,lyx} with \input{classicthesis-config}
% ****************************************************************************************************
% If you like the classicthesis, then I would appreciate a postcard.
% My address can be found in the file ClassicThesis.pdf. A collection
% of the postcards I received so far is available online at
% http://postcards.miede.de
% ****************************************************************************************************


% ****************************************************************************************************
% 0. Set the encoding of your files. UTF-8 is the only sensible encoding nowadays. If you can't read
% äöüßáéçèê∂åëæƒÏ€ then change the encoding setting in your editor, not the line below. If your editor
% does not support utf8 use another editor!
% ****************************************************************************************************
\PassOptionsToPackage{utf8}{inputenc}
  \usepackage{inputenc}

\PassOptionsToPackage{T1}{fontenc} % T2A for cyrillics
  \usepackage{fontenc}


% ****************************************************************************************************
% 1. Configure classicthesis for your needs here, e.g., remove "drafting" below
% in order to deactivate the time-stamp on the pages
% (see ClassicThesis.pdf for more information):
% ****************************************************************************************************
\PassOptionsToPackage{
  drafting=true,    % print version information on the bottom of the pages
  tocaligned=false, % the left column of the toc will be aligned (no indentation)
  dottedtoc=false,  % page numbers in ToC flushed right
  eulerchapternumbers=true, % use AMS Euler for chapter font (otherwise Palatino)
  linedheaders=false,       % chaper headers will have line above and beneath
  floatperchapter=true,     % numbering per chapter for all floats (i.e., Figure 1.1)
  eulermath=false,  % use awesome Euler fonts for mathematical formulae (only with pdfLaTeX)
  beramono=true,    % toggle a nice monospaced font (w/ bold)
  palatino=true,    % deactivate standard font for loading another one, see the last section at the end of this file for suggestions
  style=classicthesis % classicthesis, arsclassica
}{classicthesis}


% ****************************************************************************************************
% 2. Personal data and user ad-hoc commands (insert your own data here)
% ****************************************************************************************************
\newcommand{\myTitle}{Zur Theorie der ordinären Entitäten des Brentanoraumes\xspace}
% \newcommand{\mySubtitle}{Ein topologischer Interpretationsansatz\xspace}
\newcommand{\mySubtitle}{Ein repräsentantenbasierter Interpretationsansatz\xspace}
\newcommand{\myDegree}{Bachelor of science\xspace}
\newcommand{\myName}{Bärbel Hanle\xspace}
\newcommand{\mybirthday}{06.02.1982}
\newcommand{\mybirthtown}{Stuttgart}
\newcommand{\mybirthcountry}{Deutschland}
\newcommand{\myProf}{Dr. Frank Loebe\xspace}
\newcommand{\myOtherProf}{Dr. habil. Ringo Baumann\xspace}
\newcommand{\mySupervisor}{\xspace}
\newcommand{\myFaculty}{Institut für Mathematik und Informatik\xspace}
\newcommand{\myDepartment}{\xspace}
\newcommand{\myUni}{Universität Leipzig\xspace}
\newcommand{\myLocation}{Leipzig\xspace}
\newcommand{\myTime}{10.05.2022\xspace} 
\newcommand{\myVersion}{\classicthesis}

% ********************************************************************
% Setup, finetuning, and useful commands
% ********************************************************************
\providecommand{\mLyX}{L\kern-.1667em\lower.25em\hbox{Y}\kelastrn-.125emX\@}
\newcommand{\ie}{i.\,e.}
\newcommand{\Ie}{I.\,e.}
\newcommand{\eg}{e.\,g.}
\newcommand{\Eg}{E.\,g.}
% ****************************************************************************************************


% ****************************************************************************************************
% 3. Loading some handy packages
% ****************************************************************************************************
% ********************************************************************
% Packages with options that might require adjustments
% ********************************************************************
\PassOptionsToPackage{american,ngerman}{babel} % change this to your language(s), main language last
% Spanish languages need extra options in order to work with this template
%\PassOptionsToPackage{spanish,es-lcroman}{babel}
    \usepackage{babel}

\usepackage{csquotes}
\PassOptionsToPackage{%
  %backend=biber,bibencoding=utf8, %instead of bibtex
  backend=bibtex8,bibencoding=ascii,%
  language=auto,%
  style=authoryear,dashed=false%
  %style=authoryear-comp, % Author 1999, 2010
  %bibstyle=authoryear,dashed=false, % dashed: substitute rep. author with ---
  sorting=nyt, % name, year, title
  maxbibnames=10, % default: 3, et al.
  %backref=true,%
  natbib=true % natbib compatibility mode (\citep and \citet still work)
}{biblatex}
    \usepackage{biblatex}

\PassOptionsToPackage{fleqn}{amsmath}       % math environments and more by the AMS
  \usepackage{amsmath}

% ********************************************************************
% General useful packages
% ********************************************************************
\usepackage{graphicx} %
\usepackage{scrhack} % fix warnings when using KOMA with listings package
\usepackage{xspace} % to get the spacing after macros right

%\usepackage{textcase}
\PassOptionsToPackage{printonlyused,smaller}{acronym}
  \usepackage{acronym} % nice macros for handling all acronyms in the thesis
  %\renewcommand{\bflabel}[1]{{#1}\hfill} % fix the list of acronyms --> no longer working
  %\renewcommand*{\acsfont}[1]{\textsc{#1}}
  %\renewcommand*{\aclabelfont}[1]{\acsfont{#1}}
  %\def\bflabel#1{{#1\hfill}}
  \def\bflabel#1{{\acsfont{#1}\hfill}}
  \def\aclabelfont#1{\acsfont{#1}}
  
  %\renewcommand{\acsfont}[1]{{\scshape \MakeTextLowercase{#1}}}
  
\PassOptionsToPackage{activate={true,nocompatibility},final,tracking=true,kerning=true,spacing=true,factor=1100,stretch=10,shrink=10,final}{microtype}%final-even in draft mode
\usepackage[]{microtype}
% ****************************************************************************************************
%\usepackage{pgfplots} % External TikZ/PGF support (thanks to Andreas Nautsch)
%\usetikzlibrary{external}
%\tikzexternalize[mode=list and make, prefix=ext-tikz/]
% ****************************************************************************************************

% ****************************************************************************************************
% 4. Setup floats: tables, (sub)figures, and captions
% ****************************************************************************************************
\usepackage{tabularx} % better tables
  \setlength{\extrarowheight}{3pt} % increase table row height
\newcommand{\tableheadline}[1]{\multicolumn{1}{l}{\spacedlowsmallcaps{#1}}}
\newcommand{\myfloatalign}{\centering} % to be used with each float for alignment
\usepackage{subfig}
% ****************************************************************************************************


% ****************************************************************************************************
% 5. Setup code listings
% ****************************************************************************************************
\usepackage{listings}
%\lstset{emph={trueIndex,root},emphstyle=\color{BlueViolet}}%\underbar} % for special keywords
\lstset{language=[LaTeX]Tex,%C++,
  morekeywords={PassOptionsToPackage,selectlanguage},
  keywordstyle=\color{RoyalBlue},%\bfseries,
  basicstyle=\small\ttfamily,
  %identifierstyle=\color{NavyBlue},
  commentstyle=\color{Green}\ttfamily,
  stringstyle=\rmfamily,
  numbers=none,%left,%
  numberstyle=\scriptsize,%\tiny
  stepnumber=5,
  numbersep=8pt,
  showstringspaces=false,
  breaklines=true,
  %frameround=ftff,
  %frame=single,
  belowcaptionskip=.75\baselineskip
  %frame=L
}
% ****************************************************************************************************




% ****************************************************************************************************
% 6. Last calls before the bar closes
% ****************************************************************************************************
% ********************************************************************
% Her Majesty herself
% ********************************************************************
\usepackage[dottedtoc]{classicthesis}


% ********************************************************************
% Fine-tune hyperreferences (hyperref should be called last)
% ********************************************************************
\hypersetup{%
  %draft, % hyperref's draft mode, for printing see below
  colorlinks=true, linktocpage=true, pdfstartpage=3, pdfstartview=FitV,%
  % uncomment the following line if you want to have black links (e.g., for printing)
  %colorlinks=false, linktocpage=false, pdfstartpage=3, pdfstartview=FitV, pdfborder={0 0 0},%
  breaklinks=true, pageanchor=true,%
  pdfpagemode=UseNone, %
  % pdfpagemode=UseOutlines,%
  plainpages=false, bookmarksnumbered, bookmarksopen=true, bookmarksopenlevel=1,%
  hypertexnames=true, pdfhighlight=/O,%nesting=true,%frenchlinks,%
  urlcolor=CTurl, linkcolor=CTlink, citecolor=CTcitation, %pagecolor=RoyalBlue,%
  %urlcolor=Black, linkcolor=Black, citecolor=Black, %pagecolor=Black,%
  pdftitle={\myTitle},%
  pdfauthor={\textcopyright\ \myName, \myUni, \myFaculty},%
  pdfsubject={},%
  pdfkeywords={},%
  pdfcreator={pdfLaTeX},%
  pdfproducer={LaTeX with hyperref and classicthesis}%
}

%**************************************************************************
% Eigene Pakete
%**************************************************************************
\usepackage{amssymb} % für \varnothing
\usepackage{stmaryrd} % Widerspruchsblitz
\usepackage{amsthm} % Für Theoremumgebungen
  \usepackage{zref-perpage} % FL: zref-perpage is there for 
    % fixing a MikTeX problem with zref, which is loaded by mdframed;
	  % cf. https://github.com/ho-tex/zref/issues/14
\usepackage[framemethod=tikz]{mdframed} % Rand für Theoreme
\usepackage{multirow}
\usepackage{makecell} % Zeilenumburuch innerhalb einer Zelle
\usepackage{longtable} % Tabelle über mehrere Seiten´
\usepackage{subcaption}
% \usepackage[labelformat=parens,labelsep=quad,skip=3pt]{caption}
%\usepackage{graphicx}

%\usepackage{textcase}



% ********************************************************************
% Setup autoreferences (hyperref and babel)
% ********************************************************************
% There are some issues regarding autorefnames
% http://www.tex.ac.uk/cgi-bin/texfaq2html?label=latexwords
% you have to redefine the macros for the
% language you use, e.g., american, ngerman
% (as chosen when loading babel/AtBeginDocument)
% ********************************************************************
\makeatletter
\@ifpackageloaded{babel}%
  {%
    \addto\extrasamerican{%
		  \renewcommand*{\figurename}{Fig.}%
      \renewcommand*{\figureautorefname}{Figure}%
      \renewcommand*{\tableautorefname}{Table}%
      \renewcommand*{\partautorefname}{Part}%
      \renewcommand*{\chapterautorefname}{Chapter}%
      \renewcommand*{\sectionautorefname}{Section}%
      \renewcommand*{\subsectionautorefname}{Section}%
      \renewcommand*{\subsubsectionautorefname}{Section}%
    }%
    \addto\extrasngerman{%
		  \renewcommand*{\figurename}{Abb.}%
      \renewcommand*{\paragraphautorefname}{Absatz}%
      \renewcommand*{\subparagraphautorefname}{Unterabsatz}%
      \renewcommand*{\footnoteautorefname}{Fu\"snote}%
      \renewcommand*{\FancyVerbLineautorefname}{Zeile}%
      \renewcommand*{\theoremautorefname}{Theorem}%
      \renewcommand*{\appendixautorefname}{Anhang}%
      \renewcommand*{\equationautorefname}{Gleichung}%
      \renewcommand*{\itemautorefname}{Punkt}%
    }%
      % Fix to getting autorefs for subfigures right (thanks to Belinda Vogt for changing the definition)
      \providecommand{\subfigureautorefname}{\figureautorefname}%
    }{\relax}
\makeatother


% ********************************************************************
% Development Stuff
% ********************************************************************
\listfiles
%\PassOptionsToPackage{l2tabu,orthodox,abort}{nag}
%  \usepackage{nag}
%\PassOptionsToPackage{warning, all}{onlyamsmath}
%  \usepackage{onlyamsmath}


% ****************************************************************************************************
% 7. Further adjustments (experimental)
% ****************************************************************************************************
% ********************************************************************
% Changing the text area
% ********************************************************************
%\areaset[current]{312pt}{761pt} % 686 (factor 2.2) + 33 head + 42 head \the\footskip
%\setlength{\marginparwidth}{7em}%
%\setlength{\marginparsep}{2em}%

% ********************************************************************
% Using different fonts
% ********************************************************************
%\usepackage[oldstylenums]{kpfonts} % oldstyle notextcomp
% \usepackage[osf]{libertine}
%\usepackage[light,condensed,math]{iwona}
%\renewcommand{\sfdefault}{iwona}
%\usepackage{lmodern} % <-- no osf support :-(
%\usepackage{cfr-lm} %
%\usepackage[urw-garamond]{mathdesign} <-- no osf support :-(
%\usepackage[default,osfigures]{opensans} % scale=0.95
%\usepackage[sfdefault]{FiraSans}
% \usepackage[opticals,mathlf]{MinionPro} % onlytext
% ********************************************************************
%\usepackage[largesc,osf]{newpxtext}
%\linespread{1.05} % a bit more for Palatino
% Used to fix these:
% https://bitbucket.org/amiede/classicthesis/issues/139/italics-in-pallatino-capitals-chapter
% https://bitbucket.org/amiede/classicthesis/issues/45/problema-testatine-su-classicthesis-style
% ********************************************************************
% ****************************************************************************************************

% ****** Custom
%\usepackage[disable]{todonotes}
\usepackage{todonotes}
\usepackage{cleveref}

%***************************************************************************
% Umgebungen
%***************************************************************************
\theoremstyle{definition}

\newtheorem{dfn}{Def}[section]
\newtheorem{nota}[dfn]{Notation}
\newtheorem{konv}[dfn]{Konvention}

\newtheorem{satz}[dfn]{Satz}
\newtheorem{kor}[dfn]{Kor}
\newtheorem{hyp}[dfn]{Hyp}

\newtheorem{bsp}[dfn]{Bsp}
\newtheorem{gegenbsp}[dfn]{Gegenbeispiel}
\newtheorem{bem}[dfn]{Bem}
\newtheorem*{erin}{Erinnerung}
\newtheorem*{bew}{Bew}
\newtheorem*{bewidee}{Beweisidee}

\mdfdefinestyle{grey}{
    skipabove=5pt,
    skipbelow=5pt,
    innerbottommargin=12pt,
    %innerleftmargin=20pt
    innerrightmargin=20pt
    leftmargin=5pt,
    rightmargin=5pt,
    %linewidth=0.2pt,
    roundcorner=2pt,
    backgroundcolor=black!5,
    hidealllines=true,
    needspace=60pt,
		aftersingleframe={\noindent},
}

\mdfdefinestyle{white}{
    skipabove=5pt,
    skipbelow=5pt,
    innerbottommargin=12pt,
    %innerleftmargin=20pt
    innerrightmargin=20pt
    leftmargin=5pt,
    rightmargin=5pt,
    %linewidth=0.2pt,
    roundcorner=2pt,
    %backgroundcolor=black!5,
    %hidealllines=true
}

\mdfdefinestyle{beweis}{
    skipabove=0pt,
    skipbelow=5pt,
    innerbottommargin=12pt,
    %innerleftmargin=20pt
    innerrightmargin=20pt
    leftmargin=5pt,
    rightmargin=5pt,
    %linewidth=0.2pt,
    roundcorner=2pt,
    %backgroundcolor=black!5,
    %hidealllines=true
    linecolor=black!10
%     rightline=false
%     bottomline=false
%     topline=false
}

\surroundwithmdframed[style=grey]{dfn}
\surroundwithmdframed[style=grey]{nota}
\surroundwithmdframed[style=grey]{konv}
\surroundwithmdframed[style=grey]{satz}
\surroundwithmdframed[style=grey]{kor}
\surroundwithmdframed[style=grey]{hyp}
\surroundwithmdframed[style=white]{bsp}
\surroundwithmdframed[style=white]{gegenbsp}
\surroundwithmdframed[style=white]{bem}
\surroundwithmdframed[style=white]{erin}
\surroundwithmdframed[style=beweis]{bew}
\surroundwithmdframed[style=beweis]{bewidee}




\DeclareRobustCommand{\BSO}{\mathcal{BS^O}}

%--------------------------------------------

\newcommand{\thmemph}{\textbf}
\newcommand{\textemph}{\spacedlowsmallcaps}
%\renewcommand{\marginpar}[2][]{} % marginpars verstecken



%*****************************************************************************
% BS-Kommandos
%********************************************************************************
% 
% %%%%%%%%%%%%%%%%%%%%%%%%%%%%%%%%%%%%%%%%%%%%%%%%%%%%%%%%
% % OPTION uniformSymbols STARTS
% %%%%%%%%%%%%%%%%%%%%%%%%%%%%%%%%%%%%%%%%%%%%%%%%%%%%%%%%%%
% \DeclareOption{uniformSymbols}{%
% %
% \newcommand{\GSymbolFont}[1]{#1}
% %
% } %OPTION uniformSymbols ENDS HERE
% %%%%%%%%%%%%%%%%%%%%%%%%%%%%%%%%%%%%%%%%%%%%%%%%%%%
% %%%%%%%%%%%%%%%%%%%%%%%%%%%%%%%%%%%%%%%%%%%%%%%%%%% 
% 
% %%%%%%%%%%%%%%%%%%%%%%%%%%%%%%%%%%%%%%%%%%%%%%%%%%%%%%%%
% % OPTION mboxSymbols STARTS
% %%%%%%%%%%%%%%%%%%%%%%%%%%%%%%%%%%%%%%%%%%%%%%%%%%%%%%%%%%
% \DeclareOption{mboxSymbols}{%
% %
% \renewcommand{\GSymbolFont}[1]{\mbox{#1}}
% %
% } %OPTION mboxSymbols ENDS HERE
% %%%%%%%%%%%%%%%%%%%%%%%%%%%%%%%%%%%%%%%%%%%%%%%%%%%
% %%%%%%%%%%%%%%%%%%%%%%%%%%%%%%%%%%%%%%%%%%%%%%%%%%% 
% 
% %%%%%%%%%%%%%%%%%%%%%%%%%%%%%%%%%%%%%%%%%%%%%%%%%%%%%%%%
% % OPTION mboxSymbols STARTS
% %%%%%%%%%%%%%%%%%%%%%%%%%%%%%%%%%%%%%%%%%%%%%%%%%%%%%%%%%%
% \DeclareOption{italicSymbols}{%
% %
% \renewcommand{\GSymbolFont}[1]{\ensuremath{\mathit{#1}}}
% %
% } %OPTION mboxSymbols ENDS HERE
% %%%%%%%%%%%%%%%%%%%%%%%%%%%%%%%%%%%%%%%%%%%%%%%%%%%
% %%%%%%%%%%%%%%%%%%%%%%%%%%%%%%%%%%%%%%%%%%%%%%%%%%% 
% 
% % Execution of options
% \ExecuteOptions{manAxiomStyle,uniformSymbols}

\newcommand{\GSymbolFont}[1]{\ensuremath{\mathit{#1}}}
\newcommand{\AuxD}{D\hspace*{-0.25ex}}


\newcommand{\GC}{\ensuremath{\GSymbolFont{C}}}
\newcommand{\GCrossdDBn}{\ensuremath{\GSymbolFont{Cross\Gd DB_{n}}}}
\newcommand{\GCrossoneDBn}{\ensuremath{\GSymbolFont{Cross1DB_{n}}}}
\newcommand{\GCrosstwoDBn}{\ensuremath{\GSymbolFont{Cross2DB_{n}}}}
\newcommand{\GCrosszeroDBn}{\ensuremath{\GSymbolFont{Cross0DB_{n}}}}
\newcommand{\Gc}{\ensuremath{\GSymbolFont{c}}}
\newcommand{\Gcrossonedbn}{\ensuremath{\GSymbolFont{cross1db_{n}}}}
\newcommand{\Gcrosstwodbn}{\ensuremath{\GSymbolFont{cross2db_{n}}}}
%\newcommand{\Gcrosszerodb}{\ensuremath{\GSymbolFont{cross0db}}}
\newcommand{\Gcrosszerodbn}{\ensuremath{\GSymbolFont{cross0db_{n}}}}

\newcommand{\Gd}{\boldsymbol{\mathsf{d}}}
\newcommand{\Gone}{\boldsymbol{\mathsf{1}}}
\newcommand{\Gtwo}{\boldsymbol{\mathsf{2}}}
\newcommand{\Gthree}{\boldsymbol{\mathsf{3}}}
\newcommand{\GdD}{\ensuremath{\GSymbolFont{\Gd D}}}
\newcommand{\GdDB}{\ensuremath{\GSymbolFont{\Gd\AuxD B}}}
\newcommand{\GdDC}{\ensuremath{\GSymbolFont{\Gd\AuxD C}}}
\newcommand{\GdDE}{\ensuremath{\GSymbolFont{\Gd\AuxD E}}}
\newcommand{\Gddb}{\ensuremath{\GSymbolFont{\Gd db}}}
\newcommand{\Gddhypp}{\ensuremath{\GSymbolFont{\Gd dhypp}}}
\newcommand{\Gdmdhypp}{\ensuremath{\GSymbolFont{\Gdm dhypp}}}
\newcommand{\Gdircomp}{\ensuremath{\GSymbolFont{dircomp}}}
\newcommand{\Gddircomp}{\ensuremath{\GSymbolFont{\Gd dircomp}}}
\newcommand{\dircomp}{\ensuremath{\GSymbolFont{\Gdp dircomp}}}
\newcommand{\Gdp}{\boldsymbol{\mathsf{(d+1)}}}
\newcommand{\Gdm}{\boldsymbol{\mathsf{(d-1)}}}

\newcommand{\GExOrd}{\ensuremath{\GSymbolFont{ExOrd}}}
\newcommand{\Gequ}{\ensuremath{\GSymbolFont{equ}}}
\newcommand{\Geqdim}{\ensuremath{\GSymbolFont{eqdim}}}
\newcommand{\Gexc}{\ensuremath{\GSymbolFont{exc}}}

\newcommand{\GGrSB}{\ensuremath{\GSymbolFont{Gr\hspace*{-0.25ex}SB}}}
\newcommand{\Ggrsb}{\ensuremath{\GSymbolFont{gr\hspace*{-0.25ex}sb}}}

\newcommand{\Ghypp}{\ensuremath{\GSymbolFont{hypp}}}

\newcommand{\GiCCDd}{\ensuremath{\GSymbolFont{i}_{\Gd}\GSymbolFont{CC}}}
\newcommand{\GiCCDone}{\ensuremath{\GSymbolFont{i}_{1}\GSymbolFont{CC}}}
\newcommand{\GiCCDtwo}{\ensuremath{\GSymbolFont{i}_{2}\GSymbolFont{CC}}}
\newcommand{\GiCCDzero}{\ensuremath{\GSymbolFont{i}_{0}\GSymbolFont{CC}}}
\newcommand{\Ginpart}{\ensuremath{\GSymbolFont{inpart}}}
\newcommand{\Gintersect}{\ensuremath{\GSymbolFont{intsect}}}
\newcommand{\Gintersectn}{\ensuremath{\GSymbolFont{intsect_{n}}}}

\newcommand{\GLDE}{\ensuremath{\GSymbolFont{LDE}}}

\newcommand{\Gcont}{\ensuremath{\GSymbolFont{cont}}}
\newcommand{\Gstrictsb}{\ensuremath{\GSymbolFont{strictsb}}}
\newcommand{\Gweaksb}{\ensuremath{\GSymbolFont{weaksb}}}

%\newcommand{\GkCCDd}{\ensuremath{\GSymbolFont{k}_{\Gd}\GSymbolFont{CC}}}
%\newcommand{\GkCCDone}{\ensuremath{\GSymbolFont{k}_{1}\GSymbolFont{CC}}}
%\newcommand{\GkCCDtwo}{\ensuremath{\GSymbolFont{k}_{2}\GSymbolFont{CC}}}
\newcommand{\GkCCDzero}{\ensuremath{\GSymbolFont{k}_{0}\GSymbolFont{CC}}}

%\newcommand{\GlCCDd}{\ensuremath{\GSymbolFont{l}_{\Gd}\GSymbolFont{CC}}}
\newcommand{\GlCCDone}{\ensuremath{\GSymbolFont{l}_{1}\GSymbolFont{CC}}}
%\newcommand{\GlCCDtwo}{\ensuremath{\GSymbolFont{l}_{2}\GSymbolFont{CC}}}
%\newcommand{\GlCCDzero}{\ensuremath{\GSymbolFont{l}_{0}\GSymbolFont{CC}}}

\newcommand{\GnCCDd}{\ensuremath{\GSymbolFont{n}_{\Gd}\GSymbolFont{CC}}}
\newcommand{\GnCCDone}{\ensuremath{\GSymbolFont{n}_{1}\GSymbolFont{CC}}}
\newcommand{\GnCCDtwo}{\ensuremath{\GSymbolFont{n}_{2}\GSymbolFont{CC}}}
\newcommand{\GnCCDzero}{\ensuremath{\GSymbolFont{n}_{0}\GSymbolFont{CC}}}
%\newcommand{\GnminusiCCDd}{\ensuremath{\GSymbolFont{(n-i)}_{\Gd}\GSymbolFont{CC}}}
\newcommand{\GnminusiCCDone}{\ensuremath{\GSymbolFont{(n-i)}_{1}\GSymbolFont{CC}}}
%\newcommand{\GnminusiCCDtwo}{\ensuremath{\GSymbolFont{(n-i)}_{2}\GSymbolFont{CC}}}
\newcommand{\GnminusiCCDzero}{\ensuremath{\GSymbolFont{(n-i)}_{0}\GSymbolFont{CC}}}


\newcommand{\GOrd}{\ensuremath{\GSymbolFont{Ord}}}
\newcommand{\GoneCCDd}{\ensuremath{\GSymbolFont{1}_{\Gd}\GSymbolFont{CC}}}
\newcommand{\GoneCCDone}{\ensuremath{\GSymbolFont{1}_{1}\GSymbolFont{CC}}}
\newcommand{\GoneCCDtwo}{\ensuremath{\GSymbolFont{1}_{2}\GSymbolFont{CC}}}
\newcommand{\GoneCCDzero}{\ensuremath{\GSymbolFont{1}_{0}\GSymbolFont{CC}}}
\newcommand{\GoneD}{\ensuremath{\GSymbolFont{1D}}}
\newcommand{\GoneDB}{\ensuremath{\GSymbolFont{1\AuxD B}}}
\newcommand{\GoneDC}{\ensuremath{\GSymbolFont{1DC}}}
\newcommand{\GoneDE}{\ensuremath{\GSymbolFont{1\AuxD E}}}
\newcommand{\Gonedb}{\ensuremath{\GSymbolFont{1db}}}
\newcommand{\Gonecont}{\ensuremath{\GSymbolFont{1cont}}}
\newcommand{\Gonedircomp}{\ensuremath{\GSymbolFont{1dircomp}}}
\newcommand{\Gonedhypp}{\ensuremath{\GSymbolFont{1dhypp}}}

\newcommand{\Gpartition}{\ensuremath{\GSymbolFont{partition}}}
\newcommand{\Gpartitioni}{\ensuremath{\GSymbolFont{partition_{i}}}}
\newcommand{\Gpartitionn}{\ensuremath{\GSymbolFont{partition_{n}}}}

\newcommand{\GReg}{\ensuremath{\GSymbolFont{SReg}}}
\newcommand{\Grelcompl}{\ensuremath{\GSymbolFont{rcompl}}}
\newcommand{\Grelcompln}{\ensuremath{\GSymbolFont{rcompl_{n}}}}

\newcommand{\GSB}{\ensuremath{\GSymbolFont{S\hspace*{-0.25ex}B}}}
\newcommand{\GSReg}{\ensuremath{\GSymbolFont{S\hspace*{-0.25ex}Reg}}}
\newcommand{\Gsb}{\ensuremath{\GSymbolFont{sb}}}
\newcommand{\Gscoinc}{\ensuremath{\GSymbolFont{scoinc}}}
\newcommand{\Gsov}{\ensuremath{\GSymbolFont{sov}}}
\newcommand{\Gspart}{\ensuremath{\GSymbolFont{spart}}}
\newcommand{\Gsppart}{\ensuremath{\GSymbolFont{sppart}}}
\newcommand{\Gsum}{\ensuremath{\GSymbolFont{sum}}}
\newcommand{\Gsumi}{\ensuremath{\GSymbolFont{sum_{i}}}}
\newcommand{\Gsumn}{\ensuremath{\GSymbolFont{sum_{n}}}}

\newcommand{\GTop}{\ensuremath{\GSymbolFont{Top}}}
\newcommand{\Gtangpart}{\ensuremath{\GSymbolFont{tangpart}}}
\newcommand{\GtwoD}{\ensuremath{\GSymbolFont{2D}}}
\newcommand{\GtwoDB}{\ensuremath{\GSymbolFont{2\AuxD B}}}
\newcommand{\GtwoDC}{\ensuremath{\GSymbolFont{2DC}}}
\newcommand{\GtwoDE}{\ensuremath{\GSymbolFont{2\AuxD E}}}
\newcommand{\Gtwodb}{\ensuremath{\GSymbolFont{2db}}}
\newcommand{\Gtwodhypp}{\ensuremath{\GSymbolFont{2dhypp}}}

\newcommand{\GzeroD}{\ensuremath{\GSymbolFont{0D}}}
\newcommand{\GzeroDB}{\ensuremath{\GSymbolFont{0\AuxD B}}}
\newcommand{\GzeroDC}{\ensuremath{\GSymbolFont{0DC}}}
\newcommand{\GzeroDE}{\ensuremath{\GSymbolFont{0\AuxD E}}}
\newcommand{\Gzerocont}{\ensuremath{\GSymbolFont{0cont}}}
\newcommand{\Gzerodb}{\ensuremath{\GSymbolFont{0db}}}
\newcommand{\Gzerodhypp}{\ensuremath{\GSymbolFont{0dhypp}}}



%MISC
\newcommand{\bs}{\backslash}
\newcommand{\gap}{\\[0.1ex]\mbox{}}
\newcommand{\m}[1]{\ensuremath{\mathcal{#1}}}
%\newcommand{\MS}{\ensuremath{\mathrel{.}}}
\newcommand{\MS}{\ensuremath{\,.\,}}
\newcommand{\theoryBS}{\ensuremath{\mathcal{BS}}\xspace}
\newcommand{\theoryBSone}{\ensuremath{\mathcal{BS}_{\text{v}1}}\xspace}
\newcommand{\theoryBT}{\ensuremath{\mathcal{BT}}}
\newcommand{\theoryBTC}{\ensuremath{\mathcal{BT}^{\mathcal{C}}}}
\newcommand{\theoryBTR}{\ensuremath{\mathcal{BT}^{\mathcal{R}}}}
\newcommand{\trel}[1]{\textit{#1}}

%********************************************************
% Eigene Kommandos
%********************************************************

\newcommand{\theoryBSO}{\ensuremath{\mathcal{BS}^{\mathcal{O}}}}
%\newcommand{\strukt}{\ensuremath{{\mathcal{R}\text{-Struktur}}}}
\newcommand{\strukt}{$\mathcal{R}$-Struktur\xspace}
\newcommand{\rep}{\ensuremath{\mathcal{R}}}
\newcommand{\univ}{\ensuremath{\mathcal{U}}}
\newcommand{\R}{\ensuremath{\mathbb{R}}}
\newcommand{\N}{\ensuremath{\mathbb{N}}}

\newcommand{\offen}{\ensuremath{\mathcal{O}}}
\newcommand{\abg}{\ensuremath{\mathcal{C}}}
\newcommand{\einf}{\ensuremath{\mathcal{S}}}
\newcommand{\CO}{\ensuremath{\mathcal{CO}}}
\newcommand{\OC}{\ensuremath{\mathcal{OC}}}

\newcommand{\cl}{\ensuremath{\text{cl}}}
\newcommand{\op}{\ensuremath{\text{op}}}
\newcommand{\co}{\ensuremath{\text{co}}}
\newcommand{\oc}{\ensuremath{\text{oc}}}
\newcommand{\HP}{\ensuremath{\text{HP}}}
\newcommand{\rand}{\ensuremath{\partial}}
\newcommand{\ball}{\ensuremath{B}}

\newcommand{\Gdim}{\ensuremath{\text{dim}}}
\newcommand{\Gmaxcon}{\ensuremath{\GSymbolFont{maxcon}}}
\newcommand{\Gloczerodc}{\ensuremath{\GSymbolFont{loc0dc}}}
\newcommand{\Gloconedc}{\ensuremath{\GSymbolFont{loc1dc}}}

\newcommand{\deshalb}{\ensuremath{\rightarrow}}

%***************************************************************
% Struktursymbole
%***************************************************************

\newcommand{\SdDB}{\ensuremath{\GSymbolFont{\Gd\AuxD B}^\mathcal{O}}}
\newcommand{\SdDC}{\ensuremath{\GSymbolFont{\Gd\AuxD C}^\mathcal{O}}}
\newcommand{\SdDE}{\ensuremath{\GSymbolFont{\Gd\AuxD E}^\mathcal{O}}}
\newcommand{\Sddb}{\ensuremath{\GSymbolFont{\Gd db}^\mathcal{O}}}
\newcommand{\Sddhypp}{\ensuremath{\GSymbolFont{\Gd dhypp}^\mathcal{O}}}
\newcommand{\Sdmdhypp}{\ensuremath{\GSymbolFont{\Gdm dhypp}^\mathcal{O}}}
\newcommand{\Sdircomp}{\ensuremath{\GSymbolFont{dircomp}^\mathcal{O}}}
\newcommand{\Sddircomp}{\ensuremath{\GSymbolFont{\Gd dircomp}^\mathcal{O}}}

\newcommand{\SExOrd}{\ensuremath{\GSymbolFont{ExOrd}^\mathcal{O}}}
\newcommand{\Sequ}{\ensuremath{\GSymbolFont{equ}^\mathcal{O}}}
\newcommand{\Seqdim}{\ensuremath{\GSymbolFont{eqdim}^\mathcal{O}}}
\newcommand{\Sexc}{\ensuremath{\GSymbolFont{exc}^\mathcal{O}}}

\newcommand{\SGrSB}{\ensuremath{\GSymbolFont{Gr\hspace*{-0.25ex}SB}^\mathcal{O}}}
\newcommand{\Sgrsb}{\ensuremath{\GSymbolFont{gr\hspace*{-0.25ex}sb}^\mathcal{O}}}

\newcommand{\Shypp}{\ensuremath{\GSymbolFont{hypp}^\mathcal{O}}}

\newcommand{\SiCCDd}{\ensuremath{\GSymbolFont{i}_{\Gd}\GSymbolFont{CC}^\mathcal{O}}}
\newcommand{\SiCCDone}{\ensuremath{\GSymbolFont{i}_{1}\GSymbolFont{CC}^\mathcal{O}}}
\newcommand{\SiCCDtwo}{\ensuremath{\GSymbolFont{i}_{2}\GSymbolFont{CC}^\mathcal{O}}}
\newcommand{\SiCCDzero}{\ensuremath{\GSymbolFont{i}_{0}\GSymbolFont{CC}^\mathcal{O}}}
\newcommand{\Sinpart}{\ensuremath{\GSymbolFont{inpart}^\mathcal{O}}}
\newcommand{\Sintersect}{\ensuremath{\GSymbolFont{intsect}^\mathcal{O}}}
\newcommand{\Sintersectn}{\ensuremath{\GSymbolFont{intsect_{n}}^\mathcal{O}}}

\newcommand{\SLDE}{\ensuremath{\GSymbolFont{LDE}^\mathcal{O}}}

\newcommand{\Scont}{\ensuremath{\GSymbolFont{cont}^\mathcal{O}}}
\newcommand{\Sstrictsb}{\ensuremath{\GSymbolFont{strictsb}^\mathcal{O}}}
\newcommand{\Sweaksb}{\ensuremath{\GSymbolFont{weaksb}^\mathcal{O}}}

%\newcommand{\SkCCDd}{\ensuremath{\GSymbolFont{k}_{\Gd}\GSymbolFont{CC}^\mathcal{O}}}
%\newcommand{\SkCCDone}{\ensuremath{\GSymbolFont{k}_{1}\GSymbolFont{CC}^\mathcal{O}}}
%\newcommand{\SkCCDtwo}{\ensuremath{\GSymbolFont{k}_{2}\GSymbolFont{CC}^\mathcal{O}}}
\newcommand{\SkCCDzero}{\ensuremath{\GSymbolFont{k}_{0}\GSymbolFont{CC}^\mathcal{O}}}

%\newcommand{\SlCCDd}{\ensuremath{\GSymbolFont{l}_{\Gd}\GSymbolFont{CC}^\mathcal{O}}}
\newcommand{\SlCCDone}{\ensuremath{\GSymbolFont{l}_{1}\GSymbolFont{CC}^\mathcal{O}}}
%\newcommand{\SlCCDtwo}{\ensuremath{\GSymbolFont{l}_{2}\GSymbolFont{CC}^\mathcal{O}}}
%\newcommand{\SlCCDzero}{\ensuremath{\GSymbolFont{l}_{0}\GSymbolFont{CC}^\mathcal{O}}}

\newcommand{\SnCCDd}{\ensuremath{\GSymbolFont{n}_{\Gd}\GSymbolFont{CC}^\mathcal{O}}}
\newcommand{\SnCCDone}{\ensuremath{\GSymbolFont{n}_{1}\GSymbolFont{CC}^\mathcal{O}}}
\newcommand{\SnCCDtwo}{\ensuremath{\GSymbolFont{n}_{2}\GSymbolFont{CC}^\mathcal{O}}}
\newcommand{\SnCCDzero}{\ensuremath{\GSymbolFont{n}_{0}\GSymbolFont{CC}^\mathcal{O}}}
%\newcommand{\SnminusiCCDd}{\ensuremath{\GSymbolFont{(n-i)}_{\Gd}\GSymbolFont{CC}^\mathcal{O}}}
\newcommand{\SnminusiCCDone}{\ensuremath{\GSymbolFont{(n-i)}_{1}\GSymbolFont{CC}^\mathcal{O}}}
%\newcommand{\SnminusiCCDtwo}{\ensuremath{\GSymbolFont{(n-i)}_{2}\GSymbolFont{CC}^\mathcal{O}}}
\newcommand{\SnminusiCCDzero}{\ensuremath{\GSymbolFont{(n-i)}_{0}\GSymbolFont{CC}^\mathcal{O}}}


\newcommand{\SOrd}{\ensuremath{\GSymbolFont{Ord}^\mathcal{O}}}
\newcommand{\SoneCCDd}{\ensuremath{\GSymbolFont{1}_{\Gd}\GSymbolFont{CC}^\mathcal{O}}}
\newcommand{\SoneCCDone}{\ensuremath{\GSymbolFont{1}_{1}\GSymbolFont{CC}^\mathcal{O}}}
\newcommand{\SoneCCDtwo}{\ensuremath{\GSymbolFont{1}_{2}\GSymbolFont{CC}^\mathcal{O}}}
\newcommand{\SoneCCDzero}{\ensuremath{\GSymbolFont{1}_{0}\GSymbolFont{CC}^\mathcal{O}}}
\newcommand{\SoneD}{\ensuremath{\GSymbolFont{1D}^\mathcal{O}}}
\newcommand{\SoneDB}{\ensuremath{\GSymbolFont{1\AuxD B}^\mathcal{O}}}
\newcommand{\SoneDC}{\ensuremath{\GSymbolFont{1DC}^\mathcal{O}}}
\newcommand{\SoneDE}{\ensuremath{\GSymbolFont{1\AuxD E}^\mathcal{O}}}
\newcommand{\Sonedb}{\ensuremath{\GSymbolFont{1db}^\mathcal{O}}}
\newcommand{\Sonecont}{\ensuremath{\GSymbolFont{1cont}^\mathcal{O}}}
\newcommand{\Sonedircomp}{\ensuremath{\GSymbolFont{1dircomp}^\mathcal{O}}}
\newcommand{\Sonedhypp}{\ensuremath{\GSymbolFont{1dhypp}^\mathcal{O}}}

\newcommand{\Spartition}{\ensuremath{\GSymbolFont{partition}^\mathcal{O}}}
\newcommand{\Spartitioni}{\ensuremath{\GSymbolFont{partition_{i}}^\mathcal{O}}}
\newcommand{\Spartitionn}{\ensuremath{\GSymbolFont{partition_{n}}^\mathcal{O}}}

\newcommand{\SReg}{\ensuremath{\GSymbolFont{SReg}^\mathcal{O}}}
\newcommand{\Srelcompl}{\ensuremath{\GSymbolFont{rcompl}^\mathcal{O}}}
\newcommand{\Srelcompln}{\ensuremath{\GSymbolFont{rcompl_{n}}^\mathcal{O}}}

\newcommand{\SSB}{\ensuremath{\GSymbolFont{S\hspace*{-0.25ex}B}^\mathcal{O}}}
\newcommand{\SSReg}{\ensuremath{\GSymbolFont{S\hspace*{-0.25ex}Reg}^\mathcal{O}}}
\newcommand{\Ssb}{\ensuremath{\GSymbolFont{sb}^\mathcal{O}}}
\newcommand{\Sscoinc}{\ensuremath{\GSymbolFont{scoinc}^\mathcal{O}}}
\newcommand{\Ssov}{\ensuremath{\GSymbolFont{sov}^\mathcal{O}}}
\newcommand{\Sspart}{\ensuremath{\GSymbolFont{spart}^\mathcal{O}}}
\newcommand{\Ssppart}{\ensuremath{\GSymbolFont{sppart}^\mathcal{O}}}
\newcommand{\Ssum}{\ensuremath{\GSymbolFont{sum}^\mathcal{O}}}
\newcommand{\Ssumi}{\ensuremath{\GSymbolFont{sum_{i}}^\mathcal{O}}}
\newcommand{\Ssumn}{\ensuremath{\GSymbolFont{sum_{n}}^\mathcal{O}}}

\newcommand{\STop}{\ensuremath{\GSymbolFont{Top}^\mathcal{O}}}
\newcommand{\Stangpart}{\ensuremath{\GSymbolFont{tangpart}^\mathcal{O}}}
\newcommand{\StwoD}{\ensuremath{\GSymbolFont{2D}^\mathcal{O}}}
\newcommand{\StwoDB}{\ensuremath{\GSymbolFont{2\AuxD B}^\mathcal{O}}}
\newcommand{\StwoDC}{\ensuremath{\GSymbolFont{2DC}^\mathcal{O}}}
\newcommand{\StwoDE}{\ensuremath{\GSymbolFont{2\AuxD E}^\mathcal{O}}}
\newcommand{\Stwodb}{\ensuremath{\GSymbolFont{2db}^\mathcal{O}}}
\newcommand{\Stwodhypp}{\ensuremath{\GSymbolFont{2dhypp}^\mathcal{O}}}

\newcommand{\SzeroD}{\ensuremath{\GSymbolFont{0D}^\mathcal{O}}}
\newcommand{\SzeroDB}{\ensuremath{\GSymbolFont{0\AuxD B}^\mathcal{O}}}
\newcommand{\SzeroDC}{\ensuremath{\GSymbolFont{0DC}^\mathcal{O}}}
\newcommand{\SzeroDE}{\ensuremath{\GSymbolFont{0\AuxD E}^\mathcal{O}}}
\newcommand{\Szerocont}{\ensuremath{\GSymbolFont{0cont}^\mathcal{O}}}
\newcommand{\Szerodb}{\ensuremath{\GSymbolFont{0db}^\mathcal{O}}}
\newcommand{\Szerodhypp}{\ensuremath{\GSymbolFont{0dhypp}^\mathcal{O}}}

\newcommand{\Smaxcon}{\ensuremath{\GSymbolFont{maxcon}^\mathcal{O}}}
\newcommand{\Sloczerodc}{\ensuremath{\GSymbolFont{loc0dc}^\mathcal{O}}}
\newcommand{\Sloconedc}{\ensuremath{\GSymbolFont{loc1dc}^\mathcal{O}}}

% ****************************************************************************************************
% If you like the classicthesis, then I would appreciate a postcard.
% My address can be found in the file ClassicThesis.pdf. A collection
% of the postcards I received so far is available online at
% http://postcards.miede.de
% ****************************************************************************************************


% ****************************************************************************************************
% 0. Set the encoding of your files. UTF-8 is the only sensible encoding nowadays. If you can't read
% äöüßáéçèê∂åëæƒÏ€ then change the encoding setting in your editor, not the line below. If your editor
% does not support utf8 use another editor!
% ****************************************************************************************************
\PassOptionsToPackage{utf8}{inputenc}
  \usepackage{inputenc}

\PassOptionsToPackage{T1}{fontenc} % T2A for cyrillics
  \usepackage{fontenc}


% ****************************************************************************************************
% 1. Configure classicthesis for your needs here, e.g., remove "drafting" below
% in order to deactivate the time-stamp on the pages
% (see ClassicThesis.pdf for more information):
% ****************************************************************************************************
\PassOptionsToPackage{
  drafting=true,    % print version information on the bottom of the pages
  tocaligned=false, % the left column of the toc will be aligned (no indentation)
  dottedtoc=false,  % page numbers in ToC flushed right
  eulerchapternumbers=true, % use AMS Euler for chapter font (otherwise Palatino)
  linedheaders=false,       % chaper headers will have line above and beneath
  floatperchapter=true,     % numbering per chapter for all floats (i.e., Figure 1.1)
  eulermath=false,  % use awesome Euler fonts for mathematical formulae (only with pdfLaTeX)
  beramono=true,    % toggle a nice monospaced font (w/ bold)
  palatino=true,    % deactivate standard font for loading another one, see the last section at the end of this file for suggestions
  style=classicthesis % classicthesis, arsclassica
}{classicthesis}


% ****************************************************************************************************
% 2. Personal data and user ad-hoc commands (insert your own data here)
% ****************************************************************************************************
\newcommand{\myTitle}{Zur Theorie der ordinären Entitäten des Brentanoraumes\xspace}
% \newcommand{\mySubtitle}{Ein topologischer Interpretationsansatz\xspace}
\newcommand{\mySubtitle}{Ein repräsentantenbasierter Interpretationsansatz\xspace}
\newcommand{\myDegree}{Bachelor of science\xspace}
\newcommand{\myName}{Bärbel Hanle\xspace}
\newcommand{\mybirthday}{06.02.1982}
\newcommand{\mybirthtown}{Stuttgart}
\newcommand{\mybirthcountry}{Deutschland}
\newcommand{\myProf}{Dr. Frank Loebe\xspace}
\newcommand{\myOtherProf}{Dr. habil. Ringo Baumann\xspace}
\newcommand{\mySupervisor}{\xspace}
\newcommand{\myFaculty}{Institut für Mathematik und Informatik\xspace}
\newcommand{\myDepartment}{\xspace}
\newcommand{\myUni}{Universität Leipzig\xspace}
\newcommand{\myLocation}{Leipzig\xspace}
\newcommand{\myTime}{10.05.2022\xspace} 
\newcommand{\myVersion}{\classicthesis}

% ********************************************************************
% Setup, finetuning, and useful commands
% ********************************************************************
\providecommand{\mLyX}{L\kern-.1667em\lower.25em\hbox{Y}\kelastrn-.125emX\@}
\newcommand{\ie}{i.\,e.}
\newcommand{\Ie}{I.\,e.}
\newcommand{\eg}{e.\,g.}
\newcommand{\Eg}{E.\,g.}
% ****************************************************************************************************


% ****************************************************************************************************
% 3. Loading some handy packages
% ****************************************************************************************************
% ********************************************************************
% Packages with options that might require adjustments
% ********************************************************************
\PassOptionsToPackage{american,ngerman}{babel} % change this to your language(s), main language last
% Spanish languages need extra options in order to work with this template
%\PassOptionsToPackage{spanish,es-lcroman}{babel}
    \usepackage{babel}

\usepackage{csquotes}
\PassOptionsToPackage{%
  %backend=biber,bibencoding=utf8, %instead of bibtex
  backend=bibtex8,bibencoding=ascii,%
  language=auto,%
  style=authoryear,dashed=false%
  %style=authoryear-comp, % Author 1999, 2010
  %bibstyle=authoryear,dashed=false, % dashed: substitute rep. author with ---
  sorting=nyt, % name, year, title
  maxbibnames=10, % default: 3, et al.
  %backref=true,%
  natbib=true % natbib compatibility mode (\citep and \citet still work)
}{biblatex}
    \usepackage{biblatex}

\PassOptionsToPackage{fleqn}{amsmath}       % math environments and more by the AMS
  \usepackage{amsmath}

% ********************************************************************
% General useful packages
% ********************************************************************
\usepackage{graphicx} %
\usepackage{scrhack} % fix warnings when using KOMA with listings package
\usepackage{xspace} % to get the spacing after macros right

%\usepackage{textcase}
\PassOptionsToPackage{printonlyused,smaller}{acronym}
  \usepackage{acronym} % nice macros for handling all acronyms in the thesis
  %\renewcommand{\bflabel}[1]{{#1}\hfill} % fix the list of acronyms --> no longer working
  %\renewcommand*{\acsfont}[1]{\textsc{#1}}
  %\renewcommand*{\aclabelfont}[1]{\acsfont{#1}}
  %\def\bflabel#1{{#1\hfill}}
  \def\bflabel#1{{\acsfont{#1}\hfill}}
  \def\aclabelfont#1{\acsfont{#1}}
  
  %\renewcommand{\acsfont}[1]{{\scshape \MakeTextLowercase{#1}}}
  
\PassOptionsToPackage{activate={true,nocompatibility},final,tracking=true,kerning=true,spacing=true,factor=1100,stretch=10,shrink=10,final}{microtype}%final-even in draft mode
\usepackage[]{microtype}
% ****************************************************************************************************
%\usepackage{pgfplots} % External TikZ/PGF support (thanks to Andreas Nautsch)
%\usetikzlibrary{external}
%\tikzexternalize[mode=list and make, prefix=ext-tikz/]
% ****************************************************************************************************

% ****************************************************************************************************
% 4. Setup floats: tables, (sub)figures, and captions
% ****************************************************************************************************
\usepackage{tabularx} % better tables
  \setlength{\extrarowheight}{3pt} % increase table row height
\newcommand{\tableheadline}[1]{\multicolumn{1}{l}{\spacedlowsmallcaps{#1}}}
\newcommand{\myfloatalign}{\centering} % to be used with each float for alignment
\usepackage{subfig}
% ****************************************************************************************************


% ****************************************************************************************************
% 5. Setup code listings
% ****************************************************************************************************
\usepackage{listings}
%\lstset{emph={trueIndex,root},emphstyle=\color{BlueViolet}}%\underbar} % for special keywords
\lstset{language=[LaTeX]Tex,%C++,
  morekeywords={PassOptionsToPackage,selectlanguage},
  keywordstyle=\color{RoyalBlue},%\bfseries,
  basicstyle=\small\ttfamily,
  %identifierstyle=\color{NavyBlue},
  commentstyle=\color{Green}\ttfamily,
  stringstyle=\rmfamily,
  numbers=none,%left,%
  numberstyle=\scriptsize,%\tiny
  stepnumber=5,
  numbersep=8pt,
  showstringspaces=false,
  breaklines=true,
  %frameround=ftff,
  %frame=single,
  belowcaptionskip=.75\baselineskip
  %frame=L
}
% ****************************************************************************************************




% ****************************************************************************************************
% 6. Last calls before the bar closes
% ****************************************************************************************************
% ********************************************************************
% Her Majesty herself
% ********************************************************************
\usepackage[dottedtoc]{classicthesis}


% ********************************************************************
% Fine-tune hyperreferences (hyperref should be called last)
% ********************************************************************
\hypersetup{%
  %draft, % hyperref's draft mode, for printing see below
  colorlinks=true, linktocpage=true, pdfstartpage=3, pdfstartview=FitV,%
  % uncomment the following line if you want to have black links (e.g., for printing)
  %colorlinks=false, linktocpage=false, pdfstartpage=3, pdfstartview=FitV, pdfborder={0 0 0},%
  breaklinks=true, pageanchor=true,%
  pdfpagemode=UseNone, %
  % pdfpagemode=UseOutlines,%
  plainpages=false, bookmarksnumbered, bookmarksopen=true, bookmarksopenlevel=1,%
  hypertexnames=true, pdfhighlight=/O,%nesting=true,%frenchlinks,%
  urlcolor=CTurl, linkcolor=CTlink, citecolor=CTcitation, %pagecolor=RoyalBlue,%
  %urlcolor=Black, linkcolor=Black, citecolor=Black, %pagecolor=Black,%
  pdftitle={\myTitle},%
  pdfauthor={\textcopyright\ \myName, \myUni, \myFaculty},%
  pdfsubject={},%
  pdfkeywords={},%
  pdfcreator={pdfLaTeX},%
  pdfproducer={LaTeX with hyperref and classicthesis}%
}

%**************************************************************************
% Eigene Pakete
%**************************************************************************
\usepackage{amssymb} % für \varnothing
\usepackage{stmaryrd} % Widerspruchsblitz
\usepackage{amsthm} % Für Theoremumgebungen
  \usepackage{zref-perpage} % FL: zref-perpage is there for 
    % fixing a MikTeX problem with zref, which is loaded by mdframed;
	  % cf. https://github.com/ho-tex/zref/issues/14
\usepackage[framemethod=tikz]{mdframed} % Rand für Theoreme
\usepackage{multirow}
\usepackage{makecell} % Zeilenumburuch innerhalb einer Zelle
\usepackage{longtable} % Tabelle über mehrere Seiten´
\usepackage{subcaption}
% \usepackage[labelformat=parens,labelsep=quad,skip=3pt]{caption}
%\usepackage{graphicx}

%\usepackage{textcase}



% ********************************************************************
% Setup autoreferences (hyperref and babel)
% ********************************************************************
% There are some issues regarding autorefnames
% http://www.tex.ac.uk/cgi-bin/texfaq2html?label=latexwords
% you have to redefine the macros for the
% language you use, e.g., american, ngerman
% (as chosen when loading babel/AtBeginDocument)
% ********************************************************************
\makeatletter
\@ifpackageloaded{babel}%
  {%
    \addto\extrasamerican{%
		  \renewcommand*{\figurename}{Fig.}%
      \renewcommand*{\figureautorefname}{Figure}%
      \renewcommand*{\tableautorefname}{Table}%
      \renewcommand*{\partautorefname}{Part}%
      \renewcommand*{\chapterautorefname}{Chapter}%
      \renewcommand*{\sectionautorefname}{Section}%
      \renewcommand*{\subsectionautorefname}{Section}%
      \renewcommand*{\subsubsectionautorefname}{Section}%
    }%
    \addto\extrasngerman{%
		  \renewcommand*{\figurename}{Abb.}%
      \renewcommand*{\paragraphautorefname}{Absatz}%
      \renewcommand*{\subparagraphautorefname}{Unterabsatz}%
      \renewcommand*{\footnoteautorefname}{Fu\"snote}%
      \renewcommand*{\FancyVerbLineautorefname}{Zeile}%
      \renewcommand*{\theoremautorefname}{Theorem}%
      \renewcommand*{\appendixautorefname}{Anhang}%
      \renewcommand*{\equationautorefname}{Gleichung}%
      \renewcommand*{\itemautorefname}{Punkt}%
    }%
      % Fix to getting autorefs for subfigures right (thanks to Belinda Vogt for changing the definition)
      \providecommand{\subfigureautorefname}{\figureautorefname}%
    }{\relax}
\makeatother


% ********************************************************************
% Development Stuff
% ********************************************************************
\listfiles
%\PassOptionsToPackage{l2tabu,orthodox,abort}{nag}
%  \usepackage{nag}
%\PassOptionsToPackage{warning, all}{onlyamsmath}
%  \usepackage{onlyamsmath}


% ****************************************************************************************************
% 7. Further adjustments (experimental)
% ****************************************************************************************************
% ********************************************************************
% Changing the text area
% ********************************************************************
%\areaset[current]{312pt}{761pt} % 686 (factor 2.2) + 33 head + 42 head \the\footskip
%\setlength{\marginparwidth}{7em}%
%\setlength{\marginparsep}{2em}%

% ********************************************************************
% Using different fonts
% ********************************************************************
%\usepackage[oldstylenums]{kpfonts} % oldstyle notextcomp
% \usepackage[osf]{libertine}
%\usepackage[light,condensed,math]{iwona}
%\renewcommand{\sfdefault}{iwona}
%\usepackage{lmodern} % <-- no osf support :-(
%\usepackage{cfr-lm} %
%\usepackage[urw-garamond]{mathdesign} <-- no osf support :-(
%\usepackage[default,osfigures]{opensans} % scale=0.95
%\usepackage[sfdefault]{FiraSans}
% \usepackage[opticals,mathlf]{MinionPro} % onlytext
% ********************************************************************
%\usepackage[largesc,osf]{newpxtext}
%\linespread{1.05} % a bit more for Palatino
% Used to fix these:
% https://bitbucket.org/amiede/classicthesis/issues/139/italics-in-pallatino-capitals-chapter
% https://bitbucket.org/amiede/classicthesis/issues/45/problema-testatine-su-classicthesis-style
% ********************************************************************
% ****************************************************************************************************

% ****** Custom
%\usepackage[disable]{todonotes}
\usepackage{todonotes}
\usepackage{cleveref}

%***************************************************************************
% Umgebungen
%***************************************************************************
\theoremstyle{definition}

\newtheorem{dfn}{Def}[section]
\newtheorem{nota}[dfn]{Notation}
\newtheorem{konv}[dfn]{Konvention}

\newtheorem{satz}[dfn]{Satz}
\newtheorem{kor}[dfn]{Kor}
\newtheorem{hyp}[dfn]{Hyp}

\newtheorem{bsp}[dfn]{Bsp}
\newtheorem{gegenbsp}[dfn]{Gegenbeispiel}
\newtheorem{bem}[dfn]{Bem}
\newtheorem*{erin}{Erinnerung}
\newtheorem*{bew}{Bew}
\newtheorem*{bewidee}{Beweisidee}

\mdfdefinestyle{grey}{
    skipabove=5pt,
    skipbelow=5pt,
    innerbottommargin=12pt,
    %innerleftmargin=20pt
    innerrightmargin=20pt
    leftmargin=5pt,
    rightmargin=5pt,
    %linewidth=0.2pt,
    roundcorner=2pt,
    backgroundcolor=black!5,
    hidealllines=true,
    needspace=60pt,
		aftersingleframe={\noindent},
}

\mdfdefinestyle{white}{
    skipabove=5pt,
    skipbelow=5pt,
    innerbottommargin=12pt,
    %innerleftmargin=20pt
    innerrightmargin=20pt
    leftmargin=5pt,
    rightmargin=5pt,
    %linewidth=0.2pt,
    roundcorner=2pt,
    %backgroundcolor=black!5,
    %hidealllines=true
}

\mdfdefinestyle{beweis}{
    skipabove=0pt,
    skipbelow=5pt,
    innerbottommargin=12pt,
    %innerleftmargin=20pt
    innerrightmargin=20pt
    leftmargin=5pt,
    rightmargin=5pt,
    %linewidth=0.2pt,
    roundcorner=2pt,
    %backgroundcolor=black!5,
    %hidealllines=true
    linecolor=black!10
%     rightline=false
%     bottomline=false
%     topline=false
}

\surroundwithmdframed[style=grey]{dfn}
\surroundwithmdframed[style=grey]{nota}
\surroundwithmdframed[style=grey]{konv}
\surroundwithmdframed[style=grey]{satz}
\surroundwithmdframed[style=grey]{kor}
\surroundwithmdframed[style=grey]{hyp}
\surroundwithmdframed[style=white]{bsp}
\surroundwithmdframed[style=white]{gegenbsp}
\surroundwithmdframed[style=white]{bem}
\surroundwithmdframed[style=white]{erin}
\surroundwithmdframed[style=beweis]{bew}
\surroundwithmdframed[style=beweis]{bewidee}




\DeclareRobustCommand{\BSO}{\mathcal{BS^O}}

%--------------------------------------------

\newcommand{\thmemph}{\textbf}
\newcommand{\textemph}{\spacedlowsmallcaps}
%\renewcommand{\marginpar}[2][]{} % marginpars verstecken



%*****************************************************************************
% BS-Kommandos
%********************************************************************************
% 
% %%%%%%%%%%%%%%%%%%%%%%%%%%%%%%%%%%%%%%%%%%%%%%%%%%%%%%%%
% % OPTION uniformSymbols STARTS
% %%%%%%%%%%%%%%%%%%%%%%%%%%%%%%%%%%%%%%%%%%%%%%%%%%%%%%%%%%
% \DeclareOption{uniformSymbols}{%
% %
% \newcommand{\GSymbolFont}[1]{#1}
% %
% } %OPTION uniformSymbols ENDS HERE
% %%%%%%%%%%%%%%%%%%%%%%%%%%%%%%%%%%%%%%%%%%%%%%%%%%%
% %%%%%%%%%%%%%%%%%%%%%%%%%%%%%%%%%%%%%%%%%%%%%%%%%%% 
% 
% %%%%%%%%%%%%%%%%%%%%%%%%%%%%%%%%%%%%%%%%%%%%%%%%%%%%%%%%
% % OPTION mboxSymbols STARTS
% %%%%%%%%%%%%%%%%%%%%%%%%%%%%%%%%%%%%%%%%%%%%%%%%%%%%%%%%%%
% \DeclareOption{mboxSymbols}{%
% %
% \renewcommand{\GSymbolFont}[1]{\mbox{#1}}
% %
% } %OPTION mboxSymbols ENDS HERE
% %%%%%%%%%%%%%%%%%%%%%%%%%%%%%%%%%%%%%%%%%%%%%%%%%%%
% %%%%%%%%%%%%%%%%%%%%%%%%%%%%%%%%%%%%%%%%%%%%%%%%%%% 
% 
% %%%%%%%%%%%%%%%%%%%%%%%%%%%%%%%%%%%%%%%%%%%%%%%%%%%%%%%%
% % OPTION mboxSymbols STARTS
% %%%%%%%%%%%%%%%%%%%%%%%%%%%%%%%%%%%%%%%%%%%%%%%%%%%%%%%%%%
% \DeclareOption{italicSymbols}{%
% %
% \renewcommand{\GSymbolFont}[1]{\ensuremath{\mathit{#1}}}
% %
% } %OPTION mboxSymbols ENDS HERE
% %%%%%%%%%%%%%%%%%%%%%%%%%%%%%%%%%%%%%%%%%%%%%%%%%%%
% %%%%%%%%%%%%%%%%%%%%%%%%%%%%%%%%%%%%%%%%%%%%%%%%%%% 
% 
% % Execution of options
% \ExecuteOptions{manAxiomStyle,uniformSymbols}

\newcommand{\GSymbolFont}[1]{\ensuremath{\mathit{#1}}}
\newcommand{\AuxD}{D\hspace*{-0.25ex}}


\newcommand{\GC}{\ensuremath{\GSymbolFont{C}}}
\newcommand{\GCrossdDBn}{\ensuremath{\GSymbolFont{Cross\Gd DB_{n}}}}
\newcommand{\GCrossoneDBn}{\ensuremath{\GSymbolFont{Cross1DB_{n}}}}
\newcommand{\GCrosstwoDBn}{\ensuremath{\GSymbolFont{Cross2DB_{n}}}}
\newcommand{\GCrosszeroDBn}{\ensuremath{\GSymbolFont{Cross0DB_{n}}}}
\newcommand{\Gc}{\ensuremath{\GSymbolFont{c}}}
\newcommand{\Gcrossonedbn}{\ensuremath{\GSymbolFont{cross1db_{n}}}}
\newcommand{\Gcrosstwodbn}{\ensuremath{\GSymbolFont{cross2db_{n}}}}
%\newcommand{\Gcrosszerodb}{\ensuremath{\GSymbolFont{cross0db}}}
\newcommand{\Gcrosszerodbn}{\ensuremath{\GSymbolFont{cross0db_{n}}}}

\newcommand{\Gd}{\boldsymbol{\mathsf{d}}}
\newcommand{\Gone}{\boldsymbol{\mathsf{1}}}
\newcommand{\Gtwo}{\boldsymbol{\mathsf{2}}}
\newcommand{\Gthree}{\boldsymbol{\mathsf{3}}}
\newcommand{\GdD}{\ensuremath{\GSymbolFont{\Gd D}}}
\newcommand{\GdDB}{\ensuremath{\GSymbolFont{\Gd\AuxD B}}}
\newcommand{\GdDC}{\ensuremath{\GSymbolFont{\Gd\AuxD C}}}
\newcommand{\GdDE}{\ensuremath{\GSymbolFont{\Gd\AuxD E}}}
\newcommand{\Gddb}{\ensuremath{\GSymbolFont{\Gd db}}}
\newcommand{\Gddhypp}{\ensuremath{\GSymbolFont{\Gd dhypp}}}
\newcommand{\Gdmdhypp}{\ensuremath{\GSymbolFont{\Gdm dhypp}}}
\newcommand{\Gdircomp}{\ensuremath{\GSymbolFont{dircomp}}}
\newcommand{\Gddircomp}{\ensuremath{\GSymbolFont{\Gd dircomp}}}
\newcommand{\dircomp}{\ensuremath{\GSymbolFont{\Gdp dircomp}}}
\newcommand{\Gdp}{\boldsymbol{\mathsf{(d+1)}}}
\newcommand{\Gdm}{\boldsymbol{\mathsf{(d-1)}}}

\newcommand{\GExOrd}{\ensuremath{\GSymbolFont{ExOrd}}}
\newcommand{\Gequ}{\ensuremath{\GSymbolFont{equ}}}
\newcommand{\Geqdim}{\ensuremath{\GSymbolFont{eqdim}}}
\newcommand{\Gexc}{\ensuremath{\GSymbolFont{exc}}}

\newcommand{\GGrSB}{\ensuremath{\GSymbolFont{Gr\hspace*{-0.25ex}SB}}}
\newcommand{\Ggrsb}{\ensuremath{\GSymbolFont{gr\hspace*{-0.25ex}sb}}}

\newcommand{\Ghypp}{\ensuremath{\GSymbolFont{hypp}}}

\newcommand{\GiCCDd}{\ensuremath{\GSymbolFont{i}_{\Gd}\GSymbolFont{CC}}}
\newcommand{\GiCCDone}{\ensuremath{\GSymbolFont{i}_{1}\GSymbolFont{CC}}}
\newcommand{\GiCCDtwo}{\ensuremath{\GSymbolFont{i}_{2}\GSymbolFont{CC}}}
\newcommand{\GiCCDzero}{\ensuremath{\GSymbolFont{i}_{0}\GSymbolFont{CC}}}
\newcommand{\Ginpart}{\ensuremath{\GSymbolFont{inpart}}}
\newcommand{\Gintersect}{\ensuremath{\GSymbolFont{intsect}}}
\newcommand{\Gintersectn}{\ensuremath{\GSymbolFont{intsect_{n}}}}

\newcommand{\GLDE}{\ensuremath{\GSymbolFont{LDE}}}

\newcommand{\Gcont}{\ensuremath{\GSymbolFont{cont}}}
\newcommand{\Gstrictsb}{\ensuremath{\GSymbolFont{strictsb}}}
\newcommand{\Gweaksb}{\ensuremath{\GSymbolFont{weaksb}}}

%\newcommand{\GkCCDd}{\ensuremath{\GSymbolFont{k}_{\Gd}\GSymbolFont{CC}}}
%\newcommand{\GkCCDone}{\ensuremath{\GSymbolFont{k}_{1}\GSymbolFont{CC}}}
%\newcommand{\GkCCDtwo}{\ensuremath{\GSymbolFont{k}_{2}\GSymbolFont{CC}}}
\newcommand{\GkCCDzero}{\ensuremath{\GSymbolFont{k}_{0}\GSymbolFont{CC}}}

%\newcommand{\GlCCDd}{\ensuremath{\GSymbolFont{l}_{\Gd}\GSymbolFont{CC}}}
\newcommand{\GlCCDone}{\ensuremath{\GSymbolFont{l}_{1}\GSymbolFont{CC}}}
%\newcommand{\GlCCDtwo}{\ensuremath{\GSymbolFont{l}_{2}\GSymbolFont{CC}}}
%\newcommand{\GlCCDzero}{\ensuremath{\GSymbolFont{l}_{0}\GSymbolFont{CC}}}

\newcommand{\GnCCDd}{\ensuremath{\GSymbolFont{n}_{\Gd}\GSymbolFont{CC}}}
\newcommand{\GnCCDone}{\ensuremath{\GSymbolFont{n}_{1}\GSymbolFont{CC}}}
\newcommand{\GnCCDtwo}{\ensuremath{\GSymbolFont{n}_{2}\GSymbolFont{CC}}}
\newcommand{\GnCCDzero}{\ensuremath{\GSymbolFont{n}_{0}\GSymbolFont{CC}}}
%\newcommand{\GnminusiCCDd}{\ensuremath{\GSymbolFont{(n-i)}_{\Gd}\GSymbolFont{CC}}}
\newcommand{\GnminusiCCDone}{\ensuremath{\GSymbolFont{(n-i)}_{1}\GSymbolFont{CC}}}
%\newcommand{\GnminusiCCDtwo}{\ensuremath{\GSymbolFont{(n-i)}_{2}\GSymbolFont{CC}}}
\newcommand{\GnminusiCCDzero}{\ensuremath{\GSymbolFont{(n-i)}_{0}\GSymbolFont{CC}}}


\newcommand{\GOrd}{\ensuremath{\GSymbolFont{Ord}}}
\newcommand{\GoneCCDd}{\ensuremath{\GSymbolFont{1}_{\Gd}\GSymbolFont{CC}}}
\newcommand{\GoneCCDone}{\ensuremath{\GSymbolFont{1}_{1}\GSymbolFont{CC}}}
\newcommand{\GoneCCDtwo}{\ensuremath{\GSymbolFont{1}_{2}\GSymbolFont{CC}}}
\newcommand{\GoneCCDzero}{\ensuremath{\GSymbolFont{1}_{0}\GSymbolFont{CC}}}
\newcommand{\GoneD}{\ensuremath{\GSymbolFont{1D}}}
\newcommand{\GoneDB}{\ensuremath{\GSymbolFont{1\AuxD B}}}
\newcommand{\GoneDC}{\ensuremath{\GSymbolFont{1DC}}}
\newcommand{\GoneDE}{\ensuremath{\GSymbolFont{1\AuxD E}}}
\newcommand{\Gonedb}{\ensuremath{\GSymbolFont{1db}}}
\newcommand{\Gonecont}{\ensuremath{\GSymbolFont{1cont}}}
\newcommand{\Gonedircomp}{\ensuremath{\GSymbolFont{1dircomp}}}
\newcommand{\Gonedhypp}{\ensuremath{\GSymbolFont{1dhypp}}}

\newcommand{\Gpartition}{\ensuremath{\GSymbolFont{partition}}}
\newcommand{\Gpartitioni}{\ensuremath{\GSymbolFont{partition_{i}}}}
\newcommand{\Gpartitionn}{\ensuremath{\GSymbolFont{partition_{n}}}}

\newcommand{\GReg}{\ensuremath{\GSymbolFont{SReg}}}
\newcommand{\Grelcompl}{\ensuremath{\GSymbolFont{rcompl}}}
\newcommand{\Grelcompln}{\ensuremath{\GSymbolFont{rcompl_{n}}}}

\newcommand{\GSB}{\ensuremath{\GSymbolFont{S\hspace*{-0.25ex}B}}}
\newcommand{\GSReg}{\ensuremath{\GSymbolFont{S\hspace*{-0.25ex}Reg}}}
\newcommand{\Gsb}{\ensuremath{\GSymbolFont{sb}}}
\newcommand{\Gscoinc}{\ensuremath{\GSymbolFont{scoinc}}}
\newcommand{\Gsov}{\ensuremath{\GSymbolFont{sov}}}
\newcommand{\Gspart}{\ensuremath{\GSymbolFont{spart}}}
\newcommand{\Gsppart}{\ensuremath{\GSymbolFont{sppart}}}
\newcommand{\Gsum}{\ensuremath{\GSymbolFont{sum}}}
\newcommand{\Gsumi}{\ensuremath{\GSymbolFont{sum_{i}}}}
\newcommand{\Gsumn}{\ensuremath{\GSymbolFont{sum_{n}}}}

\newcommand{\GTop}{\ensuremath{\GSymbolFont{Top}}}
\newcommand{\Gtangpart}{\ensuremath{\GSymbolFont{tangpart}}}
\newcommand{\GtwoD}{\ensuremath{\GSymbolFont{2D}}}
\newcommand{\GtwoDB}{\ensuremath{\GSymbolFont{2\AuxD B}}}
\newcommand{\GtwoDC}{\ensuremath{\GSymbolFont{2DC}}}
\newcommand{\GtwoDE}{\ensuremath{\GSymbolFont{2\AuxD E}}}
\newcommand{\Gtwodb}{\ensuremath{\GSymbolFont{2db}}}
\newcommand{\Gtwodhypp}{\ensuremath{\GSymbolFont{2dhypp}}}

\newcommand{\GzeroD}{\ensuremath{\GSymbolFont{0D}}}
\newcommand{\GzeroDB}{\ensuremath{\GSymbolFont{0\AuxD B}}}
\newcommand{\GzeroDC}{\ensuremath{\GSymbolFont{0DC}}}
\newcommand{\GzeroDE}{\ensuremath{\GSymbolFont{0\AuxD E}}}
\newcommand{\Gzerocont}{\ensuremath{\GSymbolFont{0cont}}}
\newcommand{\Gzerodb}{\ensuremath{\GSymbolFont{0db}}}
\newcommand{\Gzerodhypp}{\ensuremath{\GSymbolFont{0dhypp}}}



%MISC
\newcommand{\bs}{\backslash}
\newcommand{\gap}{\\[0.1ex]\mbox{}}
\newcommand{\m}[1]{\ensuremath{\mathcal{#1}}}
%\newcommand{\MS}{\ensuremath{\mathrel{.}}}
\newcommand{\MS}{\ensuremath{\,.\,}}
\newcommand{\theoryBS}{\ensuremath{\mathcal{BS}}\xspace}
\newcommand{\theoryBSone}{\ensuremath{\mathcal{BS}_{\text{v}1}}\xspace}
\newcommand{\theoryBT}{\ensuremath{\mathcal{BT}}}
\newcommand{\theoryBTC}{\ensuremath{\mathcal{BT}^{\mathcal{C}}}}
\newcommand{\theoryBTR}{\ensuremath{\mathcal{BT}^{\mathcal{R}}}}
\newcommand{\trel}[1]{\textit{#1}}

%********************************************************
% Eigene Kommandos
%********************************************************

\newcommand{\theoryBSO}{\ensuremath{\mathcal{BS}^{\mathcal{O}}}}
%\newcommand{\strukt}{\ensuremath{{\mathcal{R}\text{-Struktur}}}}
\newcommand{\strukt}{$\mathcal{R}$-Struktur\xspace}
\newcommand{\rep}{\ensuremath{\mathcal{R}}}
\newcommand{\univ}{\ensuremath{\mathcal{U}}}
\newcommand{\R}{\ensuremath{\mathbb{R}}}
\newcommand{\N}{\ensuremath{\mathbb{N}}}

\newcommand{\offen}{\ensuremath{\mathcal{O}}}
\newcommand{\abg}{\ensuremath{\mathcal{C}}}
\newcommand{\einf}{\ensuremath{\mathcal{S}}}
\newcommand{\CO}{\ensuremath{\mathcal{CO}}}
\newcommand{\OC}{\ensuremath{\mathcal{OC}}}

\newcommand{\cl}{\ensuremath{\text{cl}}}
\newcommand{\op}{\ensuremath{\text{op}}}
\newcommand{\co}{\ensuremath{\text{co}}}
\newcommand{\oc}{\ensuremath{\text{oc}}}
\newcommand{\HP}{\ensuremath{\text{HP}}}
\newcommand{\rand}{\ensuremath{\partial}}
\newcommand{\ball}{\ensuremath{B}}

\newcommand{\Gdim}{\ensuremath{\text{dim}}}
\newcommand{\Gmaxcon}{\ensuremath{\GSymbolFont{maxcon}}}
\newcommand{\Gloczerodc}{\ensuremath{\GSymbolFont{loc0dc}}}
\newcommand{\Gloconedc}{\ensuremath{\GSymbolFont{loc1dc}}}

\newcommand{\deshalb}{\ensuremath{\rightarrow}}

%***************************************************************
% Struktursymbole
%***************************************************************

\newcommand{\SdDB}{\ensuremath{\GSymbolFont{\Gd\AuxD B}^\mathcal{O}}}
\newcommand{\SdDC}{\ensuremath{\GSymbolFont{\Gd\AuxD C}^\mathcal{O}}}
\newcommand{\SdDE}{\ensuremath{\GSymbolFont{\Gd\AuxD E}^\mathcal{O}}}
\newcommand{\Sddb}{\ensuremath{\GSymbolFont{\Gd db}^\mathcal{O}}}
\newcommand{\Sddhypp}{\ensuremath{\GSymbolFont{\Gd dhypp}^\mathcal{O}}}
\newcommand{\Sdmdhypp}{\ensuremath{\GSymbolFont{\Gdm dhypp}^\mathcal{O}}}
\newcommand{\Sdircomp}{\ensuremath{\GSymbolFont{dircomp}^\mathcal{O}}}
\newcommand{\Sddircomp}{\ensuremath{\GSymbolFont{\Gd dircomp}^\mathcal{O}}}

\newcommand{\SExOrd}{\ensuremath{\GSymbolFont{ExOrd}^\mathcal{O}}}
\newcommand{\Sequ}{\ensuremath{\GSymbolFont{equ}^\mathcal{O}}}
\newcommand{\Seqdim}{\ensuremath{\GSymbolFont{eqdim}^\mathcal{O}}}
\newcommand{\Sexc}{\ensuremath{\GSymbolFont{exc}^\mathcal{O}}}

\newcommand{\SGrSB}{\ensuremath{\GSymbolFont{Gr\hspace*{-0.25ex}SB}^\mathcal{O}}}
\newcommand{\Sgrsb}{\ensuremath{\GSymbolFont{gr\hspace*{-0.25ex}sb}^\mathcal{O}}}

\newcommand{\Shypp}{\ensuremath{\GSymbolFont{hypp}^\mathcal{O}}}

\newcommand{\SiCCDd}{\ensuremath{\GSymbolFont{i}_{\Gd}\GSymbolFont{CC}^\mathcal{O}}}
\newcommand{\SiCCDone}{\ensuremath{\GSymbolFont{i}_{1}\GSymbolFont{CC}^\mathcal{O}}}
\newcommand{\SiCCDtwo}{\ensuremath{\GSymbolFont{i}_{2}\GSymbolFont{CC}^\mathcal{O}}}
\newcommand{\SiCCDzero}{\ensuremath{\GSymbolFont{i}_{0}\GSymbolFont{CC}^\mathcal{O}}}
\newcommand{\Sinpart}{\ensuremath{\GSymbolFont{inpart}^\mathcal{O}}}
\newcommand{\Sintersect}{\ensuremath{\GSymbolFont{intsect}^\mathcal{O}}}
\newcommand{\Sintersectn}{\ensuremath{\GSymbolFont{intsect_{n}}^\mathcal{O}}}

\newcommand{\SLDE}{\ensuremath{\GSymbolFont{LDE}^\mathcal{O}}}

\newcommand{\Scont}{\ensuremath{\GSymbolFont{cont}^\mathcal{O}}}
\newcommand{\Sstrictsb}{\ensuremath{\GSymbolFont{strictsb}^\mathcal{O}}}
\newcommand{\Sweaksb}{\ensuremath{\GSymbolFont{weaksb}^\mathcal{O}}}

%\newcommand{\SkCCDd}{\ensuremath{\GSymbolFont{k}_{\Gd}\GSymbolFont{CC}^\mathcal{O}}}
%\newcommand{\SkCCDone}{\ensuremath{\GSymbolFont{k}_{1}\GSymbolFont{CC}^\mathcal{O}}}
%\newcommand{\SkCCDtwo}{\ensuremath{\GSymbolFont{k}_{2}\GSymbolFont{CC}^\mathcal{O}}}
\newcommand{\SkCCDzero}{\ensuremath{\GSymbolFont{k}_{0}\GSymbolFont{CC}^\mathcal{O}}}

%\newcommand{\SlCCDd}{\ensuremath{\GSymbolFont{l}_{\Gd}\GSymbolFont{CC}^\mathcal{O}}}
\newcommand{\SlCCDone}{\ensuremath{\GSymbolFont{l}_{1}\GSymbolFont{CC}^\mathcal{O}}}
%\newcommand{\SlCCDtwo}{\ensuremath{\GSymbolFont{l}_{2}\GSymbolFont{CC}^\mathcal{O}}}
%\newcommand{\SlCCDzero}{\ensuremath{\GSymbolFont{l}_{0}\GSymbolFont{CC}^\mathcal{O}}}

\newcommand{\SnCCDd}{\ensuremath{\GSymbolFont{n}_{\Gd}\GSymbolFont{CC}^\mathcal{O}}}
\newcommand{\SnCCDone}{\ensuremath{\GSymbolFont{n}_{1}\GSymbolFont{CC}^\mathcal{O}}}
\newcommand{\SnCCDtwo}{\ensuremath{\GSymbolFont{n}_{2}\GSymbolFont{CC}^\mathcal{O}}}
\newcommand{\SnCCDzero}{\ensuremath{\GSymbolFont{n}_{0}\GSymbolFont{CC}^\mathcal{O}}}
%\newcommand{\SnminusiCCDd}{\ensuremath{\GSymbolFont{(n-i)}_{\Gd}\GSymbolFont{CC}^\mathcal{O}}}
\newcommand{\SnminusiCCDone}{\ensuremath{\GSymbolFont{(n-i)}_{1}\GSymbolFont{CC}^\mathcal{O}}}
%\newcommand{\SnminusiCCDtwo}{\ensuremath{\GSymbolFont{(n-i)}_{2}\GSymbolFont{CC}^\mathcal{O}}}
\newcommand{\SnminusiCCDzero}{\ensuremath{\GSymbolFont{(n-i)}_{0}\GSymbolFont{CC}^\mathcal{O}}}


\newcommand{\SOrd}{\ensuremath{\GSymbolFont{Ord}^\mathcal{O}}}
\newcommand{\SoneCCDd}{\ensuremath{\GSymbolFont{1}_{\Gd}\GSymbolFont{CC}^\mathcal{O}}}
\newcommand{\SoneCCDone}{\ensuremath{\GSymbolFont{1}_{1}\GSymbolFont{CC}^\mathcal{O}}}
\newcommand{\SoneCCDtwo}{\ensuremath{\GSymbolFont{1}_{2}\GSymbolFont{CC}^\mathcal{O}}}
\newcommand{\SoneCCDzero}{\ensuremath{\GSymbolFont{1}_{0}\GSymbolFont{CC}^\mathcal{O}}}
\newcommand{\SoneD}{\ensuremath{\GSymbolFont{1D}^\mathcal{O}}}
\newcommand{\SoneDB}{\ensuremath{\GSymbolFont{1\AuxD B}^\mathcal{O}}}
\newcommand{\SoneDC}{\ensuremath{\GSymbolFont{1DC}^\mathcal{O}}}
\newcommand{\SoneDE}{\ensuremath{\GSymbolFont{1\AuxD E}^\mathcal{O}}}
\newcommand{\Sonedb}{\ensuremath{\GSymbolFont{1db}^\mathcal{O}}}
\newcommand{\Sonecont}{\ensuremath{\GSymbolFont{1cont}^\mathcal{O}}}
\newcommand{\Sonedircomp}{\ensuremath{\GSymbolFont{1dircomp}^\mathcal{O}}}
\newcommand{\Sonedhypp}{\ensuremath{\GSymbolFont{1dhypp}^\mathcal{O}}}

\newcommand{\Spartition}{\ensuremath{\GSymbolFont{partition}^\mathcal{O}}}
\newcommand{\Spartitioni}{\ensuremath{\GSymbolFont{partition_{i}}^\mathcal{O}}}
\newcommand{\Spartitionn}{\ensuremath{\GSymbolFont{partition_{n}}^\mathcal{O}}}

\newcommand{\SReg}{\ensuremath{\GSymbolFont{SReg}^\mathcal{O}}}
\newcommand{\Srelcompl}{\ensuremath{\GSymbolFont{rcompl}^\mathcal{O}}}
\newcommand{\Srelcompln}{\ensuremath{\GSymbolFont{rcompl_{n}}^\mathcal{O}}}

\newcommand{\SSB}{\ensuremath{\GSymbolFont{S\hspace*{-0.25ex}B}^\mathcal{O}}}
\newcommand{\SSReg}{\ensuremath{\GSymbolFont{S\hspace*{-0.25ex}Reg}^\mathcal{O}}}
\newcommand{\Ssb}{\ensuremath{\GSymbolFont{sb}^\mathcal{O}}}
\newcommand{\Sscoinc}{\ensuremath{\GSymbolFont{scoinc}^\mathcal{O}}}
\newcommand{\Ssov}{\ensuremath{\GSymbolFont{sov}^\mathcal{O}}}
\newcommand{\Sspart}{\ensuremath{\GSymbolFont{spart}^\mathcal{O}}}
\newcommand{\Ssppart}{\ensuremath{\GSymbolFont{sppart}^\mathcal{O}}}
\newcommand{\Ssum}{\ensuremath{\GSymbolFont{sum}^\mathcal{O}}}
\newcommand{\Ssumi}{\ensuremath{\GSymbolFont{sum_{i}}^\mathcal{O}}}
\newcommand{\Ssumn}{\ensuremath{\GSymbolFont{sum_{n}}^\mathcal{O}}}

\newcommand{\STop}{\ensuremath{\GSymbolFont{Top}^\mathcal{O}}}
\newcommand{\Stangpart}{\ensuremath{\GSymbolFont{tangpart}^\mathcal{O}}}
\newcommand{\StwoD}{\ensuremath{\GSymbolFont{2D}^\mathcal{O}}}
\newcommand{\StwoDB}{\ensuremath{\GSymbolFont{2\AuxD B}^\mathcal{O}}}
\newcommand{\StwoDC}{\ensuremath{\GSymbolFont{2DC}^\mathcal{O}}}
\newcommand{\StwoDE}{\ensuremath{\GSymbolFont{2\AuxD E}^\mathcal{O}}}
\newcommand{\Stwodb}{\ensuremath{\GSymbolFont{2db}^\mathcal{O}}}
\newcommand{\Stwodhypp}{\ensuremath{\GSymbolFont{2dhypp}^\mathcal{O}}}

\newcommand{\SzeroD}{\ensuremath{\GSymbolFont{0D}^\mathcal{O}}}
\newcommand{\SzeroDB}{\ensuremath{\GSymbolFont{0\AuxD B}^\mathcal{O}}}
\newcommand{\SzeroDC}{\ensuremath{\GSymbolFont{0DC}^\mathcal{O}}}
\newcommand{\SzeroDE}{\ensuremath{\GSymbolFont{0\AuxD E}^\mathcal{O}}}
\newcommand{\Szerocont}{\ensuremath{\GSymbolFont{0cont}^\mathcal{O}}}
\newcommand{\Szerodb}{\ensuremath{\GSymbolFont{0db}^\mathcal{O}}}
\newcommand{\Szerodhypp}{\ensuremath{\GSymbolFont{0dhypp}^\mathcal{O}}}

\newcommand{\Smaxcon}{\ensuremath{\GSymbolFont{maxcon}^\mathcal{O}}}
\newcommand{\Sloczerodc}{\ensuremath{\GSymbolFont{loc0dc}^\mathcal{O}}}
\newcommand{\Sloconedc}{\ensuremath{\GSymbolFont{loc1dc}^\mathcal{O}}}

% ****************************************************************************************************
% If you like the classicthesis, then I would appreciate a postcard.
% My address can be found in the file ClassicThesis.pdf. A collection
% of the postcards I received so far is available online at
% http://postcards.miede.de
% ****************************************************************************************************


% ****************************************************************************************************
% 0. Set the encoding of your files. UTF-8 is the only sensible encoding nowadays. If you can't read
% äöüßáéçèê∂åëæƒÏ€ then change the encoding setting in your editor, not the line below. If your editor
% does not support utf8 use another editor!
% ****************************************************************************************************
\PassOptionsToPackage{utf8}{inputenc}
  \usepackage{inputenc}

\PassOptionsToPackage{T1}{fontenc} % T2A for cyrillics
  \usepackage{fontenc}


% ****************************************************************************************************
% 1. Configure classicthesis for your needs here, e.g., remove "drafting" below
% in order to deactivate the time-stamp on the pages
% (see ClassicThesis.pdf for more information):
% ****************************************************************************************************
\PassOptionsToPackage{
  drafting=true,    % print version information on the bottom of the pages
  tocaligned=false, % the left column of the toc will be aligned (no indentation)
  dottedtoc=false,  % page numbers in ToC flushed right
  eulerchapternumbers=true, % use AMS Euler for chapter font (otherwise Palatino)
  linedheaders=false,       % chaper headers will have line above and beneath
  floatperchapter=true,     % numbering per chapter for all floats (i.e., Figure 1.1)
  eulermath=false,  % use awesome Euler fonts for mathematical formulae (only with pdfLaTeX)
  beramono=true,    % toggle a nice monospaced font (w/ bold)
  palatino=true,    % deactivate standard font for loading another one, see the last section at the end of this file for suggestions
  style=classicthesis % classicthesis, arsclassica
}{classicthesis}


% ****************************************************************************************************
% 2. Personal data and user ad-hoc commands (insert your own data here)
% ****************************************************************************************************
\newcommand{\myTitle}{Zur Theorie der ordinären Entitäten des Brentanoraumes\xspace}
% \newcommand{\mySubtitle}{Ein topologischer Interpretationsansatz\xspace}
\newcommand{\mySubtitle}{Ein repräsentantenbasierter Interpretationsansatz\xspace}
\newcommand{\myDegree}{Bachelor of science\xspace}
\newcommand{\myName}{Bärbel Hanle\xspace}
\newcommand{\mybirthday}{06.02.1982}
\newcommand{\mybirthtown}{Stuttgart}
\newcommand{\mybirthcountry}{Deutschland}
\newcommand{\myProf}{Dr. Frank Loebe\xspace}
\newcommand{\myOtherProf}{Dr. habil. Ringo Baumann\xspace}
\newcommand{\mySupervisor}{\xspace}
\newcommand{\myFaculty}{Institut für Mathematik und Informatik\xspace}
\newcommand{\myDepartment}{\xspace}
\newcommand{\myUni}{Universität Leipzig\xspace}
\newcommand{\myLocation}{Leipzig\xspace}
\newcommand{\myTime}{10.05.2022\xspace} 
\newcommand{\myVersion}{\classicthesis}

% ********************************************************************
% Setup, finetuning, and useful commands
% ********************************************************************
\providecommand{\mLyX}{L\kern-.1667em\lower.25em\hbox{Y}\kelastrn-.125emX\@}
\newcommand{\ie}{i.\,e.}
\newcommand{\Ie}{I.\,e.}
\newcommand{\eg}{e.\,g.}
\newcommand{\Eg}{E.\,g.}
% ****************************************************************************************************


% ****************************************************************************************************
% 3. Loading some handy packages
% ****************************************************************************************************
% ********************************************************************
% Packages with options that might require adjustments
% ********************************************************************
\PassOptionsToPackage{american,ngerman}{babel} % change this to your language(s), main language last
% Spanish languages need extra options in order to work with this template
%\PassOptionsToPackage{spanish,es-lcroman}{babel}
    \usepackage{babel}

\usepackage{csquotes}
\PassOptionsToPackage{%
  %backend=biber,bibencoding=utf8, %instead of bibtex
  backend=bibtex8,bibencoding=ascii,%
  language=auto,%
  style=authoryear,dashed=false%
  %style=authoryear-comp, % Author 1999, 2010
  %bibstyle=authoryear,dashed=false, % dashed: substitute rep. author with ---
  sorting=nyt, % name, year, title
  maxbibnames=10, % default: 3, et al.
  %backref=true,%
  natbib=true % natbib compatibility mode (\citep and \citet still work)
}{biblatex}
    \usepackage{biblatex}

\PassOptionsToPackage{fleqn}{amsmath}       % math environments and more by the AMS
  \usepackage{amsmath}

% ********************************************************************
% General useful packages
% ********************************************************************
\usepackage{graphicx} %
\usepackage{scrhack} % fix warnings when using KOMA with listings package
\usepackage{xspace} % to get the spacing after macros right

%\usepackage{textcase}
\PassOptionsToPackage{printonlyused,smaller}{acronym}
  \usepackage{acronym} % nice macros for handling all acronyms in the thesis
  %\renewcommand{\bflabel}[1]{{#1}\hfill} % fix the list of acronyms --> no longer working
  %\renewcommand*{\acsfont}[1]{\textsc{#1}}
  %\renewcommand*{\aclabelfont}[1]{\acsfont{#1}}
  %\def\bflabel#1{{#1\hfill}}
  \def\bflabel#1{{\acsfont{#1}\hfill}}
  \def\aclabelfont#1{\acsfont{#1}}
  
  %\renewcommand{\acsfont}[1]{{\scshape \MakeTextLowercase{#1}}}
  
\PassOptionsToPackage{activate={true,nocompatibility},final,tracking=true,kerning=true,spacing=true,factor=1100,stretch=10,shrink=10,final}{microtype}%final-even in draft mode
\usepackage[]{microtype}
% ****************************************************************************************************
%\usepackage{pgfplots} % External TikZ/PGF support (thanks to Andreas Nautsch)
%\usetikzlibrary{external}
%\tikzexternalize[mode=list and make, prefix=ext-tikz/]
% ****************************************************************************************************

% ****************************************************************************************************
% 4. Setup floats: tables, (sub)figures, and captions
% ****************************************************************************************************
\usepackage{tabularx} % better tables
  \setlength{\extrarowheight}{3pt} % increase table row height
\newcommand{\tableheadline}[1]{\multicolumn{1}{l}{\spacedlowsmallcaps{#1}}}
\newcommand{\myfloatalign}{\centering} % to be used with each float for alignment
\usepackage{subfig}
% ****************************************************************************************************


% ****************************************************************************************************
% 5. Setup code listings
% ****************************************************************************************************
\usepackage{listings}
%\lstset{emph={trueIndex,root},emphstyle=\color{BlueViolet}}%\underbar} % for special keywords
\lstset{language=[LaTeX]Tex,%C++,
  morekeywords={PassOptionsToPackage,selectlanguage},
  keywordstyle=\color{RoyalBlue},%\bfseries,
  basicstyle=\small\ttfamily,
  %identifierstyle=\color{NavyBlue},
  commentstyle=\color{Green}\ttfamily,
  stringstyle=\rmfamily,
  numbers=none,%left,%
  numberstyle=\scriptsize,%\tiny
  stepnumber=5,
  numbersep=8pt,
  showstringspaces=false,
  breaklines=true,
  %frameround=ftff,
  %frame=single,
  belowcaptionskip=.75\baselineskip
  %frame=L
}
% ****************************************************************************************************




% ****************************************************************************************************
% 6. Last calls before the bar closes
% ****************************************************************************************************
% ********************************************************************
% Her Majesty herself
% ********************************************************************
\usepackage[dottedtoc]{classicthesis}


% ********************************************************************
% Fine-tune hyperreferences (hyperref should be called last)
% ********************************************************************
\hypersetup{%
  %draft, % hyperref's draft mode, for printing see below
  colorlinks=true, linktocpage=true, pdfstartpage=3, pdfstartview=FitV,%
  % uncomment the following line if you want to have black links (e.g., for printing)
  %colorlinks=false, linktocpage=false, pdfstartpage=3, pdfstartview=FitV, pdfborder={0 0 0},%
  breaklinks=true, pageanchor=true,%
  pdfpagemode=UseNone, %
  % pdfpagemode=UseOutlines,%
  plainpages=false, bookmarksnumbered, bookmarksopen=true, bookmarksopenlevel=1,%
  hypertexnames=true, pdfhighlight=/O,%nesting=true,%frenchlinks,%
  urlcolor=CTurl, linkcolor=CTlink, citecolor=CTcitation, %pagecolor=RoyalBlue,%
  %urlcolor=Black, linkcolor=Black, citecolor=Black, %pagecolor=Black,%
  pdftitle={\myTitle},%
  pdfauthor={\textcopyright\ \myName, \myUni, \myFaculty},%
  pdfsubject={},%
  pdfkeywords={},%
  pdfcreator={pdfLaTeX},%
  pdfproducer={LaTeX with hyperref and classicthesis}%
}

%**************************************************************************
% Eigene Pakete
%**************************************************************************
\usepackage{amssymb} % für \varnothing
\usepackage{stmaryrd} % Widerspruchsblitz
\usepackage{amsthm} % Für Theoremumgebungen
  \usepackage{zref-perpage} % FL: zref-perpage is there for 
    % fixing a MikTeX problem with zref, which is loaded by mdframed;
	  % cf. https://github.com/ho-tex/zref/issues/14
\usepackage[framemethod=tikz]{mdframed} % Rand für Theoreme
\usepackage{multirow}
\usepackage{makecell} % Zeilenumburuch innerhalb einer Zelle
\usepackage{longtable} % Tabelle über mehrere Seiten´
\usepackage{subcaption}
% \usepackage[labelformat=parens,labelsep=quad,skip=3pt]{caption}
%\usepackage{graphicx}

%\usepackage{textcase}



% ********************************************************************
% Setup autoreferences (hyperref and babel)
% ********************************************************************
% There are some issues regarding autorefnames
% http://www.tex.ac.uk/cgi-bin/texfaq2html?label=latexwords
% you have to redefine the macros for the
% language you use, e.g., american, ngerman
% (as chosen when loading babel/AtBeginDocument)
% ********************************************************************
\makeatletter
\@ifpackageloaded{babel}%
  {%
    \addto\extrasamerican{%
		  \renewcommand*{\figurename}{Fig.}%
      \renewcommand*{\figureautorefname}{Figure}%
      \renewcommand*{\tableautorefname}{Table}%
      \renewcommand*{\partautorefname}{Part}%
      \renewcommand*{\chapterautorefname}{Chapter}%
      \renewcommand*{\sectionautorefname}{Section}%
      \renewcommand*{\subsectionautorefname}{Section}%
      \renewcommand*{\subsubsectionautorefname}{Section}%
    }%
    \addto\extrasngerman{%
		  \renewcommand*{\figurename}{Abb.}%
      \renewcommand*{\paragraphautorefname}{Absatz}%
      \renewcommand*{\subparagraphautorefname}{Unterabsatz}%
      \renewcommand*{\footnoteautorefname}{Fu\"snote}%
      \renewcommand*{\FancyVerbLineautorefname}{Zeile}%
      \renewcommand*{\theoremautorefname}{Theorem}%
      \renewcommand*{\appendixautorefname}{Anhang}%
      \renewcommand*{\equationautorefname}{Gleichung}%
      \renewcommand*{\itemautorefname}{Punkt}%
    }%
      % Fix to getting autorefs for subfigures right (thanks to Belinda Vogt for changing the definition)
      \providecommand{\subfigureautorefname}{\figureautorefname}%
    }{\relax}
\makeatother


% ********************************************************************
% Development Stuff
% ********************************************************************
\listfiles
%\PassOptionsToPackage{l2tabu,orthodox,abort}{nag}
%  \usepackage{nag}
%\PassOptionsToPackage{warning, all}{onlyamsmath}
%  \usepackage{onlyamsmath}


% ****************************************************************************************************
% 7. Further adjustments (experimental)
% ****************************************************************************************************
% ********************************************************************
% Changing the text area
% ********************************************************************
%\areaset[current]{312pt}{761pt} % 686 (factor 2.2) + 33 head + 42 head \the\footskip
%\setlength{\marginparwidth}{7em}%
%\setlength{\marginparsep}{2em}%

% ********************************************************************
% Using different fonts
% ********************************************************************
%\usepackage[oldstylenums]{kpfonts} % oldstyle notextcomp
% \usepackage[osf]{libertine}
%\usepackage[light,condensed,math]{iwona}
%\renewcommand{\sfdefault}{iwona}
%\usepackage{lmodern} % <-- no osf support :-(
%\usepackage{cfr-lm} %
%\usepackage[urw-garamond]{mathdesign} <-- no osf support :-(
%\usepackage[default,osfigures]{opensans} % scale=0.95
%\usepackage[sfdefault]{FiraSans}
% \usepackage[opticals,mathlf]{MinionPro} % onlytext
% ********************************************************************
%\usepackage[largesc,osf]{newpxtext}
%\linespread{1.05} % a bit more for Palatino
% Used to fix these:
% https://bitbucket.org/amiede/classicthesis/issues/139/italics-in-pallatino-capitals-chapter
% https://bitbucket.org/amiede/classicthesis/issues/45/problema-testatine-su-classicthesis-style
% ********************************************************************
% ****************************************************************************************************

% ****** Custom
%\usepackage[disable]{todonotes}
\usepackage{todonotes}
\usepackage{cleveref}

%***************************************************************************
% Umgebungen
%***************************************************************************
\theoremstyle{definition}

\newtheorem{dfn}{Def}[section]
\newtheorem{nota}[dfn]{Notation}
\newtheorem{konv}[dfn]{Konvention}

\newtheorem{satz}[dfn]{Satz}
\newtheorem{kor}[dfn]{Kor}
\newtheorem{hyp}[dfn]{Hyp}

\newtheorem{bsp}[dfn]{Bsp}
\newtheorem{gegenbsp}[dfn]{Gegenbeispiel}
\newtheorem{bem}[dfn]{Bem}
\newtheorem*{erin}{Erinnerung}
\newtheorem*{bew}{Bew}
\newtheorem*{bewidee}{Beweisidee}

\mdfdefinestyle{grey}{
    skipabove=5pt,
    skipbelow=5pt,
    innerbottommargin=12pt,
    %innerleftmargin=20pt
    innerrightmargin=20pt
    leftmargin=5pt,
    rightmargin=5pt,
    %linewidth=0.2pt,
    roundcorner=2pt,
    backgroundcolor=black!5,
    hidealllines=true,
    needspace=60pt,
		aftersingleframe={\noindent},
}

\mdfdefinestyle{white}{
    skipabove=5pt,
    skipbelow=5pt,
    innerbottommargin=12pt,
    %innerleftmargin=20pt
    innerrightmargin=20pt
    leftmargin=5pt,
    rightmargin=5pt,
    %linewidth=0.2pt,
    roundcorner=2pt,
    %backgroundcolor=black!5,
    %hidealllines=true
}

\mdfdefinestyle{beweis}{
    skipabove=0pt,
    skipbelow=5pt,
    innerbottommargin=12pt,
    %innerleftmargin=20pt
    innerrightmargin=20pt
    leftmargin=5pt,
    rightmargin=5pt,
    %linewidth=0.2pt,
    roundcorner=2pt,
    %backgroundcolor=black!5,
    %hidealllines=true
    linecolor=black!10
%     rightline=false
%     bottomline=false
%     topline=false
}

\surroundwithmdframed[style=grey]{dfn}
\surroundwithmdframed[style=grey]{nota}
\surroundwithmdframed[style=grey]{konv}
\surroundwithmdframed[style=grey]{satz}
\surroundwithmdframed[style=grey]{kor}
\surroundwithmdframed[style=grey]{hyp}
\surroundwithmdframed[style=white]{bsp}
\surroundwithmdframed[style=white]{gegenbsp}
\surroundwithmdframed[style=white]{bem}
\surroundwithmdframed[style=white]{erin}
\surroundwithmdframed[style=beweis]{bew}
\surroundwithmdframed[style=beweis]{bewidee}




\DeclareRobustCommand{\BSO}{\mathcal{BS^O}}

%--------------------------------------------

\newcommand{\thmemph}{\textbf}
\newcommand{\textemph}{\spacedlowsmallcaps}
%\renewcommand{\marginpar}[2][]{} % marginpars verstecken



%*****************************************************************************
% BS-Kommandos
%********************************************************************************
% 
% %%%%%%%%%%%%%%%%%%%%%%%%%%%%%%%%%%%%%%%%%%%%%%%%%%%%%%%%
% % OPTION uniformSymbols STARTS
% %%%%%%%%%%%%%%%%%%%%%%%%%%%%%%%%%%%%%%%%%%%%%%%%%%%%%%%%%%
% \DeclareOption{uniformSymbols}{%
% %
% \newcommand{\GSymbolFont}[1]{#1}
% %
% } %OPTION uniformSymbols ENDS HERE
% %%%%%%%%%%%%%%%%%%%%%%%%%%%%%%%%%%%%%%%%%%%%%%%%%%%
% %%%%%%%%%%%%%%%%%%%%%%%%%%%%%%%%%%%%%%%%%%%%%%%%%%% 
% 
% %%%%%%%%%%%%%%%%%%%%%%%%%%%%%%%%%%%%%%%%%%%%%%%%%%%%%%%%
% % OPTION mboxSymbols STARTS
% %%%%%%%%%%%%%%%%%%%%%%%%%%%%%%%%%%%%%%%%%%%%%%%%%%%%%%%%%%
% \DeclareOption{mboxSymbols}{%
% %
% \renewcommand{\GSymbolFont}[1]{\mbox{#1}}
% %
% } %OPTION mboxSymbols ENDS HERE
% %%%%%%%%%%%%%%%%%%%%%%%%%%%%%%%%%%%%%%%%%%%%%%%%%%%
% %%%%%%%%%%%%%%%%%%%%%%%%%%%%%%%%%%%%%%%%%%%%%%%%%%% 
% 
% %%%%%%%%%%%%%%%%%%%%%%%%%%%%%%%%%%%%%%%%%%%%%%%%%%%%%%%%
% % OPTION mboxSymbols STARTS
% %%%%%%%%%%%%%%%%%%%%%%%%%%%%%%%%%%%%%%%%%%%%%%%%%%%%%%%%%%
% \DeclareOption{italicSymbols}{%
% %
% \renewcommand{\GSymbolFont}[1]{\ensuremath{\mathit{#1}}}
% %
% } %OPTION mboxSymbols ENDS HERE
% %%%%%%%%%%%%%%%%%%%%%%%%%%%%%%%%%%%%%%%%%%%%%%%%%%%
% %%%%%%%%%%%%%%%%%%%%%%%%%%%%%%%%%%%%%%%%%%%%%%%%%%% 
% 
% % Execution of options
% \ExecuteOptions{manAxiomStyle,uniformSymbols}

\newcommand{\GSymbolFont}[1]{\ensuremath{\mathit{#1}}}
\newcommand{\AuxD}{D\hspace*{-0.25ex}}


\newcommand{\GC}{\ensuremath{\GSymbolFont{C}}}
\newcommand{\GCrossdDBn}{\ensuremath{\GSymbolFont{Cross\Gd DB_{n}}}}
\newcommand{\GCrossoneDBn}{\ensuremath{\GSymbolFont{Cross1DB_{n}}}}
\newcommand{\GCrosstwoDBn}{\ensuremath{\GSymbolFont{Cross2DB_{n}}}}
\newcommand{\GCrosszeroDBn}{\ensuremath{\GSymbolFont{Cross0DB_{n}}}}
\newcommand{\Gc}{\ensuremath{\GSymbolFont{c}}}
\newcommand{\Gcrossonedbn}{\ensuremath{\GSymbolFont{cross1db_{n}}}}
\newcommand{\Gcrosstwodbn}{\ensuremath{\GSymbolFont{cross2db_{n}}}}
%\newcommand{\Gcrosszerodb}{\ensuremath{\GSymbolFont{cross0db}}}
\newcommand{\Gcrosszerodbn}{\ensuremath{\GSymbolFont{cross0db_{n}}}}

\newcommand{\Gd}{\boldsymbol{\mathsf{d}}}
\newcommand{\Gone}{\boldsymbol{\mathsf{1}}}
\newcommand{\Gtwo}{\boldsymbol{\mathsf{2}}}
\newcommand{\Gthree}{\boldsymbol{\mathsf{3}}}
\newcommand{\GdD}{\ensuremath{\GSymbolFont{\Gd D}}}
\newcommand{\GdDB}{\ensuremath{\GSymbolFont{\Gd\AuxD B}}}
\newcommand{\GdDC}{\ensuremath{\GSymbolFont{\Gd\AuxD C}}}
\newcommand{\GdDE}{\ensuremath{\GSymbolFont{\Gd\AuxD E}}}
\newcommand{\Gddb}{\ensuremath{\GSymbolFont{\Gd db}}}
\newcommand{\Gddhypp}{\ensuremath{\GSymbolFont{\Gd dhypp}}}
\newcommand{\Gdmdhypp}{\ensuremath{\GSymbolFont{\Gdm dhypp}}}
\newcommand{\Gdircomp}{\ensuremath{\GSymbolFont{dircomp}}}
\newcommand{\Gddircomp}{\ensuremath{\GSymbolFont{\Gd dircomp}}}
\newcommand{\dircomp}{\ensuremath{\GSymbolFont{\Gdp dircomp}}}
\newcommand{\Gdp}{\boldsymbol{\mathsf{(d+1)}}}
\newcommand{\Gdm}{\boldsymbol{\mathsf{(d-1)}}}

\newcommand{\GExOrd}{\ensuremath{\GSymbolFont{ExOrd}}}
\newcommand{\Gequ}{\ensuremath{\GSymbolFont{equ}}}
\newcommand{\Geqdim}{\ensuremath{\GSymbolFont{eqdim}}}
\newcommand{\Gexc}{\ensuremath{\GSymbolFont{exc}}}

\newcommand{\GGrSB}{\ensuremath{\GSymbolFont{Gr\hspace*{-0.25ex}SB}}}
\newcommand{\Ggrsb}{\ensuremath{\GSymbolFont{gr\hspace*{-0.25ex}sb}}}

\newcommand{\Ghypp}{\ensuremath{\GSymbolFont{hypp}}}

\newcommand{\GiCCDd}{\ensuremath{\GSymbolFont{i}_{\Gd}\GSymbolFont{CC}}}
\newcommand{\GiCCDone}{\ensuremath{\GSymbolFont{i}_{1}\GSymbolFont{CC}}}
\newcommand{\GiCCDtwo}{\ensuremath{\GSymbolFont{i}_{2}\GSymbolFont{CC}}}
\newcommand{\GiCCDzero}{\ensuremath{\GSymbolFont{i}_{0}\GSymbolFont{CC}}}
\newcommand{\Ginpart}{\ensuremath{\GSymbolFont{inpart}}}
\newcommand{\Gintersect}{\ensuremath{\GSymbolFont{intsect}}}
\newcommand{\Gintersectn}{\ensuremath{\GSymbolFont{intsect_{n}}}}

\newcommand{\GLDE}{\ensuremath{\GSymbolFont{LDE}}}

\newcommand{\Gcont}{\ensuremath{\GSymbolFont{cont}}}
\newcommand{\Gstrictsb}{\ensuremath{\GSymbolFont{strictsb}}}
\newcommand{\Gweaksb}{\ensuremath{\GSymbolFont{weaksb}}}

%\newcommand{\GkCCDd}{\ensuremath{\GSymbolFont{k}_{\Gd}\GSymbolFont{CC}}}
%\newcommand{\GkCCDone}{\ensuremath{\GSymbolFont{k}_{1}\GSymbolFont{CC}}}
%\newcommand{\GkCCDtwo}{\ensuremath{\GSymbolFont{k}_{2}\GSymbolFont{CC}}}
\newcommand{\GkCCDzero}{\ensuremath{\GSymbolFont{k}_{0}\GSymbolFont{CC}}}

%\newcommand{\GlCCDd}{\ensuremath{\GSymbolFont{l}_{\Gd}\GSymbolFont{CC}}}
\newcommand{\GlCCDone}{\ensuremath{\GSymbolFont{l}_{1}\GSymbolFont{CC}}}
%\newcommand{\GlCCDtwo}{\ensuremath{\GSymbolFont{l}_{2}\GSymbolFont{CC}}}
%\newcommand{\GlCCDzero}{\ensuremath{\GSymbolFont{l}_{0}\GSymbolFont{CC}}}

\newcommand{\GnCCDd}{\ensuremath{\GSymbolFont{n}_{\Gd}\GSymbolFont{CC}}}
\newcommand{\GnCCDone}{\ensuremath{\GSymbolFont{n}_{1}\GSymbolFont{CC}}}
\newcommand{\GnCCDtwo}{\ensuremath{\GSymbolFont{n}_{2}\GSymbolFont{CC}}}
\newcommand{\GnCCDzero}{\ensuremath{\GSymbolFont{n}_{0}\GSymbolFont{CC}}}
%\newcommand{\GnminusiCCDd}{\ensuremath{\GSymbolFont{(n-i)}_{\Gd}\GSymbolFont{CC}}}
\newcommand{\GnminusiCCDone}{\ensuremath{\GSymbolFont{(n-i)}_{1}\GSymbolFont{CC}}}
%\newcommand{\GnminusiCCDtwo}{\ensuremath{\GSymbolFont{(n-i)}_{2}\GSymbolFont{CC}}}
\newcommand{\GnminusiCCDzero}{\ensuremath{\GSymbolFont{(n-i)}_{0}\GSymbolFont{CC}}}


\newcommand{\GOrd}{\ensuremath{\GSymbolFont{Ord}}}
\newcommand{\GoneCCDd}{\ensuremath{\GSymbolFont{1}_{\Gd}\GSymbolFont{CC}}}
\newcommand{\GoneCCDone}{\ensuremath{\GSymbolFont{1}_{1}\GSymbolFont{CC}}}
\newcommand{\GoneCCDtwo}{\ensuremath{\GSymbolFont{1}_{2}\GSymbolFont{CC}}}
\newcommand{\GoneCCDzero}{\ensuremath{\GSymbolFont{1}_{0}\GSymbolFont{CC}}}
\newcommand{\GoneD}{\ensuremath{\GSymbolFont{1D}}}
\newcommand{\GoneDB}{\ensuremath{\GSymbolFont{1\AuxD B}}}
\newcommand{\GoneDC}{\ensuremath{\GSymbolFont{1DC}}}
\newcommand{\GoneDE}{\ensuremath{\GSymbolFont{1\AuxD E}}}
\newcommand{\Gonedb}{\ensuremath{\GSymbolFont{1db}}}
\newcommand{\Gonecont}{\ensuremath{\GSymbolFont{1cont}}}
\newcommand{\Gonedircomp}{\ensuremath{\GSymbolFont{1dircomp}}}
\newcommand{\Gonedhypp}{\ensuremath{\GSymbolFont{1dhypp}}}

\newcommand{\Gpartition}{\ensuremath{\GSymbolFont{partition}}}
\newcommand{\Gpartitioni}{\ensuremath{\GSymbolFont{partition_{i}}}}
\newcommand{\Gpartitionn}{\ensuremath{\GSymbolFont{partition_{n}}}}

\newcommand{\GReg}{\ensuremath{\GSymbolFont{SReg}}}
\newcommand{\Grelcompl}{\ensuremath{\GSymbolFont{rcompl}}}
\newcommand{\Grelcompln}{\ensuremath{\GSymbolFont{rcompl_{n}}}}

\newcommand{\GSB}{\ensuremath{\GSymbolFont{S\hspace*{-0.25ex}B}}}
\newcommand{\GSReg}{\ensuremath{\GSymbolFont{S\hspace*{-0.25ex}Reg}}}
\newcommand{\Gsb}{\ensuremath{\GSymbolFont{sb}}}
\newcommand{\Gscoinc}{\ensuremath{\GSymbolFont{scoinc}}}
\newcommand{\Gsov}{\ensuremath{\GSymbolFont{sov}}}
\newcommand{\Gspart}{\ensuremath{\GSymbolFont{spart}}}
\newcommand{\Gsppart}{\ensuremath{\GSymbolFont{sppart}}}
\newcommand{\Gsum}{\ensuremath{\GSymbolFont{sum}}}
\newcommand{\Gsumi}{\ensuremath{\GSymbolFont{sum_{i}}}}
\newcommand{\Gsumn}{\ensuremath{\GSymbolFont{sum_{n}}}}

\newcommand{\GTop}{\ensuremath{\GSymbolFont{Top}}}
\newcommand{\Gtangpart}{\ensuremath{\GSymbolFont{tangpart}}}
\newcommand{\GtwoD}{\ensuremath{\GSymbolFont{2D}}}
\newcommand{\GtwoDB}{\ensuremath{\GSymbolFont{2\AuxD B}}}
\newcommand{\GtwoDC}{\ensuremath{\GSymbolFont{2DC}}}
\newcommand{\GtwoDE}{\ensuremath{\GSymbolFont{2\AuxD E}}}
\newcommand{\Gtwodb}{\ensuremath{\GSymbolFont{2db}}}
\newcommand{\Gtwodhypp}{\ensuremath{\GSymbolFont{2dhypp}}}

\newcommand{\GzeroD}{\ensuremath{\GSymbolFont{0D}}}
\newcommand{\GzeroDB}{\ensuremath{\GSymbolFont{0\AuxD B}}}
\newcommand{\GzeroDC}{\ensuremath{\GSymbolFont{0DC}}}
\newcommand{\GzeroDE}{\ensuremath{\GSymbolFont{0\AuxD E}}}
\newcommand{\Gzerocont}{\ensuremath{\GSymbolFont{0cont}}}
\newcommand{\Gzerodb}{\ensuremath{\GSymbolFont{0db}}}
\newcommand{\Gzerodhypp}{\ensuremath{\GSymbolFont{0dhypp}}}



%MISC
\newcommand{\bs}{\backslash}
\newcommand{\gap}{\\[0.1ex]\mbox{}}
\newcommand{\m}[1]{\ensuremath{\mathcal{#1}}}
%\newcommand{\MS}{\ensuremath{\mathrel{.}}}
\newcommand{\MS}{\ensuremath{\,.\,}}
\newcommand{\theoryBS}{\ensuremath{\mathcal{BS}}\xspace}
\newcommand{\theoryBSone}{\ensuremath{\mathcal{BS}_{\text{v}1}}\xspace}
\newcommand{\theoryBT}{\ensuremath{\mathcal{BT}}}
\newcommand{\theoryBTC}{\ensuremath{\mathcal{BT}^{\mathcal{C}}}}
\newcommand{\theoryBTR}{\ensuremath{\mathcal{BT}^{\mathcal{R}}}}
\newcommand{\trel}[1]{\textit{#1}}

%********************************************************
% Eigene Kommandos
%********************************************************

\newcommand{\theoryBSO}{\ensuremath{\mathcal{BS}^{\mathcal{O}}}}
%\newcommand{\strukt}{\ensuremath{{\mathcal{R}\text{-Struktur}}}}
\newcommand{\strukt}{$\mathcal{R}$-Struktur\xspace}
\newcommand{\rep}{\ensuremath{\mathcal{R}}}
\newcommand{\univ}{\ensuremath{\mathcal{U}}}
\newcommand{\R}{\ensuremath{\mathbb{R}}}
\newcommand{\N}{\ensuremath{\mathbb{N}}}

\newcommand{\offen}{\ensuremath{\mathcal{O}}}
\newcommand{\abg}{\ensuremath{\mathcal{C}}}
\newcommand{\einf}{\ensuremath{\mathcal{S}}}
\newcommand{\CO}{\ensuremath{\mathcal{CO}}}
\newcommand{\OC}{\ensuremath{\mathcal{OC}}}

\newcommand{\cl}{\ensuremath{\text{cl}}}
\newcommand{\op}{\ensuremath{\text{op}}}
\newcommand{\co}{\ensuremath{\text{co}}}
\newcommand{\oc}{\ensuremath{\text{oc}}}
\newcommand{\HP}{\ensuremath{\text{HP}}}
\newcommand{\rand}{\ensuremath{\partial}}
\newcommand{\ball}{\ensuremath{B}}

\newcommand{\Gdim}{\ensuremath{\text{dim}}}
\newcommand{\Gmaxcon}{\ensuremath{\GSymbolFont{maxcon}}}
\newcommand{\Gloczerodc}{\ensuremath{\GSymbolFont{loc0dc}}}
\newcommand{\Gloconedc}{\ensuremath{\GSymbolFont{loc1dc}}}

\newcommand{\deshalb}{\ensuremath{\rightarrow}}

%***************************************************************
% Struktursymbole
%***************************************************************

\newcommand{\SdDB}{\ensuremath{\GSymbolFont{\Gd\AuxD B}^\mathcal{O}}}
\newcommand{\SdDC}{\ensuremath{\GSymbolFont{\Gd\AuxD C}^\mathcal{O}}}
\newcommand{\SdDE}{\ensuremath{\GSymbolFont{\Gd\AuxD E}^\mathcal{O}}}
\newcommand{\Sddb}{\ensuremath{\GSymbolFont{\Gd db}^\mathcal{O}}}
\newcommand{\Sddhypp}{\ensuremath{\GSymbolFont{\Gd dhypp}^\mathcal{O}}}
\newcommand{\Sdmdhypp}{\ensuremath{\GSymbolFont{\Gdm dhypp}^\mathcal{O}}}
\newcommand{\Sdircomp}{\ensuremath{\GSymbolFont{dircomp}^\mathcal{O}}}
\newcommand{\Sddircomp}{\ensuremath{\GSymbolFont{\Gd dircomp}^\mathcal{O}}}

\newcommand{\SExOrd}{\ensuremath{\GSymbolFont{ExOrd}^\mathcal{O}}}
\newcommand{\Sequ}{\ensuremath{\GSymbolFont{equ}^\mathcal{O}}}
\newcommand{\Seqdim}{\ensuremath{\GSymbolFont{eqdim}^\mathcal{O}}}
\newcommand{\Sexc}{\ensuremath{\GSymbolFont{exc}^\mathcal{O}}}

\newcommand{\SGrSB}{\ensuremath{\GSymbolFont{Gr\hspace*{-0.25ex}SB}^\mathcal{O}}}
\newcommand{\Sgrsb}{\ensuremath{\GSymbolFont{gr\hspace*{-0.25ex}sb}^\mathcal{O}}}

\newcommand{\Shypp}{\ensuremath{\GSymbolFont{hypp}^\mathcal{O}}}

\newcommand{\SiCCDd}{\ensuremath{\GSymbolFont{i}_{\Gd}\GSymbolFont{CC}^\mathcal{O}}}
\newcommand{\SiCCDone}{\ensuremath{\GSymbolFont{i}_{1}\GSymbolFont{CC}^\mathcal{O}}}
\newcommand{\SiCCDtwo}{\ensuremath{\GSymbolFont{i}_{2}\GSymbolFont{CC}^\mathcal{O}}}
\newcommand{\SiCCDzero}{\ensuremath{\GSymbolFont{i}_{0}\GSymbolFont{CC}^\mathcal{O}}}
\newcommand{\Sinpart}{\ensuremath{\GSymbolFont{inpart}^\mathcal{O}}}
\newcommand{\Sintersect}{\ensuremath{\GSymbolFont{intsect}^\mathcal{O}}}
\newcommand{\Sintersectn}{\ensuremath{\GSymbolFont{intsect_{n}}^\mathcal{O}}}

\newcommand{\SLDE}{\ensuremath{\GSymbolFont{LDE}^\mathcal{O}}}

\newcommand{\Scont}{\ensuremath{\GSymbolFont{cont}^\mathcal{O}}}
\newcommand{\Sstrictsb}{\ensuremath{\GSymbolFont{strictsb}^\mathcal{O}}}
\newcommand{\Sweaksb}{\ensuremath{\GSymbolFont{weaksb}^\mathcal{O}}}

%\newcommand{\SkCCDd}{\ensuremath{\GSymbolFont{k}_{\Gd}\GSymbolFont{CC}^\mathcal{O}}}
%\newcommand{\SkCCDone}{\ensuremath{\GSymbolFont{k}_{1}\GSymbolFont{CC}^\mathcal{O}}}
%\newcommand{\SkCCDtwo}{\ensuremath{\GSymbolFont{k}_{2}\GSymbolFont{CC}^\mathcal{O}}}
\newcommand{\SkCCDzero}{\ensuremath{\GSymbolFont{k}_{0}\GSymbolFont{CC}^\mathcal{O}}}

%\newcommand{\SlCCDd}{\ensuremath{\GSymbolFont{l}_{\Gd}\GSymbolFont{CC}^\mathcal{O}}}
\newcommand{\SlCCDone}{\ensuremath{\GSymbolFont{l}_{1}\GSymbolFont{CC}^\mathcal{O}}}
%\newcommand{\SlCCDtwo}{\ensuremath{\GSymbolFont{l}_{2}\GSymbolFont{CC}^\mathcal{O}}}
%\newcommand{\SlCCDzero}{\ensuremath{\GSymbolFont{l}_{0}\GSymbolFont{CC}^\mathcal{O}}}

\newcommand{\SnCCDd}{\ensuremath{\GSymbolFont{n}_{\Gd}\GSymbolFont{CC}^\mathcal{O}}}
\newcommand{\SnCCDone}{\ensuremath{\GSymbolFont{n}_{1}\GSymbolFont{CC}^\mathcal{O}}}
\newcommand{\SnCCDtwo}{\ensuremath{\GSymbolFont{n}_{2}\GSymbolFont{CC}^\mathcal{O}}}
\newcommand{\SnCCDzero}{\ensuremath{\GSymbolFont{n}_{0}\GSymbolFont{CC}^\mathcal{O}}}
%\newcommand{\SnminusiCCDd}{\ensuremath{\GSymbolFont{(n-i)}_{\Gd}\GSymbolFont{CC}^\mathcal{O}}}
\newcommand{\SnminusiCCDone}{\ensuremath{\GSymbolFont{(n-i)}_{1}\GSymbolFont{CC}^\mathcal{O}}}
%\newcommand{\SnminusiCCDtwo}{\ensuremath{\GSymbolFont{(n-i)}_{2}\GSymbolFont{CC}^\mathcal{O}}}
\newcommand{\SnminusiCCDzero}{\ensuremath{\GSymbolFont{(n-i)}_{0}\GSymbolFont{CC}^\mathcal{O}}}


\newcommand{\SOrd}{\ensuremath{\GSymbolFont{Ord}^\mathcal{O}}}
\newcommand{\SoneCCDd}{\ensuremath{\GSymbolFont{1}_{\Gd}\GSymbolFont{CC}^\mathcal{O}}}
\newcommand{\SoneCCDone}{\ensuremath{\GSymbolFont{1}_{1}\GSymbolFont{CC}^\mathcal{O}}}
\newcommand{\SoneCCDtwo}{\ensuremath{\GSymbolFont{1}_{2}\GSymbolFont{CC}^\mathcal{O}}}
\newcommand{\SoneCCDzero}{\ensuremath{\GSymbolFont{1}_{0}\GSymbolFont{CC}^\mathcal{O}}}
\newcommand{\SoneD}{\ensuremath{\GSymbolFont{1D}^\mathcal{O}}}
\newcommand{\SoneDB}{\ensuremath{\GSymbolFont{1\AuxD B}^\mathcal{O}}}
\newcommand{\SoneDC}{\ensuremath{\GSymbolFont{1DC}^\mathcal{O}}}
\newcommand{\SoneDE}{\ensuremath{\GSymbolFont{1\AuxD E}^\mathcal{O}}}
\newcommand{\Sonedb}{\ensuremath{\GSymbolFont{1db}^\mathcal{O}}}
\newcommand{\Sonecont}{\ensuremath{\GSymbolFont{1cont}^\mathcal{O}}}
\newcommand{\Sonedircomp}{\ensuremath{\GSymbolFont{1dircomp}^\mathcal{O}}}
\newcommand{\Sonedhypp}{\ensuremath{\GSymbolFont{1dhypp}^\mathcal{O}}}

\newcommand{\Spartition}{\ensuremath{\GSymbolFont{partition}^\mathcal{O}}}
\newcommand{\Spartitioni}{\ensuremath{\GSymbolFont{partition_{i}}^\mathcal{O}}}
\newcommand{\Spartitionn}{\ensuremath{\GSymbolFont{partition_{n}}^\mathcal{O}}}

\newcommand{\SReg}{\ensuremath{\GSymbolFont{SReg}^\mathcal{O}}}
\newcommand{\Srelcompl}{\ensuremath{\GSymbolFont{rcompl}^\mathcal{O}}}
\newcommand{\Srelcompln}{\ensuremath{\GSymbolFont{rcompl_{n}}^\mathcal{O}}}

\newcommand{\SSB}{\ensuremath{\GSymbolFont{S\hspace*{-0.25ex}B}^\mathcal{O}}}
\newcommand{\SSReg}{\ensuremath{\GSymbolFont{S\hspace*{-0.25ex}Reg}^\mathcal{O}}}
\newcommand{\Ssb}{\ensuremath{\GSymbolFont{sb}^\mathcal{O}}}
\newcommand{\Sscoinc}{\ensuremath{\GSymbolFont{scoinc}^\mathcal{O}}}
\newcommand{\Ssov}{\ensuremath{\GSymbolFont{sov}^\mathcal{O}}}
\newcommand{\Sspart}{\ensuremath{\GSymbolFont{spart}^\mathcal{O}}}
\newcommand{\Ssppart}{\ensuremath{\GSymbolFont{sppart}^\mathcal{O}}}
\newcommand{\Ssum}{\ensuremath{\GSymbolFont{sum}^\mathcal{O}}}
\newcommand{\Ssumi}{\ensuremath{\GSymbolFont{sum_{i}}^\mathcal{O}}}
\newcommand{\Ssumn}{\ensuremath{\GSymbolFont{sum_{n}}^\mathcal{O}}}

\newcommand{\STop}{\ensuremath{\GSymbolFont{Top}^\mathcal{O}}}
\newcommand{\Stangpart}{\ensuremath{\GSymbolFont{tangpart}^\mathcal{O}}}
\newcommand{\StwoD}{\ensuremath{\GSymbolFont{2D}^\mathcal{O}}}
\newcommand{\StwoDB}{\ensuremath{\GSymbolFont{2\AuxD B}^\mathcal{O}}}
\newcommand{\StwoDC}{\ensuremath{\GSymbolFont{2DC}^\mathcal{O}}}
\newcommand{\StwoDE}{\ensuremath{\GSymbolFont{2\AuxD E}^\mathcal{O}}}
\newcommand{\Stwodb}{\ensuremath{\GSymbolFont{2db}^\mathcal{O}}}
\newcommand{\Stwodhypp}{\ensuremath{\GSymbolFont{2dhypp}^\mathcal{O}}}

\newcommand{\SzeroD}{\ensuremath{\GSymbolFont{0D}^\mathcal{O}}}
\newcommand{\SzeroDB}{\ensuremath{\GSymbolFont{0\AuxD B}^\mathcal{O}}}
\newcommand{\SzeroDC}{\ensuremath{\GSymbolFont{0DC}^\mathcal{O}}}
\newcommand{\SzeroDE}{\ensuremath{\GSymbolFont{0\AuxD E}^\mathcal{O}}}
\newcommand{\Szerocont}{\ensuremath{\GSymbolFont{0cont}^\mathcal{O}}}
\newcommand{\Szerodb}{\ensuremath{\GSymbolFont{0db}^\mathcal{O}}}
\newcommand{\Szerodhypp}{\ensuremath{\GSymbolFont{0dhypp}^\mathcal{O}}}

\newcommand{\Smaxcon}{\ensuremath{\GSymbolFont{maxcon}^\mathcal{O}}}
\newcommand{\Sloczerodc}{\ensuremath{\GSymbolFont{loc0dc}^\mathcal{O}}}
\newcommand{\Sloconedc}{\ensuremath{\GSymbolFont{loc1dc}^\mathcal{O}}}

% ****************************************************************************************************
% If you like the classicthesis, then I would appreciate a postcard.
% My address can be found in the file ClassicThesis.pdf. A collection
% of the postcards I received so far is available online at
% http://postcards.miede.de
% ****************************************************************************************************


% ****************************************************************************************************
% 0. Set the encoding of your files. UTF-8 is the only sensible encoding nowadays. If you can't read
% äöüßáéçèê∂åëæƒÏ€ then change the encoding setting in your editor, not the line below. If your editor
% does not support utf8 use another editor!
% ****************************************************************************************************
\PassOptionsToPackage{utf8}{inputenc}
  \usepackage{inputenc}

\PassOptionsToPackage{T1}{fontenc} % T2A for cyrillics
  \usepackage{fontenc}


% ****************************************************************************************************
% 1. Configure classicthesis for your needs here, e.g., remove "drafting" below
% in order to deactivate the time-stamp on the pages
% (see ClassicThesis.pdf for more information):
% ****************************************************************************************************
\PassOptionsToPackage{
  drafting=true,    % print version information on the bottom of the pages
  tocaligned=false, % the left column of the toc will be aligned (no indentation)
  dottedtoc=false,  % page numbers in ToC flushed right
  eulerchapternumbers=true, % use AMS Euler for chapter font (otherwise Palatino)
  linedheaders=false,       % chaper headers will have line above and beneath
  floatperchapter=true,     % numbering per chapter for all floats (i.e., Figure 1.1)
  eulermath=false,  % use awesome Euler fonts for mathematical formulae (only with pdfLaTeX)
  beramono=true,    % toggle a nice monospaced font (w/ bold)
  palatino=true,    % deactivate standard font for loading another one, see the last section at the end of this file for suggestions
  style=classicthesis % classicthesis, arsclassica
}{classicthesis}


% ****************************************************************************************************
% 2. Personal data and user ad-hoc commands (insert your own data here)
% ****************************************************************************************************
\newcommand{\myTitle}{Zur Theorie der ordinären Entitäten des Brentanoraumes\xspace}
% \newcommand{\mySubtitle}{Ein topologischer Interpretationsansatz\xspace}
\newcommand{\mySubtitle}{Ein repräsentantenbasierter Interpretationsansatz\xspace}
\newcommand{\myDegree}{Bachelor of science\xspace}
\newcommand{\myName}{Bärbel Hanle\xspace}
\newcommand{\mybirthday}{06.02.1982}
\newcommand{\mybirthtown}{Stuttgart}
\newcommand{\mybirthcountry}{Deutschland}
\newcommand{\myProf}{Dr. Frank Loebe\xspace}
\newcommand{\myOtherProf}{Dr. habil. Ringo Baumann\xspace}
\newcommand{\mySupervisor}{\xspace}
\newcommand{\myFaculty}{Institut für Mathematik und Informatik\xspace}
\newcommand{\myDepartment}{\xspace}
\newcommand{\myUni}{Universität Leipzig\xspace}
\newcommand{\myLocation}{Leipzig\xspace}
\newcommand{\myTime}{10.05.2022\xspace} 
\newcommand{\myVersion}{\classicthesis}

% ********************************************************************
% Setup, finetuning, and useful commands
% ********************************************************************
\providecommand{\mLyX}{L\kern-.1667em\lower.25em\hbox{Y}\kelastrn-.125emX\@}
\newcommand{\ie}{i.\,e.}
\newcommand{\Ie}{I.\,e.}
\newcommand{\eg}{e.\,g.}
\newcommand{\Eg}{E.\,g.}
% ****************************************************************************************************


% ****************************************************************************************************
% 3. Loading some handy packages
% ****************************************************************************************************
% ********************************************************************
% Packages with options that might require adjustments
% ********************************************************************
\PassOptionsToPackage{american,ngerman}{babel} % change this to your language(s), main language last
% Spanish languages need extra options in order to work with this template
%\PassOptionsToPackage{spanish,es-lcroman}{babel}
    \usepackage{babel}

\usepackage{csquotes}
\PassOptionsToPackage{%
  %backend=biber,bibencoding=utf8, %instead of bibtex
  backend=bibtex8,bibencoding=ascii,%
  language=auto,%
  style=authoryear,dashed=false%
  %style=authoryear-comp, % Author 1999, 2010
  %bibstyle=authoryear,dashed=false, % dashed: substitute rep. author with ---
  sorting=nyt, % name, year, title
  maxbibnames=10, % default: 3, et al.
  %backref=true,%
  natbib=true % natbib compatibility mode (\citep and \citet still work)
}{biblatex}
    \usepackage{biblatex}

\PassOptionsToPackage{fleqn}{amsmath}       % math environments and more by the AMS
  \usepackage{amsmath}

% ********************************************************************
% General useful packages
% ********************************************************************
\usepackage{graphicx} %
\usepackage{scrhack} % fix warnings when using KOMA with listings package
\usepackage{xspace} % to get the spacing after macros right

%\usepackage{textcase}
\PassOptionsToPackage{printonlyused,smaller}{acronym}
  \usepackage{acronym} % nice macros for handling all acronyms in the thesis
  %\renewcommand{\bflabel}[1]{{#1}\hfill} % fix the list of acronyms --> no longer working
  %\renewcommand*{\acsfont}[1]{\textsc{#1}}
  %\renewcommand*{\aclabelfont}[1]{\acsfont{#1}}
  %\def\bflabel#1{{#1\hfill}}
  \def\bflabel#1{{\acsfont{#1}\hfill}}
  \def\aclabelfont#1{\acsfont{#1}}
  
  %\renewcommand{\acsfont}[1]{{\scshape \MakeTextLowercase{#1}}}
  
\PassOptionsToPackage{activate={true,nocompatibility},final,tracking=true,kerning=true,spacing=true,factor=1100,stretch=10,shrink=10,final}{microtype}%final-even in draft mode
\usepackage[]{microtype}
% ****************************************************************************************************
%\usepackage{pgfplots} % External TikZ/PGF support (thanks to Andreas Nautsch)
%\usetikzlibrary{external}
%\tikzexternalize[mode=list and make, prefix=ext-tikz/]
% ****************************************************************************************************

% ****************************************************************************************************
% 4. Setup floats: tables, (sub)figures, and captions
% ****************************************************************************************************
\usepackage{tabularx} % better tables
  \setlength{\extrarowheight}{3pt} % increase table row height
\newcommand{\tableheadline}[1]{\multicolumn{1}{l}{\spacedlowsmallcaps{#1}}}
\newcommand{\myfloatalign}{\centering} % to be used with each float for alignment
\usepackage{subfig}
% ****************************************************************************************************


% ****************************************************************************************************
% 5. Setup code listings
% ****************************************************************************************************
\usepackage{listings}
%\lstset{emph={trueIndex,root},emphstyle=\color{BlueViolet}}%\underbar} % for special keywords
\lstset{language=[LaTeX]Tex,%C++,
  morekeywords={PassOptionsToPackage,selectlanguage},
  keywordstyle=\color{RoyalBlue},%\bfseries,
  basicstyle=\small\ttfamily,
  %identifierstyle=\color{NavyBlue},
  commentstyle=\color{Green}\ttfamily,
  stringstyle=\rmfamily,
  numbers=none,%left,%
  numberstyle=\scriptsize,%\tiny
  stepnumber=5,
  numbersep=8pt,
  showstringspaces=false,
  breaklines=true,
  %frameround=ftff,
  %frame=single,
  belowcaptionskip=.75\baselineskip
  %frame=L
}
% ****************************************************************************************************




% ****************************************************************************************************
% 6. Last calls before the bar closes
% ****************************************************************************************************
% ********************************************************************
% Her Majesty herself
% ********************************************************************
\usepackage[dottedtoc]{classicthesis}

%%% MK: comment out hyperref related code to create files  
%%%  topologie-erweiterung-overview.pdf and 
%%%  topologie-grundlagen-overview.pdf
%%%  see README file


% ********************************************************************
% Fine-tune hyperreferences (hyperref should be called last)
% ********************************************************************
%%% \hypersetup{%
%%%  %draft, % hyperref's draft mode, for printing see below
%%%   colorlinks=true, linktocpage=true, pdfstartpage=3, pdfstartview=FitV,%
%%%  % uncomment the following line if you want to have black links (e.g., for printing)
%%%  %colorlinks=false, linktocpage=false, pdfstartpage=3, pdfstartview=FitV, pdfborder={0 0 0},%
%%%  breaklinks=true, pageanchor=true,%
%%%  pdfpagemode=UseNone, %
%%%  % pdfpagemode=UseOutlines,%
%%%  plainpages=false, bookmarksnumbered, bookmarksopen=true, bookmarksopenlevel=1,%
%%%  hypertexnames=true, pdfhighlight=/O,%nesting=true,%frenchlinks,%
%%%  urlcolor=CTurl, linkcolor=CTlink, citecolor=CTcitation, %pagecolor=RoyalBlue,%
%%%  %urlcolor=Black, linkcolor=Black, citecolor=Black, %pagecolor=Black,%
%%%  pdftitle={\myTitle},%
%%%  pdfauthor={\textcopyright\ \myName, \myUni, \myFaculty},%
%%%  pdfsubject={},%
%%%  pdfkeywords={},%
%%%  pdfcreator={pdfLaTeX},%
%%%  pdfproducer={LaTeX with hyperref and classicthesis}%
%%% }

%**************************************************************************
% Eigene Pakete
%**************************************************************************
\usepackage{amssymb} % für \varnothing
\usepackage{stmaryrd} % Widerspruchsblitz
\usepackage{amsthm} % Für Theoremumgebungen
  \usepackage{zref-perpage} % FL: zref-perpage is there for 
    % fixing a MikTeX problem with zref, which is loaded by mdframed;
	  % cf. https://github.com/ho-tex/zref/issues/14
\usepackage[framemethod=tikz]{mdframed} % Rand für Theoreme
\usepackage{multirow}
\usepackage{makecell} % Zeilenumburuch innerhalb einer Zelle
\usepackage{longtable} % Tabelle über mehrere Seiten´
\usepackage{subcaption}
% \usepackage[labelformat=parens,labelsep=quad,skip=3pt]{caption}
%\usepackage{graphicx}

%\usepackage{textcase}

% \usepackage[strict]{changepage} % Anpassung der Seitenränder



% ********************************************************************
% Setup autoreferences (hyperref and babel)
% ********************************************************************
% There are some issues regarding autorefnames
% http://www.tex.ac.uk/cgi-bin/texfaq2html?label=latexwords
% you have to redefine the macros for the
% language you use, e.g., american, ngerman
% (as chosen when loading babel/AtBeginDocument)
% ********************************************************************
\makeatletter
\@ifpackageloaded{babel}%
  {%
    \addto\extrasamerican{%
		  \renewcommand*{\figurename}{Fig.}%
      \renewcommand*{\figureautorefname}{Figure}%
      \renewcommand*{\tableautorefname}{Table}%
      \renewcommand*{\partautorefname}{Part}%
      \renewcommand*{\chapterautorefname}{Chapter}%
      \renewcommand*{\sectionautorefname}{Section}%
      \renewcommand*{\subsectionautorefname}{Section}%
      \renewcommand*{\subsubsectionautorefname}{Section}%
    }%
    \addto\extrasngerman{%
		  \renewcommand*{\figurename}{Abb.}%
      \renewcommand*{\paragraphautorefname}{Absatz}%
      \renewcommand*{\subparagraphautorefname}{Unterabsatz}%
      \renewcommand*{\footnoteautorefname}{Fu\"snote}%
      \renewcommand*{\FancyVerbLineautorefname}{Zeile}%
      \renewcommand*{\theoremautorefname}{Theorem}%
      \renewcommand*{\appendixautorefname}{Anhang}%
      \renewcommand*{\equationautorefname}{Gleichung}%
      \renewcommand*{\itemautorefname}{Punkt}%
    }%
      % Fix to getting autorefs for subfigures right (thanks to Belinda Vogt for changing the definition)
      \providecommand{\subfigureautorefname}{\figureautorefname}%
    }{\relax}
\makeatother


% ********************************************************************
% Development Stuff
% ********************************************************************
\listfiles
%\PassOptionsToPackage{l2tabu,orthodox,abort}{nag}
%  \usepackage{nag}
%\PassOptionsToPackage{warning, all}{onlyamsmath}
%  \usepackage{onlyamsmath}


% ****************************************************************************************************
% 7. Further adjustments (experimental)
% ****************************************************************************************************
% ********************************************************************
% Changing the text area
% ********************************************************************
%\areaset[current]{312pt}{761pt} % 686 (factor 2.2) + 33 head + 42 head \the\footskip
%\setlength{\marginparwidth}{7em}%
%\setlength{\marginparsep}{2em}%

% ********************************************************************
% Using different fonts
% ********************************************************************
%\usepackage[oldstylenums]{kpfonts} % oldstyle notextcomp
% \usepackage[osf]{libertine}
%\usepackage[light,condensed,math]{iwona}
%\renewcommand{\sfdefault}{iwona}
%\usepackage{lmodern} % <-- no osf support :-(
%\usepackage{cfr-lm} %
%\usepackage[urw-garamond]{mathdesign} <-- no osf support :-(
%\usepackage[default,osfigures]{opensans} % scale=0.95
%\usepackage[sfdefault]{FiraSans}
% \usepackage[opticals,mathlf]{MinionPro} % onlytext
% ********************************************************************
%\usepackage[largesc,osf]{newpxtext}
%\linespread{1.05} % a bit more for Palatino
% Used to fix these:
% https://bitbucket.org/amiede/classicthesis/issues/139/italics-in-pallatino-capitals-chapter
% https://bitbucket.org/amiede/classicthesis/issues/45/problema-testatine-su-classicthesis-style
% ********************************************************************
% ****************************************************************************************************

% ****** Custom
%\usepackage[disable]{todonotes}
\usepackage{todonotes}
\usepackage{cleveref}

%***************************************************************************
% Umgebungen
%***************************************************************************
\theoremstyle{definition}

\newtheorem{dfn}{Def.}[section]
\newtheorem{nota}[dfn]{Notation}
\newtheorem{konv}[dfn]{Konvention}

\newtheorem{satz}[dfn]{Satz}
\newtheorem{kor}[dfn]{Kor.}
\newtheorem{hyp}[dfn]{Hypothese}

\newtheorem{bsp}[dfn]{Bsp.}
\newtheorem{gegenbsp}[dfn]{Gegenbeispiel}
\newtheorem{bem}[dfn]{Bem.}
\newtheorem*{erin}{Erinnerung}
\newtheorem*{bew}{Bew}
\newtheorem*{bewidee}{Beweisidee}

\mdfdefinestyle{grey}{
    skipabove=5pt,
    skipbelow=5pt,
    innerbottommargin=12pt,
    %innerleftmargin=20pt
    innerrightmargin=20pt
    leftmargin=5pt,
    rightmargin=5pt,
    %linewidth=0.2pt,
    roundcorner=2pt,
    backgroundcolor=black!5,
    hidealllines=true,
    %needspace=60pt,
    aftersingleframe={\noindent},
}

\mdfdefinestyle{white}{
    skipabove=5pt,
    skipbelow=5pt,
    innerbottommargin=12pt,
    %innerleftmargin=20pt
    innerrightmargin=20pt
    leftmargin=5pt,
    rightmargin=5pt,
    %linewidth=0.2pt,
    roundcorner=2pt,
    %backgroundcolor=black!5,
    %hidealllines=true
    aftersingleframe={\noindent},
}

\mdfdefinestyle{beweis}{
    skipabove=0pt,
    skipbelow=5pt,
    innerbottommargin=12pt,
    %innerleftmargin=20pt
    innerrightmargin=20pt
    leftmargin=5pt,
    rightmargin=5pt,
    %linewidth=0.2pt,
    roundcorner=2pt,
    %backgroundcolor=black!5,
    %hidealllines=true
    linecolor=black!10,
%     rightline=false
%     bottomline=false
%     topline=false
    aftersingleframe={\noindent},
}

\surroundwithmdframed[style=grey]{dfn}
\surroundwithmdframed[style=grey]{nota}
\surroundwithmdframed[style=grey]{konv}
\surroundwithmdframed[style=grey]{satz}
\surroundwithmdframed[style=grey]{kor}
\surroundwithmdframed[style=grey]{hyp}
\surroundwithmdframed[style=white]{bsp}
\surroundwithmdframed[style=white]{gegenbsp}
\surroundwithmdframed[style=white]{bem}
\surroundwithmdframed[style=white]{erin}
\surroundwithmdframed[style=beweis]{bew}
\surroundwithmdframed[style=beweis]{bewidee}




\DeclareRobustCommand{\BSO}{\mathcal{BS^O}}

%--------------------------------------------

\newcommand{\thmemph}{\textbf}
\newcommand{\textemph}{\spacedlowsmallcaps}
%\renewcommand{\marginpar}[2][]{} % marginpars verstecken



%*****************************************************************************
% BS-Kommandos
%********************************************************************************
% 
% %%%%%%%%%%%%%%%%%%%%%%%%%%%%%%%%%%%%%%%%%%%%%%%%%%%%%%%%
% % OPTION uniformSymbols STARTS
% %%%%%%%%%%%%%%%%%%%%%%%%%%%%%%%%%%%%%%%%%%%%%%%%%%%%%%%%%%
% \DeclareOption{uniformSymbols}{%
% %
% \newcommand{\GSymbolFont}[1]{#1}
% %
% } %OPTION uniformSymbols ENDS HERE
% %%%%%%%%%%%%%%%%%%%%%%%%%%%%%%%%%%%%%%%%%%%%%%%%%%%
% %%%%%%%%%%%%%%%%%%%%%%%%%%%%%%%%%%%%%%%%%%%%%%%%%%% 
% 
% %%%%%%%%%%%%%%%%%%%%%%%%%%%%%%%%%%%%%%%%%%%%%%%%%%%%%%%%
% % OPTION mboxSymbols STARTS
% %%%%%%%%%%%%%%%%%%%%%%%%%%%%%%%%%%%%%%%%%%%%%%%%%%%%%%%%%%
% \DeclareOption{mboxSymbols}{%
% %
% \renewcommand{\GSymbolFont}[1]{\mbox{#1}}
% %
% } %OPTION mboxSymbols ENDS HERE
% %%%%%%%%%%%%%%%%%%%%%%%%%%%%%%%%%%%%%%%%%%%%%%%%%%%
% %%%%%%%%%%%%%%%%%%%%%%%%%%%%%%%%%%%%%%%%%%%%%%%%%%% 
% 
% %%%%%%%%%%%%%%%%%%%%%%%%%%%%%%%%%%%%%%%%%%%%%%%%%%%%%%%%
% % OPTION mboxSymbols STARTS
% %%%%%%%%%%%%%%%%%%%%%%%%%%%%%%%%%%%%%%%%%%%%%%%%%%%%%%%%%%
% \DeclareOption{italicSymbols}{%
% %
% \renewcommand{\GSymbolFont}[1]{\ensuremath{\mathit{#1}}}
% %
% } %OPTION mboxSymbols ENDS HERE
% %%%%%%%%%%%%%%%%%%%%%%%%%%%%%%%%%%%%%%%%%%%%%%%%%%%
% %%%%%%%%%%%%%%%%%%%%%%%%%%%%%%%%%%%%%%%%%%%%%%%%%%% 
% 
% % Execution of options
% \ExecuteOptions{manAxiomStyle,uniformSymbols}

\newcommand{\GSymbolFont}[1]{\ensuremath{\mathit{#1}}}
\newcommand{\AuxD}{D\hspace*{-0.25ex}}


\newcommand{\GC}{\ensuremath{\GSymbolFont{C}}}
\newcommand{\GCrossdDBn}{\ensuremath{\GSymbolFont{Cross\Gd DB_{n}}}}
\newcommand{\GCrossoneDBn}{\ensuremath{\GSymbolFont{Cross1DB_{n}}}}
\newcommand{\GCrosstwoDBn}{\ensuremath{\GSymbolFont{Cross2DB_{n}}}}
\newcommand{\GCrosszeroDBn}{\ensuremath{\GSymbolFont{Cross0DB_{n}}}}
\newcommand{\Gc}{\ensuremath{\GSymbolFont{c}}}
\newcommand{\Gcrossonedbn}{\ensuremath{\GSymbolFont{cross1db_{n}}}}
\newcommand{\Gcrosstwodbn}{\ensuremath{\GSymbolFont{cross2db_{n}}}}
%\newcommand{\Gcrosszerodb}{\ensuremath{\GSymbolFont{cross0db}}}
\newcommand{\Gcrosszerodbn}{\ensuremath{\GSymbolFont{cross0db_{n}}}}

\newcommand{\Gd}{\boldsymbol{\mathsf{d}}}
\newcommand{\Gone}{\boldsymbol{\mathsf{1}}}
\newcommand{\Gtwo}{\boldsymbol{\mathsf{2}}}
\newcommand{\Gthree}{\boldsymbol{\mathsf{3}}}
\newcommand{\GdD}{\ensuremath{\GSymbolFont{\Gd D}}}
\newcommand{\GdDB}{\ensuremath{\GSymbolFont{\Gd\AuxD B}}}
\newcommand{\GdDC}{\ensuremath{\GSymbolFont{\Gd\AuxD C}}}
\newcommand{\GdDE}{\ensuremath{\GSymbolFont{\Gd\AuxD E}}}
\newcommand{\Gddb}{\ensuremath{\GSymbolFont{\Gd db}}}
\newcommand{\Gddhypp}{\ensuremath{\GSymbolFont{\Gd dhypp}}}
\newcommand{\Gdmdhypp}{\ensuremath{\GSymbolFont{\Gdm dhypp}}}
\newcommand{\Gdircomp}{\ensuremath{\GSymbolFont{dircomp}}}
\newcommand{\Gddircomp}{\ensuremath{\GSymbolFont{\Gd dircomp}}}
\newcommand{\dircomp}{\ensuremath{\GSymbolFont{\Gdp dircomp}}}
\newcommand{\Gdp}{\boldsymbol{\mathsf{(d+1)}}}
\newcommand{\Gdm}{\boldsymbol{\mathsf{(d-1)}}}

\newcommand{\GExOrd}{\ensuremath{\GSymbolFont{ExOrd}}}
\newcommand{\Gequ}{\ensuremath{\GSymbolFont{equ}}}
\newcommand{\Geqdim}{\ensuremath{\GSymbolFont{eqdim}}}
\newcommand{\Gexc}{\ensuremath{\GSymbolFont{exc}}}

\newcommand{\GGrSB}{\ensuremath{\GSymbolFont{Gr\hspace*{-0.25ex}SB}}}
\newcommand{\Ggrsb}{\ensuremath{\GSymbolFont{gr\hspace*{-0.25ex}sb}}}

\newcommand{\Ghypp}{\ensuremath{\GSymbolFont{hypp}}}

\newcommand{\GiCCDd}{\ensuremath{\GSymbolFont{i}_{\Gd}\GSymbolFont{CC}}}
\newcommand{\GiCCDone}{\ensuremath{\GSymbolFont{i}_{1}\GSymbolFont{CC}}}
\newcommand{\GiCCDtwo}{\ensuremath{\GSymbolFont{i}_{2}\GSymbolFont{CC}}}
\newcommand{\GiCCDzero}{\ensuremath{\GSymbolFont{i}_{0}\GSymbolFont{CC}}}
\newcommand{\Ginpart}{\ensuremath{\GSymbolFont{inpart}}}
\newcommand{\Gintersect}{\ensuremath{\GSymbolFont{intsect}}}
\newcommand{\Gintersectn}{\ensuremath{\GSymbolFont{intsect_{n}}}}

\newcommand{\GLDE}{\ensuremath{\GSymbolFont{LDE}}}

\newcommand{\Gcont}{\ensuremath{\GSymbolFont{cont}}}
\newcommand{\Gstrictsb}{\ensuremath{\GSymbolFont{strictsb}}}
\newcommand{\Gweaksb}{\ensuremath{\GSymbolFont{weaksb}}}

%\newcommand{\GkCCDd}{\ensuremath{\GSymbolFont{k}_{\Gd}\GSymbolFont{CC}}}
%\newcommand{\GkCCDone}{\ensuremath{\GSymbolFont{k}_{1}\GSymbolFont{CC}}}
%\newcommand{\GkCCDtwo}{\ensuremath{\GSymbolFont{k}_{2}\GSymbolFont{CC}}}
\newcommand{\GkCCDzero}{\ensuremath{\GSymbolFont{k}_{0}\GSymbolFont{CC}}}

%\newcommand{\GlCCDd}{\ensuremath{\GSymbolFont{l}_{\Gd}\GSymbolFont{CC}}}
\newcommand{\GlCCDone}{\ensuremath{\GSymbolFont{l}_{1}\GSymbolFont{CC}}}
%\newcommand{\GlCCDtwo}{\ensuremath{\GSymbolFont{l}_{2}\GSymbolFont{CC}}}
%\newcommand{\GlCCDzero}{\ensuremath{\GSymbolFont{l}_{0}\GSymbolFont{CC}}}

\newcommand{\GnCCDd}{\ensuremath{\GSymbolFont{n}_{\Gd}\GSymbolFont{CC}}}
\newcommand{\GnCCDone}{\ensuremath{\GSymbolFont{n}_{1}\GSymbolFont{CC}}}
\newcommand{\GnCCDtwo}{\ensuremath{\GSymbolFont{n}_{2}\GSymbolFont{CC}}}
\newcommand{\GnCCDzero}{\ensuremath{\GSymbolFont{n}_{0}\GSymbolFont{CC}}}
%\newcommand{\GnminusiCCDd}{\ensuremath{\GSymbolFont{(n-i)}_{\Gd}\GSymbolFont{CC}}}
\newcommand{\GnminusiCCDone}{\ensuremath{\GSymbolFont{(n-i)}_{1}\GSymbolFont{CC}}}
%\newcommand{\GnminusiCCDtwo}{\ensuremath{\GSymbolFont{(n-i)}_{2}\GSymbolFont{CC}}}
\newcommand{\GnminusiCCDzero}{\ensuremath{\GSymbolFont{(n-i)}_{0}\GSymbolFont{CC}}}


\newcommand{\GOrd}{\ensuremath{\GSymbolFont{Ord}}}
\newcommand{\GoneCCDd}{\ensuremath{\GSymbolFont{1}_{\Gd}\GSymbolFont{CC}}}
\newcommand{\GoneCCDone}{\ensuremath{\GSymbolFont{1}_{1}\GSymbolFont{CC}}}
\newcommand{\GoneCCDtwo}{\ensuremath{\GSymbolFont{1}_{2}\GSymbolFont{CC}}}
\newcommand{\GoneCCDzero}{\ensuremath{\GSymbolFont{1}_{0}\GSymbolFont{CC}}}
\newcommand{\GoneD}{\ensuremath{\GSymbolFont{1D}}}
\newcommand{\GoneDB}{\ensuremath{\GSymbolFont{1\AuxD B}}}
\newcommand{\GoneDC}{\ensuremath{\GSymbolFont{1DC}}}
\newcommand{\GoneDE}{\ensuremath{\GSymbolFont{1\AuxD E}}}
\newcommand{\Gonedb}{\ensuremath{\GSymbolFont{1db}}}
\newcommand{\Gonecont}{\ensuremath{\GSymbolFont{1cont}}}
\newcommand{\Gonedircomp}{\ensuremath{\GSymbolFont{1dircomp}}}
\newcommand{\Gonedhypp}{\ensuremath{\GSymbolFont{1dhypp}}}

\newcommand{\Gpartition}{\ensuremath{\GSymbolFont{partition}}}
\newcommand{\Gpartitioni}{\ensuremath{\GSymbolFont{partition_{i}}}}
\newcommand{\Gpartitionn}{\ensuremath{\GSymbolFont{partition_{n}}}}

\newcommand{\GReg}{\ensuremath{\GSymbolFont{SReg}}}
\newcommand{\Grelcompl}{\ensuremath{\GSymbolFont{rcompl}}}
\newcommand{\Grelcompln}{\ensuremath{\GSymbolFont{rcompl_{n}}}}

\newcommand{\GSB}{\ensuremath{\GSymbolFont{S\hspace*{-0.25ex}B}}}
\newcommand{\GSReg}{\ensuremath{\GSymbolFont{S\hspace*{-0.25ex}Reg}}}
\newcommand{\Gsb}{\ensuremath{\GSymbolFont{sb}}}
\newcommand{\Gscoinc}{\ensuremath{\GSymbolFont{scoinc}}}
\newcommand{\Gsov}{\ensuremath{\GSymbolFont{sov}}}
\newcommand{\Gspart}{\ensuremath{\GSymbolFont{spart}}}
\newcommand{\Gsppart}{\ensuremath{\GSymbolFont{sppart}}}
\newcommand{\Gsum}{\ensuremath{\GSymbolFont{sum}}}
\newcommand{\Gsumi}{\ensuremath{\GSymbolFont{sum_{i}}}}
\newcommand{\Gsumn}{\ensuremath{\GSymbolFont{sum_{n}}}}

\newcommand{\GTop}{\ensuremath{\GSymbolFont{Top}}}
\newcommand{\Gtangpart}{\ensuremath{\GSymbolFont{tangpart}}}
\newcommand{\GtwoD}{\ensuremath{\GSymbolFont{2D}}}
\newcommand{\GtwoDB}{\ensuremath{\GSymbolFont{2\AuxD B}}}
\newcommand{\GtwoDC}{\ensuremath{\GSymbolFont{2DC}}}
\newcommand{\GtwoDE}{\ensuremath{\GSymbolFont{2\AuxD E}}}
\newcommand{\Gtwodb}{\ensuremath{\GSymbolFont{2db}}}
\newcommand{\Gtwodhypp}{\ensuremath{\GSymbolFont{2dhypp}}}

\newcommand{\GzeroD}{\ensuremath{\GSymbolFont{0D}}}
\newcommand{\GzeroDB}{\ensuremath{\GSymbolFont{0\AuxD B}}}
\newcommand{\GzeroDC}{\ensuremath{\GSymbolFont{0DC}}}
\newcommand{\GzeroDE}{\ensuremath{\GSymbolFont{0\AuxD E}}}
\newcommand{\Gzerocont}{\ensuremath{\GSymbolFont{0cont}}}
\newcommand{\Gzerodb}{\ensuremath{\GSymbolFont{0db}}}
\newcommand{\Gzerodhypp}{\ensuremath{\GSymbolFont{0dhypp}}}



%MISC
\newcommand{\bs}{\backslash}
\newcommand{\gap}{\\[0.1ex]\mbox{}}
\newcommand{\m}[1]{\ensuremath{\mathcal{#1}}}
%\newcommand{\MS}{\ensuremath{\mathrel{.}}}
\newcommand{\MS}{\ensuremath{\,.\,}}
\newcommand{\theoryBS}{\ensuremath{\mathcal{BS}}\xspace}
\newcommand{\theoryBSone}{\ensuremath{\mathcal{BS}_{\text{v}1}}\xspace}
\newcommand{\theoryBT}{\ensuremath{\mathcal{BT}}}
\newcommand{\theoryBTC}{\ensuremath{\mathcal{BT}^{\mathcal{C}}}}
\newcommand{\theoryBTR}{\ensuremath{\mathcal{BT}^{\mathcal{R}}}}
\newcommand{\trel}[1]{\textit{#1}}

%********************************************************
% Eigene Kommandos
%********************************************************

\newcommand{\theoryBSO}{\ensuremath{\mathcal{BS}^{\mathcal{O}}}}
%\newcommand{\strukt}{\ensuremath{{\mathcal{R}\text{-Struktur}}}}
\newcommand{\strukt}{$\mathcal{R}$-Struktur\xspace}
\newcommand{\rep}{\ensuremath{\mathcal{R}}}
\newcommand{\univ}{\ensuremath{\mathcal{U}}}
\newcommand{\R}{\ensuremath{\mathbb{R}}}
\newcommand{\N}{\ensuremath{\mathbb{N}}}

\newcommand{\offen}{\ensuremath{\mathcal{O}}}
\newcommand{\abg}{\ensuremath{\mathcal{C}}}
\newcommand{\einf}{\ensuremath{\mathcal{S}}}
\newcommand{\CO}{\ensuremath{\mathcal{CO}}}
\newcommand{\OC}{\ensuremath{\mathcal{OC}}}

\newcommand{\cl}{\ensuremath{\text{cl}}}
\newcommand{\op}{\ensuremath{\text{op}}}
\newcommand{\co}{\ensuremath{\text{co}}}
\newcommand{\oc}{\ensuremath{\text{oc}}}
\newcommand{\HP}{\ensuremath{\text{HP}}}
\newcommand{\rand}{\ensuremath{\partial}}
\newcommand{\ball}{\ensuremath{B}}

\newcommand{\Gdim}{\ensuremath{\text{dim}}}
\newcommand{\Gmaxcon}{\ensuremath{\GSymbolFont{maxcon}}}
\newcommand{\Gloczerodc}{\ensuremath{\GSymbolFont{loc0dc}}}
\newcommand{\Gloconedc}{\ensuremath{\GSymbolFont{loc1dc}}}

\newcommand{\deshalb}{\ensuremath{\rightarrow}}

%***************************************************************
% Struktursymbole
%***************************************************************

\newcommand{\SdDB}{\ensuremath{\GSymbolFont{\Gd\AuxD B}^\mathcal{O}}}
\newcommand{\SdDC}{\ensuremath{\GSymbolFont{\Gd\AuxD C}^\mathcal{O}}}
\newcommand{\SdDE}{\ensuremath{\GSymbolFont{\Gd\AuxD E}^\mathcal{O}}}
\newcommand{\Sddb}{\ensuremath{\GSymbolFont{\Gd db}^\mathcal{O}}}
\newcommand{\Sddhypp}{\ensuremath{\GSymbolFont{\Gd dhypp}^\mathcal{O}}}
\newcommand{\Sdmdhypp}{\ensuremath{\GSymbolFont{\Gdm dhypp}^\mathcal{O}}}
\newcommand{\Sdircomp}{\ensuremath{\GSymbolFont{dircomp}^\mathcal{O}}}
\newcommand{\Sddircomp}{\ensuremath{\GSymbolFont{\Gd dircomp}^\mathcal{O}}}

\newcommand{\SExOrd}{\ensuremath{\GSymbolFont{ExOrd}^\mathcal{O}}}
\newcommand{\Sequ}{\ensuremath{\GSymbolFont{equ}^\mathcal{O}}}
\newcommand{\Seqdim}{\ensuremath{\GSymbolFont{eqdim}^\mathcal{O}}}
\newcommand{\Sexc}{\ensuremath{\GSymbolFont{exc}^\mathcal{O}}}

\newcommand{\SGrSB}{\ensuremath{\GSymbolFont{Gr\hspace*{-0.25ex}SB}^\mathcal{O}}}
\newcommand{\Sgrsb}{\ensuremath{\GSymbolFont{gr\hspace*{-0.25ex}sb}^\mathcal{O}}}

\newcommand{\Shypp}{\ensuremath{\GSymbolFont{hypp}^\mathcal{O}}}

\newcommand{\SiCCDd}{\ensuremath{\GSymbolFont{i}_{\Gd}\GSymbolFont{CC}^\mathcal{O}}}
\newcommand{\SiCCDone}{\ensuremath{\GSymbolFont{i}_{1}\GSymbolFont{CC}^\mathcal{O}}}
\newcommand{\SiCCDtwo}{\ensuremath{\GSymbolFont{i}_{2}\GSymbolFont{CC}^\mathcal{O}}}
\newcommand{\SiCCDzero}{\ensuremath{\GSymbolFont{i}_{0}\GSymbolFont{CC}^\mathcal{O}}}
\newcommand{\Sinpart}{\ensuremath{\GSymbolFont{inpart}^\mathcal{O}}}
\newcommand{\Sintersect}{\ensuremath{\GSymbolFont{intsect}^\mathcal{O}}}
\newcommand{\Sintersectn}{\ensuremath{\GSymbolFont{intsect_{n}}^\mathcal{O}}}

\newcommand{\SLDE}{\ensuremath{\GSymbolFont{LDE}^\mathcal{O}}}

\newcommand{\Scont}{\ensuremath{\GSymbolFont{cont}^\mathcal{O}}}
\newcommand{\Sstrictsb}{\ensuremath{\GSymbolFont{strictsb}^\mathcal{O}}}
\newcommand{\Sweaksb}{\ensuremath{\GSymbolFont{weaksb}^\mathcal{O}}}

%\newcommand{\SkCCDd}{\ensuremath{\GSymbolFont{k}_{\Gd}\GSymbolFont{CC}^\mathcal{O}}}
%\newcommand{\SkCCDone}{\ensuremath{\GSymbolFont{k}_{1}\GSymbolFont{CC}^\mathcal{O}}}
%\newcommand{\SkCCDtwo}{\ensuremath{\GSymbolFont{k}_{2}\GSymbolFont{CC}^\mathcal{O}}}
\newcommand{\SkCCDzero}{\ensuremath{\GSymbolFont{k}_{0}\GSymbolFont{CC}^\mathcal{O}}}

%\newcommand{\SlCCDd}{\ensuremath{\GSymbolFont{l}_{\Gd}\GSymbolFont{CC}^\mathcal{O}}}
\newcommand{\SlCCDone}{\ensuremath{\GSymbolFont{l}_{1}\GSymbolFont{CC}^\mathcal{O}}}
%\newcommand{\SlCCDtwo}{\ensuremath{\GSymbolFont{l}_{2}\GSymbolFont{CC}^\mathcal{O}}}
%\newcommand{\SlCCDzero}{\ensuremath{\GSymbolFont{l}_{0}\GSymbolFont{CC}^\mathcal{O}}}

\newcommand{\SnCCDd}{\ensuremath{\GSymbolFont{n}_{\Gd}\GSymbolFont{CC}^\mathcal{O}}}
\newcommand{\SnCCDone}{\ensuremath{\GSymbolFont{n}_{1}\GSymbolFont{CC}^\mathcal{O}}}
\newcommand{\SnCCDtwo}{\ensuremath{\GSymbolFont{n}_{2}\GSymbolFont{CC}^\mathcal{O}}}
\newcommand{\SnCCDzero}{\ensuremath{\GSymbolFont{n}_{0}\GSymbolFont{CC}^\mathcal{O}}}
%\newcommand{\SnminusiCCDd}{\ensuremath{\GSymbolFont{(n-i)}_{\Gd}\GSymbolFont{CC}^\mathcal{O}}}
\newcommand{\SnminusiCCDone}{\ensuremath{\GSymbolFont{(n-i)}_{1}\GSymbolFont{CC}^\mathcal{O}}}
%\newcommand{\SnminusiCCDtwo}{\ensuremath{\GSymbolFont{(n-i)}_{2}\GSymbolFont{CC}^\mathcal{O}}}
\newcommand{\SnminusiCCDzero}{\ensuremath{\GSymbolFont{(n-i)}_{0}\GSymbolFont{CC}^\mathcal{O}}}


\newcommand{\SOrd}{\ensuremath{\GSymbolFont{Ord}^\mathcal{O}}}
\newcommand{\SoneCCDd}{\ensuremath{\GSymbolFont{1}_{\Gd}\GSymbolFont{CC}^\mathcal{O}}}
\newcommand{\SoneCCDone}{\ensuremath{\GSymbolFont{1}_{1}\GSymbolFont{CC}^\mathcal{O}}}
\newcommand{\SoneCCDtwo}{\ensuremath{\GSymbolFont{1}_{2}\GSymbolFont{CC}^\mathcal{O}}}
\newcommand{\SoneCCDzero}{\ensuremath{\GSymbolFont{1}_{0}\GSymbolFont{CC}^\mathcal{O}}}
\newcommand{\SoneD}{\ensuremath{\GSymbolFont{1D}^\mathcal{O}}}
\newcommand{\SoneDB}{\ensuremath{\GSymbolFont{1\AuxD B}^\mathcal{O}}}
\newcommand{\SoneDC}{\ensuremath{\GSymbolFont{1DC}^\mathcal{O}}}
\newcommand{\SoneDE}{\ensuremath{\GSymbolFont{1\AuxD E}^\mathcal{O}}}
\newcommand{\Sonedb}{\ensuremath{\GSymbolFont{1db}^\mathcal{O}}}
\newcommand{\Sonecont}{\ensuremath{\GSymbolFont{1cont}^\mathcal{O}}}
\newcommand{\Sonedircomp}{\ensuremath{\GSymbolFont{1dircomp}^\mathcal{O}}}
\newcommand{\Sonedhypp}{\ensuremath{\GSymbolFont{1dhypp}^\mathcal{O}}}

\newcommand{\Spartition}{\ensuremath{\GSymbolFont{partition}^\mathcal{O}}}
\newcommand{\Spartitioni}{\ensuremath{\GSymbolFont{partition_{i}}^\mathcal{O}}}
\newcommand{\Spartitionn}{\ensuremath{\GSymbolFont{partition_{n}}^\mathcal{O}}}

\newcommand{\SReg}{\ensuremath{\GSymbolFont{SReg}^\mathcal{O}}}
\newcommand{\Srelcompl}{\ensuremath{\GSymbolFont{rcompl}^\mathcal{O}}}
\newcommand{\Srelcompln}{\ensuremath{\GSymbolFont{rcompl_{n}}^\mathcal{O}}}

\newcommand{\SSB}{\ensuremath{\GSymbolFont{S\hspace*{-0.25ex}B}^\mathcal{O}}}
\newcommand{\SSReg}{\ensuremath{\GSymbolFont{S\hspace*{-0.25ex}Reg}^\mathcal{O}}}
\newcommand{\Ssb}{\ensuremath{\GSymbolFont{sb}^\mathcal{O}}}
\newcommand{\Sscoinc}{\ensuremath{\GSymbolFont{scoinc}^\mathcal{O}}}
\newcommand{\Ssov}{\ensuremath{\GSymbolFont{sov}^\mathcal{O}}}
\newcommand{\Sspart}{\ensuremath{\GSymbolFont{spart}^\mathcal{O}}}
\newcommand{\Ssppart}{\ensuremath{\GSymbolFont{sppart}^\mathcal{O}}}
\newcommand{\Ssum}{\ensuremath{\GSymbolFont{sum}^\mathcal{O}}}
\newcommand{\Ssumi}{\ensuremath{\GSymbolFont{sum_{i}}^\mathcal{O}}}
\newcommand{\Ssumn}{\ensuremath{\GSymbolFont{sum_{n}}^\mathcal{O}}}

\newcommand{\STop}{\ensuremath{\GSymbolFont{Top}^\mathcal{O}}}
\newcommand{\Stangpart}{\ensuremath{\GSymbolFont{tangpart}^\mathcal{O}}}
\newcommand{\StwoD}{\ensuremath{\GSymbolFont{2D}^\mathcal{O}}}
\newcommand{\StwoDB}{\ensuremath{\GSymbolFont{2\AuxD B}^\mathcal{O}}}
\newcommand{\StwoDC}{\ensuremath{\GSymbolFont{2DC}^\mathcal{O}}}
\newcommand{\StwoDE}{\ensuremath{\GSymbolFont{2\AuxD E}^\mathcal{O}}}
\newcommand{\Stwodb}{\ensuremath{\GSymbolFont{2db}^\mathcal{O}}}
\newcommand{\Stwodhypp}{\ensuremath{\GSymbolFont{2dhypp}^\mathcal{O}}}

\newcommand{\SzeroD}{\ensuremath{\GSymbolFont{0D}^\mathcal{O}}}
\newcommand{\SzeroDB}{\ensuremath{\GSymbolFont{0\AuxD B}^\mathcal{O}}}
\newcommand{\SzeroDC}{\ensuremath{\GSymbolFont{0DC}^\mathcal{O}}}
\newcommand{\SzeroDE}{\ensuremath{\GSymbolFont{0\AuxD E}^\mathcal{O}}}
\newcommand{\Szerocont}{\ensuremath{\GSymbolFont{0cont}^\mathcal{O}}}
\newcommand{\Szerodb}{\ensuremath{\GSymbolFont{0db}^\mathcal{O}}}
\newcommand{\Szerodhypp}{\ensuremath{\GSymbolFont{0dhypp}^\mathcal{O}}}

\newcommand{\Smaxcon}{\ensuremath{\GSymbolFont{maxcon}^\mathcal{O}}}
\newcommand{\Sloczerodc}{\ensuremath{\GSymbolFont{loc0dc}^\mathcal{O}}}
\newcommand{\Sloconedc}{\ensuremath{\GSymbolFont{loc1dc}^\mathcal{O}}}
