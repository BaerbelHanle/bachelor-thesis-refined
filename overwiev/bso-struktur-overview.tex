\documentclass{article}

\usepackage[a4paper,left=3cm,right=3cm,top=3cm,bottom=3cm,bindingoffset=0mm]{geometry}

\usepackage[utf8]{inputenc}
\usepackage[ngerman]{babel}
\usepackage[manAxiomStyle,uniformSymbols]{space-abbrevs}
\usepackage{xcolor}
\usepackage{amsmath,amssymb}
%\usepackage{fontenc}
\usepackage{xr} % external references
\externaldocument{thesis.tex}

\usepackage{fancyhdr}
\pagestyle{fancy}
\setlength{\headheight}{16pt}
\lhead{Übersicht \overviewnumber}
\chead{}
\rhead{\overviewname}
\cfoot{\thepage}



%%%%%%%%%%%%%%%%%%%%%%%%%%%%%%%%%%%%%%%%%%%%%%%%%%%%%%%%%%%%%%%%%%
% For external refences
%%%%%%%%%%%%%%%%%%%%%%%%%%%%%%%%%%%%%%%%%%%%%%%%%%%%%%%%%%%%%%%
% 
% \makeatletter
% \newcommand*{\addFileDependency}[1]{% argument=file name and extension
%   \typeout{(#1)}
%   \@addtofilelist{#1}
%   \IfFileExists{#1}{}{\typeout{No file #1.}}
% }
% \makeatother
% 
% \newcommand*{\myexternaldocument}[1]{%
%     \externaldocument{#1}%
%     \addFileDependency{#1.tex}%
%     \addFileDependency{#1.aux}%
% }
% 
% \myexternaldocument{thesis.tex}






% -------------------------------------------------------------------------
% §§§§§§§§§§§§§§§§§§§§§§§§§§§§§    Symbole    §§§§§§§§§§§§§§§§§§§§§§§§§§§§§
% -------------------------------------------------------------------------


% -------------------------------------------------------------------------
% --    Allgemein    ------------------------------------------------------
% -------------------------------------------------------------------------

\newcommand{\N}{\ensuremath{\mathbb{N}}}
\newcommand{\R}{\ensuremath{\mathbb{R}}}
\newcommand{\C}{\ensuremath{\mathbb{C}}}


% -------------------------------------------------------------------------
% --    topologie Grundlagen    -------------------------------------------
% -------------------------------------------------------------------------

\newcommand{\offen}{\ensuremath{\mathcal{O}}}
\newcommand{\abg}{\ensuremath{\mathcal{C}}}

\newcommand{\op}{\ensuremath{\text{op}}}
\newcommand{\cl}{\ensuremath{\text{cl}}}
\newcommand{\rand}{\ensuremath{\partial}}
\newcommand{\HP}{\ensuremath{\text{HP}}}
\newcommand{\ball}{\ensuremath{B}}


% -------------------------------------------------------------------------
% --    topologie Erweiterung    ------------------------------------------
% -------------------------------------------------------------------------

\newcommand{\CO}{\ensuremath{\mathcal{CO}}}
\newcommand{\OC}{\ensuremath{\mathcal{OC}}}
\newcommand{\einf}{\ensuremath{\mathcal{S}}}

\newcommand{\oc}{\ensuremath{\text{oc}}}
\newcommand{\co}{\ensuremath{\text{co}}}
\newcommand{\arand}{\ensuremath{\delta}}


% -------------------------------------------------------------------------
% --    BS    -------------------------------------------------------------
% -------------------------------------------------------------------------

%\newcommand{\GSymbolFont}[1]{\ensuremath{\mathit{#1}}}
% \newcommand{\AuxD}{D\hspace*{-0.25ex}}
% 
% \newcommand{\theoryBS}{\ensuremath{\mathcal{BS}}\xspace}
% \newcommand{\theoryBT}{\ensuremath{\mathcal{BT}}}
% \newcommand{\theoryBTC}{\ensuremath{\mathcal{BT}^{\mathcal{C}}}}
% \newcommand{\theoryBTR}{\ensuremath{\mathcal{BT}^{\mathcal{R}}}}
% 
% \newcommand{\GC}{\ensuremath{\GSymbolFont{C}}}
% \newcommand{\Gc}{\ensuremath{\GSymbolFont{c}}}
% 
% \newcommand{\Gd}{\boldsymbol{\mathsf{d}}}
% \newcommand{\Gone}{\boldsymbol{\mathsf{1}}}
% \newcommand{\Gtwo}{\boldsymbol{\mathsf{2}}}
% \newcommand{\Gthree}{\boldsymbol{\mathsf{3}}}
% \newcommand{\GdD}{\ensuremath{\GSymbolFont{\Gd D}}}
% \newcommand{\GdDB}{\ensuremath{\GSymbolFont{\Gd\AuxD B}}}
% \newcommand{\GdDC}{\ensuremath{\GSymbolFont{\Gd\AuxD C}}}
% \newcommand{\GdDE}{\ensuremath{\GSymbolFont{\Gd\AuxD E}}}
% \newcommand{\Gddb}{\ensuremath{\GSymbolFont{\Gd db}}}
% \newcommand{\Gddhypp}{\ensuremath{\GSymbolFont{\Gd dhypp}}}
% \newcommand{\Gdmdhypp}{\ensuremath{\GSymbolFont{\Gdm dhypp}}}
% \newcommand{\Gdircomp}{\ensuremath{\GSymbolFont{dircomp}}}
% \newcommand{\Gddircomp}{\ensuremath{\GSymbolFont{\Gd dircomp}}}
% \newcommand{\dircomp}{\ensuremath{\GSymbolFont{\Gdp dircomp}}}
% \newcommand{\Gdp}{\boldsymbol{\mathsf{(d+1)}}}
% \newcommand{\Gdm}{\boldsymbol{\mathsf{(d-1)}}}
% 
% \newcommand{\GExOrd}{\ensuremath{\GSymbolFont{ExOrd}}}
% \newcommand{\Gequ}{\ensuremath{\GSymbolFont{equ}}}
% \newcommand{\Geqdim}{\ensuremath{\GSymbolFont{eqdim}}}
% \newcommand{\Gexc}{\ensuremath{\GSymbolFont{exc}}}
% 
% \newcommand{\GGrSB}{\ensuremath{\GSymbolFont{Gr\hspace*{-0.25ex}SB}}}
% \newcommand{\Ggrsb}{\ensuremath{\GSymbolFont{gr\hspace*{-0.25ex}sb}}}
% 
% \newcommand{\Ghypp}{\ensuremath{\GSymbolFont{hypp}}}
% 
% \newcommand{\GiCCDd}{\ensuremath{\GSymbolFont{i}_{\Gd}\GSymbolFont{CC}}}
% \newcommand{\GiCCDone}{\ensuremath{\GSymbolFont{i}_{1}\GSymbolFont{CC}}}
% \newcommand{\GiCCDtwo}{\ensuremath{\GSymbolFont{i}_{2}\GSymbolFont{CC}}}
% \newcommand{\GiCCDzero}{\ensuremath{\GSymbolFont{i}_{0}\GSymbolFont{CC}}}
% \newcommand{\Ginpart}{\ensuremath{\GSymbolFont{inpart}}}
% \newcommand{\Gintersect}{\ensuremath{\GSymbolFont{intsect}}}
% \newcommand{\Gintersectn}{\ensuremath{\GSymbolFont{intsect_{n}}}}
% 
% \newcommand{\GLDE}{\ensuremath{\GSymbolFont{LDE}}}
% 
% \newcommand{\Gcont}{\ensuremath{\GSymbolFont{cont}}}
% \newcommand{\Gstrictsb}{\ensuremath{\GSymbolFont{strictsb}}}
% \newcommand{\Gweaksb}{\ensuremath{\GSymbolFont{weaksb}}}
% 
% \newcommand{\GkCCDzero}{\ensuremath{\GSymbolFont{k}_{0}\GSymbolFont{CC}}}
% 
% \newcommand{\GlCCDone}{\ensuremath{\GSymbolFont{l}_{1}\GSymbolFont{CC}}}
% 
% \newcommand{\GnCCDd}{\ensuremath{\GSymbolFont{n}_{\Gd}\GSymbolFont{CC}}}
% \newcommand{\GnCCDone}{\ensuremath{\GSymbolFont{n}_{1}\GSymbolFont{CC}}}
% \newcommand{\GnCCDtwo}{\ensuremath{\GSymbolFont{n}_{2}\GSymbolFont{CC}}}
% \newcommand{\GnCCDzero}{\ensuremath{\GSymbolFont{n}_{0}\GSymbolFont{CC}}}
% \newcommand{\GnminusiCCDone}{\ensuremath{\GSymbolFont{(n-i)}_{1}\GSymbolFont{CC}}}
% \newcommand{\GnminusiCCDzero}{\ensuremath{\GSymbolFont{(n-i)}_{0}\GSymbolFont{CC}}}
% 
% \newcommand{\GOrd}{\ensuremath{\GSymbolFont{Ord}}}
% \newcommand{\GoneCCDd}{\ensuremath{\GSymbolFont{1}_{\Gd}\GSymbolFont{CC}}}
% \newcommand{\GoneCCDone}{\ensuremath{\GSymbolFont{1}_{1}\GSymbolFont{CC}}}
% \newcommand{\GoneCCDtwo}{\ensuremath{\GSymbolFont{1}_{2}\GSymbolFont{CC}}}
% \newcommand{\GoneCCDzero}{\ensuremath{\GSymbolFont{1}_{0}\GSymbolFont{CC}}}
% \newcommand{\GoneD}{\ensuremath{\GSymbolFont{1D}}}
% \newcommand{\GoneDB}{\ensuremath{\GSymbolFont{1\AuxD B}}}
% \newcommand{\GoneDC}{\ensuremath{\GSymbolFont{1DC}}}
% \newcommand{\GoneDE}{\ensuremath{\GSymbolFont{1\AuxD E}}}
% \newcommand{\Gonedb}{\ensuremath{\GSymbolFont{1db}}}
% \newcommand{\Gonecont}{\ensuremath{\GSymbolFont{1cont}}}
% \newcommand{\Gonedircomp}{\ensuremath{\GSymbolFont{1dircomp}}}
% \newcommand{\Gonedhypp}{\ensuremath{\GSymbolFont{1dhypp}}}
% 
% \newcommand{\Gpartition}{\ensuremath{\GSymbolFont{partition}}}
% \newcommand{\Gpartitioni}{\ensuremath{\GSymbolFont{partition_{i}}}}
% \newcommand{\Gpartitionn}{\ensuremath{\GSymbolFont{partition_{n}}}}
% 
% \newcommand{\GReg}{\ensuremath{\GSymbolFont{SReg}}}
% \newcommand{\Grelcompl}{\ensuremath{\GSymbolFont{rcompl}}}
% \newcommand{\Grelcompln}{\ensuremath{\GSymbolFont{rcompl_{n}}}}
% 
% \newcommand{\GSB}{\ensuremath{\GSymbolFont{S\hspace*{-0.25ex}B}}}
% \newcommand{\GSReg}{\ensuremath{\GSymbolFont{S\hspace*{-0.25ex}Reg}}}
% \newcommand{\Gsb}{\ensuremath{\GSymbolFont{sb}}}
% \newcommand{\Gscoinc}{\ensuremath{\GSymbolFont{scoinc}}}
% \newcommand{\Gsov}{\ensuremath{\GSymbolFont{sov}}}
% \newcommand{\Gspart}{\ensuremath{\GSymbolFont{spart}}}
% \newcommand{\Gsppart}{\ensuremath{\GSymbolFont{sppart}}}
% \newcommand{\Gsum}{\ensuremath{\GSymbolFont{sum}}}
% \newcommand{\Gsumi}{\ensuremath{\GSymbolFont{sum_{i}}}}
% \newcommand{\Gsumn}{\ensuremath{\GSymbolFont{sum_{n}}}}
% 
% \newcommand{\GTop}{\ensuremath{\GSymbolFont{Top}}}
% \newcommand{\Gtangpart}{\ensuremath{\GSymbolFont{tangpart}}}
% \newcommand{\GtwoD}{\ensuremath{\GSymbolFont{2D}}}
% \newcommand{\GtwoDB}{\ensuremath{\GSymbolFont{2\AuxD B}}}
% \newcommand{\GtwoDC}{\ensuremath{\GSymbolFont{2DC}}}
% \newcommand{\GtwoDE}{\ensuremath{\GSymbolFont{2\AuxD E}}}
% \newcommand{\Gtwodb}{\ensuremath{\GSymbolFont{2db}}}
% \newcommand{\Gtwodhypp}{\ensuremath{\GSymbolFont{2dhypp}}}
% 
% \newcommand{\GzeroD}{\ensuremath{\GSymbolFont{0D}}}
% \newcommand{\GzeroDB}{\ensuremath{\GSymbolFont{0\AuxD B}}}
% \newcommand{\GzeroDC}{\ensuremath{\GSymbolFont{0DC}}}
% \newcommand{\GzeroDE}{\ensuremath{\GSymbolFont{0\AuxD E}}}
% \newcommand{\Gzerocont}{\ensuremath{\GSymbolFont{0cont}}}
% \newcommand{\Gzerodb}{\ensuremath{\GSymbolFont{0db}}}
% \newcommand{\Gzerodhypp}{\ensuremath{\GSymbolFont{0dhypp}}}


% -------------------------------------------------------------------------
% --    BSO    ------------------------------------------------------------
% -------------------------------------------------------------------------

\newcommand{\theoryBSO}{\ensuremath{\mathcal{BS}^{\mathcal{O}}}}
%\newcommand{\strukt}{\ensuremath{{\mathcal{R}\text{-Struktur}}}}
\newcommand{\strukt}{$\mathcal{R}$-Struktur\ }
\newcommand{\rep}{\ensuremath{\mathcal{R}}}
\newcommand{\univ}{\ensuremath{\mathcal{U}}}

\newcommand{\Gdim}{\ensuremath{\text{dim}}}





% -------------------------------------------------------------------------
% §§§§§§§§§§§§§§§§§§§§§§§§§§§§    Umgebungen    §§§§§§§§§§§§§§§§§§§§§§§§§§§
% -------------------------------------------------------------------------
% 
% %%%%%%%%%%%%%%%%%%%%%%%%%%%%%%%%%%%%%%%%%%%%%%%%%%%%%%%%%
% % AXIOMS and DEFINITIONS, numbered
% 
% % counters for enumerating axioms
% % CenumAxLast?-counters are for storing last values,
% %   which allows for continuous counting through the document
% \newcounter{CenumAxA}      %axiom
% \newcounter{CenumAxB}      %basic symbol
% \newcounter{CenumAxC}      %consequence/corollary
% \newcounter{CenumAxD}      %definition
% \newcounter{CenumAxT}      %theorem
% \setcounter{CenumAxA}{1}
% \setcounter{CenumAxB}{1}
% \setcounter{CenumAxC}{1}
% \setcounter{CenumAxD}{1}
% \setcounter{CenumAxT}{1}
% \newcounter{CenumAxLastA}
% \newcounter{CenumAxLastB}
% \newcounter{CenumAxLastC}
% \newcounter{CenumAxLastD}
% \newcounter{CenumAxLastT}
% \setcounter{CenumAxLastA}{0}
% \setcounter{CenumAxLastB}{0}
% \setcounter{CenumAxLastC}{0}
% \setcounter{CenumAxLastD}{0}
% \setcounter{CenumAxLastT}{0}
% 
% % new for consistency proof and interpretation into IP
% %\newcounter{CenumAxCD}      %interpretation of BTC symbol
% %\newcounter{CenumAxIA}      %interpretation of axiom
% %\newcounter{CenumAxID}      %interpretation of definition
% %\newcounter{CenumAxIP}      %System IP (Vila)
% %\newcounter{CenumAxTr}      %translation/interpretation of IP formula
% %\setcounter{CenumAxCD}{1}
% %\setcounter{CenumAxIA}{1}
% %\setcounter{CenumAxID}{1}
% %\setcounter{CenumAxIP}{1}
% %\setcounter{CenumAxTr}{1}
% %\newcounter{CenumAxLastCD}
% %\newcounter{CenumAxLastIA}
% %\newcounter{CenumAxLastID}
% %\newcounter{CenumAxLastIP}
% %\newcounter{CenumAxLastTr}
% %\setcounter{CenumAxLastCD}{0}
% %\setcounter{CenumAxLastIA}{0}
% %\setcounter{CenumAxLastID}{0}
% %\setcounter{CenumAxLastIP}{0}
% %\setcounter{CenumAxLastTr}{0}
% 
% \newcommand{\resetCenumAx}[2][0]{%
% \setcounter{CenumAx#2}{1}
% \addtocounter{CenumAx#2}{#1}
% \setcounter{CenumAxLast#2}{0}
% \addtocounter{CenumAxLast#2}{#1}
% }
% 
% 
% 
% %length \itemWidth for storing the length of an item in an enumAx-List
% \newlength{\itemWidth}
% \setlength{\itemWidth}{\linewidth}
% 
% %command for storing the type of formula within one enumAx-block
% % (A is just a default)
% % careful: used for counter construction, cf. CenumAx?
% \newcommand{\enumAxType}{A}
% \newcommand{\backupCenumAx}[1][A]{\setcounter{CenumAxLast#1}{\value{CenumAx#1}}}
% 
% % list environment for annotated formulas
% %
% % usage examples: 1. some basic symbols
% %                 2. some axioms
% %                    (contains labels, and itemTP -> note explicit mathmode)
% %                 
% %\begin{enumAx}[B]
% %   \itemT{x+y}{abstract mereological sum}
% %   \itemT{x \Lintsect y}{abstract intersection}
% %   \itemT{c(x)}{abstract complement}
% %   \itemT{x-y}{abstract relative complement}
% %\end{enumAx}
% %
% %\begin{enumAx}
% %   \itemT{\label{ax:MmrAppliedToDabsPP} (\MEntCM(x) \Land \MEntCM(y) \Limp
% %          \Gpp(x, y)) \Liff (\MEntCM(x) \Land \MEntCM(y) \Limp \Gp (x, y)) \Land
% %          (\MEntCM(x) \Land \MEntCM(y) \Limp x \neq y)}{}
% %   \itemTP{\label{ax:MmrAppliedToDoverlap} $(\MEntCM(x) \Land \MEntCM(y) \Limp
% %          \Go(x, y)) \Liff$\\
% %                  \hspace*{\fill}$\exists z ( \MEntCM(z) \Land (\MEntCM(z) \Land
% %             \MEntCM(x) \Limp \Gp(z, x)) \Land (\MEntCM(z) \Land \MEntCM(y) \Limp
% %             \Gp(z, y))\:)$}{}
% %\end{enumAx}
% %
% %
% %\newcommand{\axSecConjecture}{\textit{Conjecture}:\hspace*{1em}}
% %\newcommand{\axSecProblem}{\textit{Problem}:\hspace*{1em}}
% %\newcommand{\axSecRemark}{\textit{Remark}:\hspace*{1em}}
% 
% % initialize optional commands
% \newenvironment{enumAx}[1][A]{}{}
% \newcommand{\itemT}{}
% \newcommand{\itemTP}{}
% \newcommand{\itemTA}{}
% %\newcommand{\GSymbolFont}{}
% 
% % AXIOMS & DEFINITIONS ends %%%%%%%%%%%%%%%%%%%%%
% %%%%%%%%%%%%%%%%%%%%%%%%%%%%%%%%%%%%%%%%%%%%%%%%%
% 
% 
% % Declaration of options
% 
% %%%%%%%%%%%%%%%%%%%%%%%%%%%%%%%%%%%%%%%%%%%%%%%%%%%%%%%%
% % OPTION manAxiomStyle STARTS
% %%%%%%%%%%%%%%%%%%%%%%%%%%%%%%%%%%%%%%%%%%%%%%%%%%%%%%%%%%
% \DeclareOption{manAxiomStyle}{%
% \renewenvironment{enumAx}[1][A]%
%    {%\renewcommand{\baselinestretch}{1.5}\selectfont
%     \renewcommand{\enumAxType}{#1}
%     \begin{list}{#1\arabic{CenumAx#1}.}%
%       {\usecounter{CenumAx#1}\addtocounter{CenumAx#1}{\value{CenumAxLast#1}}%
%        \setlength{\topsep}{0.5ex}
%       %\setlength{\labelsep}{1ex}%\setlength{\leftmargin}{1em}\setlength{\listparindent}{0mm}%
%        \setlength{\itemsep}{0.20\baselineskip}%  n06.12.2013: intermediate use of 0.50\baselineskip for double line spacing
%        \setlength{\itemWidth}{\linewidth-\leftmargin-\rightmargin}%
%       }%
%     \small
%    }%
%    {\backupCenumAx[\enumAxType]%
%     \end{list}%
%     \setlength{\itemWidth}{\linewidth}%
%     %\renewcommand{\baselinestretch}{2}\selectfont
%    }
% 
% % %itemT executes \item, then puts up tabular structure: arg2 first col, arg3 sec. col
% % % arg1 optional as standard for item
% % \renewcommand{\itemT}[3][]{\ifthenelse{\equal{#1}{}}{\item}{\item[#1]}%
% %    \begin{tabular*}{.98\itemWidth}{@{}l@{\hspace{1em}\extracolsep\fill}r@{}}%
% %       $#2$ & \ifthenelse{\equal{#3}{}}{}{(#3)}
% %    \end{tabular*}
% % }
% % revised itemT (without tables)
% \renewcommand{\itemT}[3][]{%
%    \ifthenelse{\equal{#1}{}}{\item}{\item[#1]}%
%    \parbox[t]{.994\itemWidth}{$#2$\hspace{1em}\hspace{\fill}\ifthenelse{\equal{#3}{}}{}{(#3)}}
% }
% 
% %itemTP executes \item, then puts up parbox: arg2 first, followed by arg3 in parentheses
% \renewcommand{\itemTP}[3][]{%
%    \ifthenelse{\equal{#1}{}}{\item}{\item[#1]}%
%    \parbox[t]{.994\itemWidth}{#2\hspace*{1em}\hspace{\fill}\ifthenelse{\equal{#3}{}}{}{(#3)}}
% }
% %itemTA executes \itemT as above and adds a new command for a copy
% % args: 1.) type letter, 2.) internal label, 3.) formula, 4.) explanation
% % example: \itemTA{B}{Babspartof}{\Gp(x, y)}{abstract part-of}
% \renewcommand{\itemTA}[4]{\itemT{\label{#2} #3}{#4}%
%     \expandafter\gdef\csname itemCopy#2\endcsname{\itemT[#1\ref{#2}.]{#3}{#4}}%
% }
% %
% %
% } %OPTION manAxiomStyle ENDS HERE
% %%%%%%%%%%%%%%%%%%%%%%%%%%%%%%%%%%%%%%%%%%%%%%%%%%%
% %%%%%%%%%%%%%%%%%%%%%%%%%%%%%%%%%%%%%%%%%%%%%%%%%%% 
% 
% %%%%%%%%%%%%%%%%%%%%%%%%%%%%%%%%%%%%%%%%%%%%%%%%%%%%%%%%
% % OPTION simpleAxiomStyle STARTS
% %%%%%%%%%%%%%%%%%%%%%%%%%%%%%%%%%%%%%%%%%%%%%%%%%%%%%%%%%%
% \DeclareOption{simpleAxiomStyle}{%
% \renewenvironment{enumAx}[1][A]%
%    {\begin{enumerate}%
%     \setlength{\itemWidth}{\linewidth}%
%    }%
%    {\end{enumerate}%
%    }
% 
% %itemT executes \item with arg2 in mathmode
% \renewcommand{\itemT}[3][]{\item $#2$}
% 
% %itemTP executes \item, then puts up parbox with arg2
% \renewcommand{\itemTP}[3][]{\item 
%    \parbox[t]{\itemWidth}{#2}
% }
% %itemTA NOT AVAILABLE IN THIS STYLE!
% % executes \itemT as above and adds a new command for a copy
% % args: 1.) type letter, 2.) internal label, 3.) formula, 4.) explanation
% % example: \itemTA{B}{Babspartof}{\Gp(x, y)}{abstract part-of}
% %\newcommand{\itemTA}[4]{\itemT{\label{#2} #3}{#4}%
% %    \expandafter\gdef\csname itemCopy#2\endcsname{\itemT[#1\ref{#2}.]{#3}{#4}}%
% %}
% %
% %
% } %OPTION simpleAxiomStyle ENDS HERE
% 



\newcommand{\overviewnumber}{3}
\newcommand{\overviewname}{\strukt}

\begin{document}


\paragraph{Raumregionen}\ \\
    Eine Teilmenge $A \subseteq \R^3$ heißt \textbf{Raumregion}, wenn
    \begin{enumerate}
        \item $\varnothing \neq A \in \OC$ und
        \item $A$ beschränkt%  
                \footnote{Eine Teilmenge $A \subseteq \R^n$ ist beschränkt falls $\exists\: M \in \R : A \subseteq B_M(0)$.}
                ist
    \end{enumerate}
    Die Menge der Raumregionen bezeichnen wir mit $\univ^3$


\paragraph{Repräsentanten}
%
        \begin{enumerate}
            \item $(A,B)$ ist ein \textbf{Flächenrepräsentant}, wenn 
                \begin{enumerate}
                    \item $A$ eine Raumregion und
                    \item $\varnothing \neq B \in \OC_{\rand_A}$ ist.
                \end{enumerate}
                Die \textbf{Menge der Flächenrepräsentanten} bezeichnen wir mit $\rep^2$
            \item $(A,B,C)$ ist ein \textbf{Linienrepräsentant}, wenn 
                \begin{enumerate}
                    \item $(A,B)$ ein Flächenrepräsentant und
                    \item $\varnothing \neq C \in \OC_{\delta B}$ ist.
                \end{enumerate}
                Die \textbf{Menge der Linienrepräsentanten} bezeichnen wir mit $\rep^1$
            \item $(A,B,C,D)$ ist ein \textbf{Punktrepräsentant}, wenn 
                \begin{enumerate}
                    \item $(A,B,C)$ ein Linienrepräsentant und
                    \item $D \subseteq \delta^2 C$ ist.
                \end{enumerate}
                Die \textbf{Menge der Punktrepräsentanten} bezeichnen wir mit $\rep^0$
        \end{enumerate}

        
\paragraph{Objektäquivalenz}
%
    \begin{enumerate}
        \item Zwei \textbf{Flächenrepräsentanten} $(A_1,B_1)$ und $(A_2,B_2)$ sind \textbf{objektäquivalent}, wenn 
            \begin{enumerate}
                \item $B_1 = B_2$ und
                \item $A_1 =_{B_1} A_2$
            \end{enumerate}
            gelten
        \item Zwei \textbf{Linienrepräsentanten} $(A_1,B_1,C_1)$ und $(A_2,B_2,C_2)$ sind \textbf{objektäquivalent}, wenn 
            \begin{enumerate}
                \item $C_1 = C_2$,
                \item $A_1 =_{C_1} A_2$ und $B_1 =_{C_1} B_2$
            \end{enumerate}
            gelten.
        \item Zwei \textbf{Punktrepräsentanten} $(A_1,B_1,C_1,D_1)$ und $(A_2,B_2,C_2,D_2)$ sind \textbf{objektäquivalent}, wenn 
            \begin{enumerate}
                \item $D_1 = D_2$,
                \item $A_1 =_{D_1} A_2$, $B_1 =_{D_1} B_2$ und $C_1 =_{D_1} C_2$
            \end{enumerate}	
            gelten.		
    \end{enumerate}
    
    Wenn zwei Repräsentanten $x$ und $y$ objektäquivalent sind, so schreiben wir $x \sim y$.
    
    
\paragraph{Niederdimensionale Raumentitäten}
%
    \begin{enumerate}
        \item Für $(A,B) \in \rep^2$ bezeichnet 
            \begin{align*}
                [A,B] := \{(A',B') \in \rep^2 \mid (A',B') \sim (A,B)\}
            \end{align*}
            die Äquivalenzklasse von $(A,B)$ bezüglich der Objektäquivalenz.\\
            Diese Äquivalenzklassen heißen \textbf{Flächenregionen}.
        \item Für $(A,B,C) \in \rep^1$ bezeichnet
            \begin{align*}
                [A,B,C] := \{(A',B',C') \in \rep^1 \mid (A',B',C') \sim (A,B,C)\}
            \end{align*}			 
            die Äquivalenzklasse von $(A,B,C)$ bezüglich der Objektäquivalenz.\\
            Diese Äquivalenzklassen heißen \textbf{Linieregionen}.
        \item Für $(A,B,C,D) \in \rep^0$ bezeichnet
            \begin{align*}
                [A,B,C,D] :=
                \{(A',B',C',D') \in \rep^0 \mid (A',B',C',D') \sim (A,B,C,D)\}
            \end{align*}			 
            die Äquivalenzklasse von $(A,B,C,D)$ bezüglich der Objektäquivalenz.\\
            Diese Äquivalenzklassen heißen \textbf{Punktregionen}.
    \end{enumerate}
    %
    \noindent	
        Für $i \in \{0,1,2\}$ ist $\univ^i := \rep^i /_\sim$ die \textbf{Menge der Äquivalenzklassen} von $\rep^i$.\\

        
\paragraph{Universum}\ \\
    Das Universum der \strukt ist definiert als
    \begin{align*}
        \univ = \univ^3 \cup \univ^2 \cup \univ^1 \cup \univ^0
    \end{align*}
    
    
\paragraph{Dimensionsfunktion}\ \\
$\Gdim : \univ \to \N$ ist definiert durch
            $ \Gdim(x) = i \quad \Leftrightarrow \quad x \in \univ^i $
    
    
\paragraph{$\boldsymbol{\GSReg}$:}
Für $x \in \univ$ gilt $\GSReg(x)$ gdw.\ $x\in \univ^3$.


\paragraph{$\boldsymbol{\Gsb}$}
%
    \begin{enumerate}
        \item Für $(A_1,B_1) \in \rep^2$, $A_2 \in \univ^3$ gilt 
            \begin{align*}
                \Gsb([A_1,B_1],A_2) \quad \Leftrightarrow \quad (A_1,B_1) \sim (A_2,B_1)
            \end{align*}
        \item Für $(A_1,B_1,C_1) \in \rep^1$, $(A_2,B_2) \in \rep^2$ gilt 
            \begin{align*}
                \Gsb([A_1,B_1,C_1],[A_2,B_2]) \quad 
                \Leftrightarrow  \quad (A_1,B_1,C_1) \sim (A_2,B_2,C_1)
            \end{align*}
        \item Für $(A_1,B_1,C_1,D_1) \in \rep^0$, $(A_2,B_2,C_2) \in \rep^1$ gilt 
            \begin{align*}
                \Gsb([A_1,B_1,C_1,D_1],[A_2,B_2,C_2]) \quad 
                \Leftrightarrow  \quad (A_1,B_1,C_1,D_1) \sim (A_2,B_2,C_2,D_1)
            \end{align*}
        \item Für $\Gdim(x) + 1 \neq \Gdim(y)$ gilt $\neg \Gsb(x,y)$
    \end{enumerate}
    
    
\paragraph{$\boldsymbol{\Gscoinc}$}

    \begin{enumerate}
        \item Für $A_1, A_2 \in \univ^3$ gilt: $\neg \Gscoinc(A_1,A_2)$
        \item Für $(A_1,B_1), (A_2,B_2) \in \rep^2$ gilt 
            \begin{align*}
                \Gscoinc([A_1,B_1],[A_2,B_2]) \quad \Leftrightarrow  \quad B_1 = B_2
            \end{align*}
        \item Für $(A_1,B_1,C_1), (A_2,B_2,C_2) \in \rep^1$ gilt 
            \begin{align*}
                \Gscoinc([A_1,B_1,C_1],[A_2,B_2,C_2]) \quad \Leftrightarrow  \quad C_1 = C_2
            \end{align*}
        \item Für $(A_1,B_1,C_1,D_1), (A_2,B_2,C_2,D_2) \in \rep^0$ gilt 
            \begin{align*}
                &\Gscoinc([A_1,B_1,C_1,D_1],[A_2,B_2,C_2,D_2]) \\
                &\Leftrightarrow  \quad D_1 = D_2
            \end{align*}
        \item Für $\Gdim(x) \neq \Gdim(y)$ gilt $\neg \Gscoinc(x,y)$.
    \end{enumerate}
    
    
\paragraph{$\boldsymbol{\Gspart}$}

    \begin{enumerate}
        \item Für $A_1, A_2 \in \univ^3$ gilt $\Gspart(A_1,A_2)$ gdw.\ $A_1 \subseteq A_2$		
        \item Für $(A_1,B_1), (A_2,B_2) \in \rep^2$ gilt
        $\Gspart([A_1,B_1],[A_2,B_2])$ gdw.\ 
            \begin{enumerate}
                \item $B_1 \subseteq B_2$
                \item $(A_1, B_1) \sim (A_2, B_1)$
            \end{enumerate}
        \item Für $(A_1,B_1,C_1), (A_2,B_2,C_2) \in \rep^1$ gilt
        $\Gspart([A_1,B_1,C_1],[A_2,B_2,C_2])$ gdw.\ 
            \begin{enumerate}
                \item $C_1 \subseteq C_2$
                \item $(A_1, B_1, C_1) \sim (A_2, B_2, C_1)$
            \end{enumerate}
        \item Für $(A_1,B_1,C_1,D_1), (A_2,B_2,C_2,D_2) \in \rep^0$ gilt
        $\Gspart([A_1,B_1,C_1,D_1],[A_2,B_2,C_2,D_2])$ gdw.\
            \begin{enumerate}
                \item $D_1 = D_2$
                \item $(A_1, B_1, C_1, D_1) \sim (A_2, B_2, C_2, D_1)$
            \end{enumerate}
        \item Für $\Gdim(x) \neq \Gdim(y)$ gilt $\neg \Gspart(x,y)$.
    \end{enumerate}
 
\end{document}

