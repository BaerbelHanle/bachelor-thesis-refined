%*******************************************************
% Abstract
%*******************************************************
\pdfbookmark[1]{Abstrakt}{Abstrakt}
\chapter*{Abstrakt}
\addcontentsline{toc}{chapter}{Abstrakt}
%Der Abstrakt soll kurz gehalten werden und dazu dienen, den Leser zum Lesen der kompletten Arbeit zu motivieren. Er ist optional.

Die General Formal Ontology (GFO) ist eine Top-Level-Ontologie, die grundlegende Begriffe bereitstellen soll, die für ein breites Anwendungsfeld von Bedeutung sind.
Die Theorie des Raumes in GFO ist von den Arbeiten des Philosophen Franz Brentano zu Raum, Zeit und Kontinuum inspiriert und heißt deshalb Brentanoraum.
Die vorliegende Arbeit will einen Beitrag zu einem konstruktiven Konsistenzbeweis für deren Axiomatisierung $\theoryBS$ leisten.

Dazu wird die Teiltheorie $\theoryBSO$ der ordinären Entitäten des Brentanoraumes eingeführt, die sich von $\theoryBS$ unter anderem dadurch unterscheidet, dass die Existenz der mereologischen Summe eingeschränkt ist.
Möglicherweise lässt sich also jedes Modell für $\theoryBSO$ durch Abschluss unter Summenbildung zu einem Modell für $\theoryBS$ erweitern.

Für die Theorie $\theoryBSO$ wird eine Interpretation -- die \strukt\ -- als Kandidat für ein Modell vorgeschlagen.
Die wesentliche Idee dabei ist, Raumentitäten als Äquivalenzklassen sogenannter Repräsentanten aufzufassen.
Um diese Idee mathematisch zu formulieren, werden auch Werkzeuge aus der Topologie benötigt, deren Einführung ein weiterer wichtiger Teil dieser Arbeit ist.

% Die vorliegende Arbeit besteht im Wesentlichen aus drei Teilen:
%
% \begin{enumerate}
%     \item In den ersten Kapiteln wird der Brentanoraum und seine Axiomatiserung $\theoryBS$ eingeführt, die Einschränkung $\theoryBSO$ vorgestellt und die grundlegenden Ideen der \strukt erläutert.
%     
%     \item Darauf folgt ein mathematischer Teil, in dem die grundlegenden Begriffe der Topologie eingeführt werden, die für diese Arbeit relevant sind und darauf aufbauen neue topologische Begriffe definiert werden mit deren Hilfe sich die im ersten Teil formulierten Ideen auf mathematisch formale Weise umsetzen lassen.
%     
%     \item Der letzte Teil führt diese beiden Stränge zusammen und liefert eine lückenlose Definition der \strukt.
% \end{enumerate}








%Ein Konsistenzbeweis dieser Theorie steht noch aus. Die vorliegende Arbeit will hierzu einen Beitrag leisten.

%Sie ist von des Arbeiten des Philosophen Franz Brentano zu Raum, Zeit und Kontinuum inspiriert.

%breites Anwendungsfeld bereitstellen soll.
%Dazu gehören räumliche Begriffe

%die räumliche Beziehungen zwischen Objekten und ihren Grenzen beschreiben

%Diese Begriffe werden innerhalb von GFO durch $\theoryBS$ -- die Axiomatisierung als Prädikatenlogik erster Stufe des sogenannten Brentanoraumes bereitgestellt.
%er stützt sich auf die Arbeiten des Philosophen Franz von Brentano zu Raum Zeit und Kontinuum

%Sie wurde ab ... maßgeblich von Herre, Baumann und Loebe  entwickelt
%Bislang sthet ein Konsistenzbeweis noch aus.

% Diese Arbeit will einen Beitrag zu einem konstruktiven Kosistenzbeweis liefern indem sie als eine Interpretation für eine teiltheorie von $\theoryBS$ eine Struktur einführt
% 
% Diese teiltheorie geht -- im gegensatz zu $\theoryBS$ nicht von der uneingeschränkten Existenz der mereologischen Summe aus.
% 
% Möglicherweise lässt sie sich durch Abschluss unter Summenbildung zue einm Modell für $\theoryBSO$ erweitern.
% 
% Die wesentliche Idee dabei ist, raumentitäten ais Äquivalenzklassen der sogennanten Repräsentanten aufzufassen.
% Für die Definition der hierfür benötigenten Konzepte werden Werkzeuge aus der Topologie verwendet
% Deshalb besteht die Arbeit als zweiten ausführlich Teil aus der bereitstellung der topologischen grundbegriffe und der weiterführenden definition weiterer bergriffe, die auf diese aufbauen
% 
% ..................................





\vfill
