\section{Modifizierte BS-Theorie}
Unterschied: scoinc wird als Äquivalenzrelation auf den Raumentitäten betrachtet.
Grund: Modellbau einfacher
alter Koinzidenzbegriff lässt sich rekonstruieren durch
\begin{align*}
  \Gscoinc^{alt}(x,y) := \Gscoinc(x,y) \land \GSB(x) \land \GSB(y)
\end{align*}


\begin{erin}[A1. Existenz einer Raumregion]
    \begin{align*}
        \exists x\ \GSReg(x)
    \end{align*}
\end{erin}

\begin{erin}[A2. Raumetitäten sind entweder Raumregionen oder niederdimensional]
    \begin{align*}
        \GLDE(x) \leftrightarrow \neg\GSReg(x)
    \end{align*}
\end{erin}
                
\begin{erin}[A3. Niederdimensionale Raumentitäten zerfallen in 3 disjunkte Klassen]
    \begin{align*}
        \neg\exists x\ (\ (\GtwoDE(x) \wedge \GoneDE(x)) \vee (\GtwoDE(x) \wedge \GzeroDE(x)) \vee (\GoneDE(x) \wedge \GzeroDE(x))\ )
    \end{align*}
\end{erin}

\begin{erin}[A4. Reflexivität von $\Gspart$]
    \begin{align*}
        \Gspart(x,x)
    \end{align*}
\end{erin}

\begin{erin}[A5. Antisymmetrie von $\Gspart$]
    \begin{align*}
        \Gspart(x,y) \wedge \Gspart(y,x) \to x=y
    \end{align*}
\end{erin}

\begin{erin}[A6. Transitivität von $\Gspart$]
    \begin{align*}
        \Gspart(x,y) \wedge \Gspart(y,z) \to \Gspart(x,z)
    \end{align*}
\end{erin}
             
\begin{erin}[A7. Definitionsbereich von $\Gspart$]
    \begin{align*}
        \Gspart(x,y) \to \Geqdim(x,y)
    \end{align*}
\end{erin}

\begin{erin}[A8. Starkes Ergänzungsprinzip]
    \begin{align*}
        \neg\Gspart(x,y) \to \exists z\ (\Gspart(z,x) \wedge \neg\Gsov(z,y))
    \end{align*}
\end{erin}

\begin{erin}[A9. Nur Punkte sind atomar]
    \begin{align*}
        \neg\GzeroD(x) \to \exists y\ (\Gsppart(y,x) \wedge \Ginpart(y,x))
    \end{align*}
\end{erin}
        
\begin{erin}[A10. Erweiterbarkeit von Raumentitäten]
    \begin{align*}
        \exists y\ \Gsppart(x,y)
    \end{align*}
\end{erin}

\begin{erin}[A11. Einbettendes Topoid für Raumregionen]
    \begin{align*}
        \GSReg(x) \to \exists y\ (\GTop(y) \wedge \Gspart(x,y))
    \end{align*}
\end{erin}

\begin{erin}[A12. Existenz von Grenzen für Raumregionen]
    \begin{align*}
        \GSReg(x) \to \exists y\ \Gsb(y,x)
    \end{align*}
\end{erin}

\begin{erin}[A13. Existenz von Grenzen für Teile von Flächenregionen]
    \begin{align*}
        \GtwoDE(x) \to \exists y z\ (\Gsppart(y,x) \wedge \Gsb(z,y))
    \end{align*}
\end{erin}

\begin{erin}[A14. Existenz von Grenzen für Teile von Linienregionen]
    \begin{align*}
        \GoneDE(x) \to \exists y z\ ( \Gsppart(y,x) \wedge \Gsb(z,y))
    \end{align*}
\end{erin}

\begin{erin}[A15. Existenz von größten Grenzen]
    \begin{align*}
        \GOrd(x) \wedge \exists y\ \Gsb(y,x) \to \exists z\ \Ggrsb(z,x)
    \end{align*}
\end{erin}

\begin{erin}[A16. Existenz der mereologischen Summe]
    \begin{align*}
        \Geqdim(x,y) \to \exists z\ \Gsum(x,y,z)
    \end{align*}
\end{erin}

\begin{erin}[A17. Existenz des mereologischen Schnitts]
    \begin{align*}
        \Gsov(x,y) \to \exists z\ \Gintersect(x,y,z)
    \end{align*}
\end{erin}

\begin{erin}[A18. Existenz des relativen Komplements]
    \begin{align*}
        \Geqdim(x,y) \wedge \neg\Gspart(y,x) \to \exists z\ \Grelcompl(x,y,z)
    \end{align*}
\end{erin}

\begin{dfn}[A19'. Reflexivität von $\Gscoinc$]
    \begin{align*}
        \Gscoinc(x,x)
    \end{align*}
\end{dfn}

\begin{erin}[A20. Symmetrie von $\Gscoinc$]
    \begin{align*}
        \Gscoinc(x,y) \to \Gscoinc(y,x)
    \end{align*}
\end{erin}

\begin{erin}[A21. Transitivität von $\Gscoinc$]
    \begin{align*}
        \Gscoinc(x,y) \wedge \Gscoinc(y,z) \to \Gscoinc(x,z)
    \end{align*}
\end{erin}

\begin{dfn}[A22'. Definitionsbereich von $\Gscoinc$]
    \begin{align*}
        \Gscoinc(x,y) \to \Geqdim(x,y) \wedge (\GSReg(x) \to x=y)
    \end{align*}
\end{dfn}

\begin{erin}[A23. Ordinarität und Grenzen]
    \begin{align*}
        \Gsb(x,y) \wedge \GOrd(y) \to \GOrd(x)
    \end{align*}
\end{erin}

\begin{erin}[A24. Teile von Grenzen sind Grenzen]
    \begin{align*}
        \Gsb(y,z) \wedge \Gspart(x,y) \to \Gsb(x,z)
    \end{align*}
\end{erin}

\begin{erin}[A25. Es gibt keine neuen Grenzen]
    \begin{align*}
        \Gtangpart(x,y) \wedge \Gsb(x',x) \wedge \Gsb(y',y) \wedge \Gscoinc(x',y') \to \Gsb(x',y)
    \end{align*}
\end{erin}
                
\begin{erin}[A26. Definitionsbereich von $\Gsb$]
    \begin{align*}
        \Gsb(x,y) \to (\GtwoDB(x) \wedge \GSReg(y)) \vee (\GoneDB(x) \wedge \GtwoDE(y)) \vee (\GzeroDB(x) \wedge \GoneDE(y))
    \end{align*}
\end{erin}

\begin{erin}[A27. Existenz koinzidierender Teile]
    \begin{align*}
        \Gscoinc(x,y) \wedge \Gspart(x',x) \to \exists y' \ (\Gspart(y',y) \wedge \Gscoinc(x',y'))
    \end{align*}
\end{erin}

\begin{erin}[A28. Existenz koinzidierender Hyperteile]
    \begin{align*}
        \Gscoinc(x,y) \wedge \Ghypp(x',x) \to \exists y' \ (\Ghypp(y',y) \wedge \Gscoinc(x',y'))
    \end{align*}
\end{erin}

\begin{erin}[A29. Existenz koinzidierender Grenzen]
    \begin{align*}
        \Gscoinc(x,y) \wedge \Gsb(x',x) \to \exists y' \ (\Gsb(y',y) \wedge \Gscoinc(x',y'))
    \end{align*}
\end{erin}

\begin{erin}[A30.]
    \begin{align*}
        \Geqdim(x,y) \wedge \neg\Gsov(x,y) \wedge \Ghypp(x',x) \wedge \Ghypp(y',y) \to x' \neq y'
    \end{align*}
\end{erin}
                
\begin{erin}[A31. Entitäten mit einer gemeinsamen Grenze besitzen einen gemeinsamen Teil an dieser Grenze]
    \begin{align*}
        \Gsb(x,y) \wedge \Gsb(x,z) \leftrightarrow \exists u (\Gsb(x,u) \land \Gsppart(u,y) \land \Gsppart(u,z))
    \end{align*}
\end{erin}

\begin{erin}[A32'.]
    \begin{align*}
        \Gsb(x,y) \land \Gsb(u,v) \land \neg \Gsov(x,u) \leftrightarrow \exists y'v'\:(\: \Gspart(y',y) \land \Gsb(x,y') \land \Gspart(v',v) \land \Gsb(u,v') \land \neg \Gsov(y',v'))
    \end{align*}
\end{erin}

\begin{erin}[A33. Ordinäre niederdimensionale Raumentitäten sind Grenzen]
    \begin{align*}
        \GLDE(x) \land \GOrd(x) \leftrightarrow \GSB(x)
    \end{align*}
\end{erin}





