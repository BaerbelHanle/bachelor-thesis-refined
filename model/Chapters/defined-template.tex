\section{Definierte Relationen}

\begin{erin}[D1. $x$ ist eine 2-dim.\ Grenze von $y$]
    \begin{align*}
        \Gtwodb(x,y) := \GSReg(y) \wedge \Gsb(x,y)
    \end{align*}
\end{erin}

\begin{erin}[D2. $x$ ist eine 1-dim.\ Grenze von $y$]
    \begin{align*}
        \Gonedb(x,y) := \GtwoDB(y) \wedge \Gsb(x,y)
    \end{align*}
\end{erin}

\begin{erin}[D3. $x$ ist eine 0-dim.\ Grenze von $y$]
    \begin{align*}
        \Gzerodb(x,y) := \GoneDB(y) \wedge \Gsb(x,y)
    \end{align*}
\end{erin}

\begin{erin}[D4. $x$ ist eine räumliche Grenze]
    \begin{align*}
        \GSB(x) := \exists y\ \Gsb(x,y)
    \end{align*}
\end{erin}

\begin{erin}[D5. $x$ ist eine $\Gd$-dim.\ Grenze]
    \begin{align*}
        \GdDB(x) := \exists y\ \Gddb(x, y)
    \end{align*}
\end{erin}

\begin{erin}[D6. $x$ ist die größte Grenze von $y$]
    \begin{align*}
        \Ggrsb(x,y) := \Gsb(x,y) \wedge \forall z\ (\Gsb(z,y) \to \Gspart(z,x))
    \end{align*}
\end{erin}
 
\begin{erin}[D7. $x$ ist eine niederdimensionale Raumentität]
    \begin{align*}
        \GLDE(x) := \exists y\ (\GSB(y) \wedge \Gspart(y, x))
    \end{align*}
\end{erin}

\begin{erin}[D8. $x$ ist ein Flächen-/Linien-/Punktregion]
    \begin{align*}
        \GdDE(x) := \exists y\ (\GdDB(y) \wedge \Gspart(y, x))
    \end{align*}
\end{erin}

\begin{erin}[D9. $x$ und $y$ haben die selbe Dimension]
    \begin{align*}
        \Geqdim(x,y) := (&\GSReg(x) \wedge \GSReg(y)) \vee \GtwoDE(x) \wedge \GtwoDE(y))\\ \vee (&\GoneDE(x) \wedge \GoneDE(y)) \vee (\GzeroDE(x) \wedge \GzeroDE(y))
    \end{align*}
\end{erin}

\begin{erin}[D10. $x$ ist ein echter Teil von $y$]
    \begin{align*}
        \Gsppart(x, y) := \Gspart(x, y) \land  x \neq y
    \end{align*}
\end{erin}

\begin{erin}[D11. $x$ und $y$ überlappen]
    \begin{align*}
        \Gsov(x, y) := \exists z\  (\Gspart(z, x) \wedge \Gspart(z, y))
    \end{align*}
\end{erin}

\begin{erin}[D12. $x$ ist die mereologische Summe von $x_{1}$,\ldots,$x_{n}$]
    \begin{align*}
        \Gsumn(x_{1},\ldots,x_{n},x) := \forall y\ (\Gsov(y,x) \leftrightarrow \bigvee^{n}_{i=1} \Gsov(y,x_{i}))
    \end{align*}
\end{erin}

\begin{erin}[D13. $x$ ist der mereologische Schnitt von $x_{1}$,\ldots,$x_{n}$]
    \begin{align*}
        \Gintersectn(x_{1},\ldots,x_{n},x)  := \forall y\ (\Gspart(y,x) \leftrightarrow \bigwedge^{n}_{i=1} \Gspart(y,x_{i}))
    \end{align*}
\end{erin}

\begin{erin}[D14. $x$ ist das rel. Kompl. von $x_{n}$ und $x_{1}$,\ldots,$x_{n-1}$]
    \begin{align*}
        \Grelcompln(x_{1},\ldots,x_{n},x) := \hspace*{1em}\bigwedge_{1\leq i < j \leq n} \Geqdim(x_i, x) \wedge\\
        \forall y\ (\Gspart(y,x)
        \leftrightarrow \bigwedge^{n-1}_{i=1} \neg\Gsov(y,x_{i}) \wedge \Gspart(y, x_{n}))
    \end{align*}
\end{erin}

\begin{erin}[D15. $x_{1}$,\ldots,$x_{n}$ zerlegen $x$]
    \begin{align*}
        \Gpartitionn(x_{1},\ldots,x_{n},x) := &\Gsumn(x_{1},\ldots,x_{n},x) \wedge\\ &\bigwedge_{1 \leq i < j \leq n} \neg\Gsov(x_i,x_j)
    \end{align*}
\end{erin}

\begin{erin}[D16. $x$ ist ein 2-dim.\ Hyperteil von $y$]
    \begin{align*}
        \Gtwodhypp(x,y) := \exists z\ (\Gspart(z,y) \wedge \Gtwodb(x,z))
    \end{align*}
\end{erin}

\begin{erin}[D17. $x$ ist ein 1-dim.\ Hyperteil von $y$]
    \begin{align*}
        \Gonedhypp(x,y) := \exists z\ ((&\Gspart(z,y) \vee \Gtwodhypp(z,y))\\
        \wedge  &\Gonedb(x,z))
    \end{align*}
\end{erin}

\begin{erin}[D18. $x$ ist ein 0-dim.\ Hyperteil von $y$]
    \begin{align*}
        \Gzerodhypp(x,y) := \exists z\ ((&\Gspart(z,y) \vee \Gonedhypp(z,y))\\ 
        \wedge  &\Gzerodb(x,z))
    \end{align*}
\end{erin}
               
\begin{erin}[D19. $x$ ist ein Hyperteil von $y$]
    \begin{align*}
        \Ghypp(x,y) := \Gtwodhypp(x,y) \vee \Gonedhypp(x,y) \vee \Gzerodhypp(x,y)
    \end{align*}
\end{erin}
  
\begin{erin}[D20. $x$ ist ein tangentialer Teil von $y$]
    \begin{align*}
        \Gtangpart(x,y) := (\Gspart(x,y) \vee \Ghypp(x,y))\ \ \wedge\\
        \exists x'zz'\ ((\Gspart(x',x) \vee \Ghypp(x',x)) \wedge 
        \Gsb(z,y) \\
        \wedge (\Gspart(z',z) \vee \Ghypp(z',z)) \wedge \Gscoinc(x',z'))
    \end{align*}
\end{erin}

\begin{erin}[D21. $x$ ist ein innerer Teil von $y$]
    \begin{align*}
        \Ginpart(x, y) := (\Gspart(x,y) \vee \Ghypp(x,y)) \land \neg \Gtangpart(x, y)
    \end{align*}
\end{erin}

\begin{erin}[D22. $x$ ist eine extraordinäre Raumentität]
    \begin{align*}
        \GExOrd(x) := \exists yz\ (&\Gspart(y,x) \wedge \Gspart(z,x) \\
        \wedge \neg&\Gsov(y,z) \wedge \Gscoinc(y,z))
    \end{align*}
\end{erin}

\begin{erin}[D23. $x$ ist eine ordinäre Raumentität]
    \begin{align*}
        \GOrd(x) := \neg\GExOrd(x)
    \end{align*}
\end{erin}

\begin{erin}[D24. $x$ ist $\Gd$-dim.\ zusammenhängend]
    \begin{align*}
        \GdDC(x) := \exists uv\ (\Gpartition(u,v,x)) \wedge \forall yz\ (\Gpartition(y,z,x)\\ 
        \to 
        \exists y'z'(\Gddhypp(y',y) \wedge \Gddhypp(z',z) \wedge \Gscoinc(y',z')))
    \end{align*}
\end{erin}

\begin{erin}[D25. $x$ ist zusammenhängend]
    \begin{align*}
        \GC(x) := \GtwoDC(x) \vee \GoneDC(x) \vee \GzeroDC(x)
    \end{align*}
\end{erin}

\begin{erin}[D26. $x$ und $y$ sind zusammenhängend]
    \begin{align*}
        \Gc(x,y) := \exists z\ (\Gsum(x,y,z) \wedge \GC(z))
    \end{align*}
\end{erin}

\begin{erin}[D27. $x$ und $y$ sind extern zusammenhängend]
    \begin{align*}
        \Gexc(x,y) := \Gc(x,y) \wedge \neg\Gsov(x,y)
    \end{align*}
\end{erin}

\begin{erin}[D28. $x$ hat eine $\Gd$-dim.\ zusammenhängende Komponente]
    \begin{align*}
        \GoneCCDd(x) := \GdDC(x)
    \end{align*}
\end{erin}

\begin{erin}[D29. $x$ hat $n$ $\Gd$-dim.\ zusammenhängende Komp.]
    \begin{align*}
        \GnCCDd(x) := &\bigwedge_{i=1}^{n-1} \neg\GiCCDd(x) \wedge\\
                    &\exists x_{1}...x_{n}(\Gpartitionn(x_{1},...,x_{n},x)
                    \wedge \bigwedge_{i=1}^{n} \GoneCCDd(x_{i}))
    \end{align*}
\end{erin}

\begin{erin}[D30. $x$ ist ein Topoid]
    \begin{align*}
        \GTop(x) := \GSReg(x) \wedge \GOrd(x) \wedge \GtwoDC(x)
    \end{align*}
\end{erin}

\begin{erin}[D31. $x$ ist eine Fläche]
    \begin{align*}
        \GtwoD(x) := \GtwoDE(x) \wedge \GOrd(x) \wedge \GoneDC(x)
    \end{align*}
\end{erin}

\begin{erin}[D32. $x$ ist eine Linie]
    \begin{align*}
        \GoneD(x) := \GoneDE(x) \wedge \GOrd(x) \wedge \GzeroDC(x)
    \end{align*}
\end{erin}

\begin{erin}[D33. $x$ ist ein Punkt]
    \begin{align*}
        \GzeroD(x) := \GzeroDE(x) \wedge \GOrd(x) \wedge \neg\exists y\ \Gsppart(y,x)
    \end{align*}
\end{erin}

\begin{erin}[D34. $x$ ist eine strikte räumliche Grenze]
    \begin{align*}
        \Gstrictsb(x,y) :=
        &\Gsb(x,y) \wedge \\
        &\forall x' (\Ghypp(x',y) \wedge \Gscoinc(x,x') \to x=x')
    \end{align*}
\end{erin}

\begin{erin}[D35. $x$ ist eine schwache räumliche Grenze]
    \begin{align*}
        \Gweaksb(x,y) := \Gsb(x,y) \wedge \neg\Gstrictsb(x,y)
    \end{align*}
\end{erin}
% 
% \begin{erin}
%     \begin{align*}
%         
%     \end{align*}
% \end{erin}
% 
%     &\Gonedircomp(x,y) := \GzeroD(x) \wedge \GzeroD(y) \wedge \Gscoinc(x,y) \wedge \mbox{}
%         \\ &\hspace{1em}
%         \exists x',y',x'',y''\ (\Gsb(x,x') \wedge \Gsb(x',x'') \wedge \Gsb(y,y') \wedge \Gsb(y',y'') \wedge
%         \\ &\hspace{2em}
%         \neg \Gsov(x',y') \wedge \neg \Gsov(x'', y'') \wedge \mbox{}
%         \\ &\hspace{2em}
%         \forall z\ (\Gscoinc(x,z) \wedge \Ghypp(z, \Ggrsb_f(\Gsum_f(x'',y''))) \to z=x \vee z=y))
%  
% \begin{erin}
%     \begin{align*}
%         
%     \end{align*}
% \end{erin}
% 
%     \begin{enumAx}[D]
% \itemTP{$\Gonedircomp(x,y) \Ldef \Gscoinc(x, y) \wedge \mbox{}$\\
% \hspace*{1em}
% 	$\exists x' y' z'\: 
% 	 (\:\GOrd(x') \wedge
% 	    \GOrd(y') \wedge
% 		  \Gsb(x, x') \wedge
% 			\Gsb(y, y') \wedge
%       \neg \Gsov(x', y') \wedge	\mbox{}$\\
%   \hspace*{2.5em}
%     $ ( \GSB(x') \wedge \GSB(y') \Limp \mbox{}$\\
% 	\hspace*{4em}
% 			 $\exists x'' y'' z'' g' \:
% 			  (\:\Gsb(x', x'') \wedge
% 					 \Gsb(y', y'') \wedge
%            \neg \Gsov(x'', y'') \wedge \mbox{}$\\
% 	\hspace*{6em}
% 	     $	 \Gsum(x'', y'', z'') \wedge
% 					 \Ggrsb(g', z'') \wedge
% 					 \Gzerodhypp(x, g') \wedge 
% 					 \Gzerodhypp(y, g') \wedge \mbox{}$\\
% 	\hspace*{6em}
% 				 $ \forall z (  \Gzerodhypp(z, g') \Limp \mbox{}$\\
% 	\hspace*{8em}		
% 					  $((\Gscoinc(x, z) \wedge \neg\Gsov(x,z) \Limp y=z ) \wedge \mbox{}$\\
% 	\hspace*{8em}\hphantom{$($}		
% 						$(\Gscoinc(y, z) \wedge \neg\Gsov(y,z) \Limp x=z ) 
% 					\ )))))$\\ \mbox{} \label{def:npodircomp}}
%   {$x$ and $y$ are 1-directionally compatible}
% 
% \begin{erin}
%     \begin{align*}
%         
%     \end{align*}
% \end{erin}
% 
% \itemTP{$\Gonecont(x) \Ldef \GoneDE(x) \wedge \mbox{}$\\
% \hspace*{1em}
%   $\forall y\:(\:(\Gzerodhypp(y,x) \wedge
% 	                \forall y' ( \Gspart(y',y) \to \neg\Gstrictsb(y',x))) \Limp \mbox{} $\\
% 	\hspace*{2.5em}	
% 	  $\exists z ( \Gzerodhypp(z,x) \wedge
% 		             \neg\Gsov(z,y) \wedge 
%                  \Gonedircomp(y,z)))$\\ \mbox{} \label{def:Gdcont}}
%              {$x$ is 1-continuous}	
% \end{enumAx}
