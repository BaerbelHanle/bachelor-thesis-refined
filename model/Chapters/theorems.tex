\section{Theorems of $\theoryBS$}

\begin{satz}[T23.]\label{satz:t23}
    $\forall z\ (\Gsov(z,x) \leftrightarrow \Gsov(z,y)) \to x = y$
\end{satz}

\begin{bew}
    Sei $x \neq y$.
    O.B.d.A. gelte $\neg \Gspart(x,y)$. 
    Dann gibt es nach dem starken Supplementationsprinzip (A8) ein $z$ mit $\Gspart(z,x)$ und $\neg \Gsov(z,y)$. 
    Wegen der Reflexivität der $\Gspart$-Relation (A4) folgt damit $\Gsov(z,x)$.
    Somit gilt $\neg \forall z\ (\Gsov(z,x) \leftrightarrow \Gsov(z,y))$.
\end{bew}



\subsection{Uniqueness theorems}
The following theorems allow the introduction of partial functions based on the uniqueness of certain relations in certain arguments.

\begin{erin}[Greatest spatial boundary]
    $\Ggrsb(x,y) := \Gsb(x,y) \wedge \forall z\ (\Gsb(z,y) \to \Gspart(z,x))$
\end{erin}

\begin{satz}
    The $\Ggrsb$-relation is unique in its first argument. I.e.
    \begin{align*}
        \Ggrsb(x_1,y) \land \Ggrsb(x_2,y) \to x_1 = x_2
    \end{align*}
\end{satz}

\begin{bew}\ \\
    \begin{longtable}{r c c l}
        & & 1. & $\Ggrsb(x_1,y)$\\
        & & 2. & $\Ggrsb(x_2,y)$ \\
        1 & $\deshalb$ & 3. & $\Gsb(x_1,y)$ \\
        2 & $\deshalb$ & 4. & $\Gsb(x_2,y)$ \\
        1 & $\deshalb$ & 5. & $\forall z (\Gsb(z,y) \to \Gspart(z,x_1))$ \\
        2 & $\deshalb$ & 6. & $\forall z (\Gsb(z,y) \to \Gspart(z,x_2))$ \\
        3,6 & $\deshalb$ & 7. & $\Gspart(x_1,x_2)$ \\
        4,5 & $\deshalb$ & 8. & $\Gspart(x_1,x_2)$\\
        7,8,A5 & $\deshalb$ & 9. & $x_1 = x_2$
    \end{longtable}
\end{bew}

\begin{konv}[The $\Ggrsb$-function]
    By $\Ggrsb(x)$ we denote the space entity $y$ with $\Ggrsb(y,x)$ (if existing).
\end{konv}


%%%%%%%%%%%%%%%%%%%%%%%%%%%%%%%%%%%%%%%%%%%%%%%%%%%

\begin{erin}[Mereologische Summe]\ \\
    $\Gsumn(x_{1},\ldots,x_{n},x) := \forall y\ (\Gsov(y,x) \leftrightarrow \bigvee^{n}_{i=1} \Gsov(y,x_{i}))$
\end{erin}

\begin{satz}\ \\
    Die mereologische Summe ist in ihrem letzten Argument eindeutig, d.h.
    \begin{align*}
        \Gsumn(x_{1},\ldots,x_{n},y_1) \land \Gsumn(x_{1},\ldots,x_{n},y_2) \to y_1 = y_2
    \end{align*}
\end{satz}

\begin{bew}\ \\
    \begin{longtable}{r c c l}
        & & 1. & $\Gsumn(x_{1},\ldots,x_{n},y_1)$\\
        & & 2. & $\Gsumn(x_{1},\ldots,x_{n},y_2)$\\
        1 & $\deshalb$ & 3. & $\forall z\ (\Gsov(z,y_1) \leftrightarrow \bigvee^{n}_{i=1} \Gsov(z,x_{i}))$ \\
        2 & $\deshalb$ & 4. & $\forall z\ (\Gsov(z,y_2) \leftrightarrow \bigvee^{n}_{i=1} \Gsov(z,x_{i}))$ \\
        3,4 & $\deshalb$ & 5. & $\forall z\ (\Gsov(z,y_1) \leftrightarrow \Gsov(z,y_2))$ \\
        5, T23. & $\deshalb$ & 6. & $y_1 = y_2$
    \end{longtable}
\end{bew}

\begin{konv}[Die $\Gsumn^f$-Funktion]\ \\
    Wir schreiben $\Gsumn^f(x_{1},\ldots,x_{n}) = y$ falls $\Gsumn(x_{1},\ldots,x_{n},y)$ gilt.
\end{konv}

%%%%%%%%%%%%%%%%%%%%%%%%%%%%%%%%%%%%%%%%%%%%%%%

\begin{erin}[Mereologischer Schnitt]\ \\
    \begin{align*}
        \Gintersectn(x_{1},\ldots,x_{n},x)  := \forall y\ (\Gspart(y,x) \leftrightarrow \bigwedge^{n}_{i=1} \Gspart(y,x_{i}))
    \end{align*}
\end{erin}

\begin{satz}\ \\
    Der mereologische Schnitt ist im letzten Argument eindeutig, d.h.
    \begin{align*}
        \Gintersectn(x_{1},\ldots,x_{n},y_1) \land \Gintersectn(x_{1},\ldots,x_{n},y_2) \to y_1 = y_2
    \end{align*}
\end{satz}

\begin{bew}\ \\
    \begin{longtable}{r c c l}
        & & 1. & $\Gintersectn(x_{1},\ldots,x_{n},y_1)$\\
        & & 2. & $\Gintersectn(x_{1},\ldots,x_{n},y_2)$\\
        1 & $\deshalb$ & 3. & $\forall z\ (\Gspart(z,y_1) \leftrightarrow \bigwedge^{n}_{i=1} \Gspart(z,x_{i}))$\\
        2 & $\deshalb$ & 4. & $\forall z\ (\Gspart(z,y_2) \leftrightarrow \bigwedge^{n}_{i=1} \Gspart(z,x_{i}))$ \\
        3,4 & $\deshalb$ & 5. & $\forall z\ (\Gspart(z,y_1) \leftrightarrow \Gspart(z,y_2))$ \\
        5, T1. & $\deshalb$ & 6. & $y_1 = y_2$
    \end{longtable}
\end{bew}

\begin{konv}[Die $\Gintersectn^f$-Funktion]\ \\
    Wir schreiben $\Gintersectn^f(x_{1},\ldots,x_{n}) = y$ falls $\Gintersectn(x_{1},\ldots,x_{n},y)$ gilt.
\end{konv}

%%%%%%%%%%%%%%%%%%%%%%%%%%%%%%%%%%%%%%%%%%%%%%%%%%%%%%%%%%

\begin{erin}[Relatives Komplement]\ \\
    \begin{align*}
        \Grelcompln(x_{1},\ldots,x_{n},x) := \hspace*{1em}\bigwedge_{1\leq i \leq n} \Geqdim(x_i, x) \wedge\\
        \forall y\ (\Gspart(y,x)
        \leftrightarrow \bigwedge^{n-1}_{i=1} \neg\Gsov(y,x_{i}) \wedge \Gspart(y, x_{n}))
    \end{align*}
\end{erin}

\begin{satz}\ \\
    Das relative Komplement ist im letzten Argument eindeutig, d.h.
    \begin{align*}
        \Grelcompln(x_{1},\ldots,x_{n},y_1) \land \Grelcompln(x_{1},\ldots,x_{n},y_2) \to y_1 = y_2
    \end{align*}
\end{satz}

\begin{bew}\ \\
    \begin{longtable}{r c c l}
        & & 1. & $\Grelcompln(x_{1},\ldots,x_{n},y_1)$\\
        & & 2. & $\Grelcompln(x_{1},\ldots,x_{n},y_2)$\\
        1 & $\deshalb$ & 3. & $\forall z\ (\Gspart(z,y_1)
        \leftrightarrow \bigwedge^{n-1}_{i=1} \neg\Gsov(z,x_{i}) \wedge \Gspart(z, x_{n}))$\\
        2 & $\deshalb$ & 4. & $\forall z\ (\Gspart(z,y_2)
        \leftrightarrow \bigwedge^{n-1}_{i=1} \neg\Gsov(z,x_{i}) \wedge \Gspart(z, x_{n}))$ \\
        3,4 & $\deshalb$ & 5. & $\forall z\ (\Gspart(z,y_1) \leftrightarrow \Gspart(z,y_2))$ \\
        5, T1. & $\deshalb$ & 6. & $y_1 = y_2$
    \end{longtable}
\end{bew}

\begin{konv}[Die $\Grelcompln^f$-Funktion]\ \\
    Wir schreiben $\Grelcompln^f(x_{1},\ldots,x_{n}) = y$ falls $\Grelcompln(x_{1},\ldots,x_{n},y)$ gilt.
\end{konv}



