verletzt A32'

\chapter{New Ideas}




\section{Grundlegende Definitionen}

\begin{dfn}[Gerichtete Zahl]\ \\
    Eine gerichtete Zahl ist ein Paar $(r,d)$ mit $r \in \R$, $d \in \udset$.\\
    Statt $(r,d)$ schreiben wir auch $r^d$.\\
    $\R^\ud$ bezeichnet die Menge der gerichteten Zahlen.
\end{dfn}


% \begin{dfn}[Beschränkte gerichtete Zahl]\ \\
%     Für $[a,b] \subseteq \R$ ist
%     \begin{align*}
%         [a,b]^\ud := \{r^d \in \R^\ud \mid r \in [a,b] \land (r = a \to d=\upa) \land (r=b \to d=\downa)\}
%     \end{align*}
% \end{dfn}
% 
% 
% \begin{bsp}
%     \begin{align*}
%         7\pi^\upa &\in \R^\ud \\
%         2^\downa &\in [0,7]^\ud \\
%         2^\downa &\in [0,2]^\ud \\
%         2^\upa &\notin [0,2]^\ud
%     \end{align*}
% \end{bsp}


\begin{dfn}[Die $<$-Relation auf $\{\upa, \downa\}$ und $\R^\ud$]\ \\
    \begin{enumerate}
        \item Für $d_1,d_2 \in \udset$ schreiben wir $d_1 < d_2$ falls $d_1 = \downa$ und $d_2 = \upa$ ist. 
        \item Für $r_1^{d_1},r_2^{d_2} \in \R^\ud$ schreiben wir $r_1^{d_1} < r_2^{d_2}$ falls $r_1 < r_2$ ist oder $r_1 = r_2$ und $d_1 < d_2$.
    \end{enumerate}
\end{dfn}

Wir benutzen die üblichen Notationen $x \leq y$ für $x < y \lor x = y$ und $x \nless y$ für $\neq x < y$.

Folgender Satz rechtfertigt die Nutzung des $<$-Symbols.
%
\begin{satz}
    Die $<$-Relation ist eine strenge lineare Ordung auf $\{\upa, \downa\}$ und $\R^\ud$, d.h. für $x,y,z \in \{\upa, \downa\}$ bzw. $x,y,z \in \R^\ud$ gelten
    \begin{itemize}
        \item $x = y \lor x < y \lor y < x$
        \item $x < y \to x \neq y \land y \nless x$
        \item $x < y \land y < z \to x < z$
    \end{itemize}
\end{satz}
Beweis trivial.


Die Festlegung eine linearen Ordung erlaubt uns die Nutzung der Intervallschreibweise auf $\R^\ud$ sowie der Operatoren $\min, \max, \sup$ und $\inf$ auf Teilmengen des $\R^\ud$ analog zu $\R$. 
Auf dieser Notation können wir die Menge der Basisintervalle einführen.
%
\begin{dfn}[Basisintervalle]\ \\
    Die Menge $\B$ der Basisintervalle ist definiert durch
    \begin{align*}
        \B = \{[r^\upa, s^\downa] \mid r,s \in \R, r<s\}.
    \end{align*}
    Für $\alpha \subseteq \R^\ud$ ist
    \begin{align*}
        \B_{\alpha} := \{ \beta \in \B \mid \beta \subseteq \alpha\}
    \end{align*}
    die Menge der und durch $\alpha$ beschränkten Basisintervalle.
\end{dfn}


\begin{bsp}
 \begin{align*}
  [-3^\upa, 2^\downa] &\in \B\\
  [6^\downa,9^\upa] &\notin \B\\
  [0.2^\upa, 0.7^\downa] &\in \B_{[0^\upa, 1^\downa]}\\
  [0^\upa, 1^\downa] &\in \B_{[0^\upa, 1^\downa]}\\
  [-1^\upa,1^\downa] &\notin \B_{[0^\upa, 1^\downa]}
 \end{align*}
\end{bsp}

\begin{dfn}[Grenzen]\ \\
    Die Funktion $\bnd \B: \to 2^{\R^\ud}$ wird definiert durch
    \begin{align*}
        \bnd([p,q])=\{p,q\}.
    \end{align*}
\end{dfn}

\begin{dfn}[Einfache Entitäten]\ %\\
    \begin{enumerate}
        \item Für $I \in \B$ ist die Menge
            \begin{align*}
                \langle I \rangle := I \times [0^\upa, 1^\downa]^2 \subseteq (\R^\ud)^3
            \end{align*}
            ein \emph{einfacher Körper}.\\ 
            Die Menge der einfachen Körper bezeichnen wir auch mit $\einf^3$.
        \item Für $p \in \R^\ud$, $I \in \B_{[0^\upa, 1^\downa]}$ ist die Menge
            \begin{align*}
                \langle p,I \rangle := \{p\} \times I \times [0^\upa, 1^\downa] \subseteq (\R^\ud)^3
            \end{align*}
            eine \emph{einfache Fläche}.\\
            Die Menge der einfachen Flächen bezeichnen wir mit $\einf^2$.
        \item Für $p \in \R^\ud$, $q \in [0^\upa,1^\downa]$, $I \in \B_{[0^\upa, 1^\downa]}$ ist die Menge
            \begin{align*}
                \langle p,q,I \rangle := \{p\} \times \{q\} \times I \times \subseteq (\R^\ud)^3
            \end{align*}
            eine \emph{einfache Linie}.\\
            Die Menge der einfachen Linien bezeichnen wir mit $\einf^1$.
        \item Für $p \in \R^\ud$, $q,r \in [0,1]^\ud$ ist die Menge
            \begin{align*}
                \langle p,q,r \rangle := \{(p,q,r)\} \subseteq (\R^\ud)^3
            \end{align*}
            ein \emph{einfacher Punkt}.\\
            Die Menge der einfachen Punkte bezeichnen wir mit $\einf^0$.
        %\item $\einf : \einf^3 \cup \einf^2 \cup \einf^1 \cup \einf^0$ ist die \emph{Menge der einfachen Entitäten}
    \end{enumerate}
    $\einf : \einf^3 \cup \einf^2 \cup \einf^1 \cup \einf^0$ ist die \emph{Menge der einfachen Entitäten}
\end{dfn}

\begin{dfn}[Base, Core]\ \\
    Wir definieren die Funktionen 
    \begin{align*}
        \base &: \einf \to (\R^\ud)^3\\ 
        \core &: \einf \to \B \cup \{\varnothing\}
    \end{align*}
    folgendermaßen:
    \begin{enumerate}
        \item Für $x = \langle I \rangle \in \einf^3$ sind
            \begin{align*}
                \base(x) &= \R^\ud \times [0^\upa, 1^\downa]^2\\
                \core(x) &= I.
            \end{align*}
        \item Für $x = \langle p, I \rangle \in \einf^2$ sind
            \begin{align*}
                \base(x) &= \{p\} \times [0^\upa, 1^\downa]^2\\
                \core(x) &= I
            \end{align*}
        \item Für $x = \langle p,q,I \rangle \in \einf^1$ sind
            \begin{align*}
                \base(x) &= \{p\} \times \{q\} \times [0^\upa, 1^\downa]^2\\
                \core(x) &= I
            \end{align*}
        \item Für $x = \langle p,q,r \rangle \in \einf^0$ sind
            \begin{align*}
                \base(x) &= \{(p,q,r)\}\\
                \core(x) &= \varnothing
            \end{align*}
    \end{enumerate}
\end{dfn}


\begin{satz}\label{satz:base-intv-ident}
    Für einfache Entitäten $x,y$ gilt
    \begin{align*}
        x = y \quad \Leftrightarrow \quad \base(x) = \base(y) \land \core(x) = \core(y).
    \end{align*}
\end{satz}
Beweis trivial.


\begin{dfn}[Euklidische Projektion]\ \\
    Für eine Menge $M \subseteq (\R^\ud)^3$ ist
    \begin{align*}
        \eukl(M) := \{(x,y,z) \in \R^3 \mid \exists\, def \in \{\upa, \downa\}: (x^d, y^e, z^f) \in M\}
    \end{align*}
    die \emph{euklidische Projektion} von $M$.
\end{dfn}


\begin{dfn}[Komplexe Entitäten]\
    \begin{enumerate}
        \item Ein komplexer Körper ist eine endliche Vereinigung einfacher Körper.\\
            Die \emph{Menge der komplexen Körper} bezeichnen wir mit $\univ^3$.
        \item Eine komplexe Fläche ist eine endliche Vereinigung einfacher Flächen.\\
            Die \emph{Menge der komplexen Flächen} bezeichnen wir mit $\univ^2$.
        \item Eine komplexe Linie ist eine endliche Vereinigung einfacher Linien.\\
            Die \emph{Menge der komplexen Linien} bezeichnen wir mit $\univ^1$.
        \item Eine komplexe Punktmenge ist eine endliche Vereinigung einfacher Punkte.\\
            Die \emph{Menge der komplexen Punktmengen} bezeichnen wir mit $\univ^0$.
    \end{enumerate}
    $\univ := \univ^3 \cup \univ^2 \cup \univ^1 \cup \univ^0$ ist die \emph{Menge der komplexen Entitäten}.
\end{dfn}


\begin{dfn}[Die Dimensionsfunktion]\ \\
    Wir definieren die Funktion $\Gdim : \univ \to \{0,1,2,3\}$ durch
    $$\Gdim(x) = d \quad \Leftrightarrow \quad x \in \univ^d.$$
\end{dfn}


\begin{dfn}[Höher- und niederdimensionale Entitäten]\ \\
    Eine komplexe Entität $x \in \univ$ heißt
    \begin{itemize}
        \item \emph{höherdimensional}, wenn $\Gdim(x) \neq 0$ ist und
        \item \emph{niederdimensional}, wenn $\Gdim(x) \neq 3$ ist.
    \end{itemize}
\end{dfn}

\begin{dfn}[Die $\ivor$-Relation]
    Für $I_1, I_2 \in \B$ gilt $I_1 \ivor I_2$ falls $\max(I_1) < \min(I_2)$ ist.
\end{dfn}
\todo{anderes Symbol einführen}


\begin{dfn}[Die $\vor$-Relation]\ 
    \begin{itemize}
%         \item Für $d_1,d_2 \in \udset$ gilt $d_1 \vor d_2$ falls $d_1 = \downa$ und $d_2 = \upa$ ist. 
%         \item Für $r_1^{d_1},r_2^{d_2} \in \R^\ud$ gilt $r_1^{d_1} \vor r_2^{d_2}$ falls $r_1 < r_2$ ist oder $r_1 = r_2$ und $d_1 \vor d_2$.
%        \item Für $I_1, I_2 \in \B$ gilt $I_1 \vor I_2$ falls $\max(I_1) < \min(I_2)$ ist.
        \item Für $x = \langle I_1 \rangle, y = \langle I_2 \rangle \in \einf^3$ gilt $x \vor y$ falls $I_1 \ivor I_2$ ist.
        \item Für $x = \langle p_1, I_1 \rangle, y = \langle p_2 I_2 \rangle \in \einf^2$ gilt $x \vor y$, falls $p_1 < p_2$ gilt oder $p_1 = p=2$ und $I_1 \ivor I_2$.
        \item Für $x = \langle p_1, q_1, I_1 \rangle, y = \langle p_2, q_2, I_2 \rangle \in \einf^1$ gilt $x \vor y$, falls $p_1 < p_2$ gilt oder $p_1 = p_2 \land  q_1 < q_2$ oder $p_1 = p_2 \land q_1 = q_2 \land I_1 \ivor I_2$.
        \item Für $x = \langle p_1, q_1, r_1 \rangle, y = \langle p_2, q_2, r_2 \rangle \in \einf^0$ gilt $x \vor y$, falls $p_1 < p_2$ gilt oder $p_1 = p_2 \land  q_1 < q_2$ oder $p_1 = p_2 \land q_1 = q_2 \land r_1 < r_2$.
    \end{itemize}
\end{dfn}


\begin{satz}\label{satz:teil-nicht-vor}
    Für einfache Entitäten $x,y$ mit $\base(x) = \base(y)$ und $\core(x) \subseteq
    \core(y)$ gilt weder $x \vor y$ noch $y \vor x$.
\end{satz}
Beweis trivial.


Für jede komplexe Entität $x$ gibt es eindeutig bestimmte einfache Entitäten $A_1, ..., A_n \in \einf^d$ mit
\begin{enumerate}
    \item $x = \bigcup_{i=1}^n y_i$.
    \item $y_A \vor ... \vor A_n$
\end{enumerate}
\todo[inline]{beweisen}
Damit können wir die kanonische Darstellung definieren.


\begin{dfn}[Kanonische Darstellung]\ \\
    Wir definieren eine Abbildung $\can: \univ \to \einf^*$ durch $\can(x) = A_1 ... A_n$ falls
    \begin{enumerate}
        \item $x = \bigcup_{i=1}^n A_i$.
        \item $A_1 \vor ... \vor A_n$
    \end{enumerate}
\end{dfn}

\begin{nota}[Konkatenation]\ \\
    Statt $x = A_1 ... A_n$ schreiben wir auch $x = \prod\limits_{i=1}^n A_i$.
\end{nota}


\begin{satzdef}\ \\
    Für $x \in \univ^3$ gibt es ein eindeutig bestimmtes $l \in \N$ und $l$ Basisintervalle $I_1, ..., I_n \in \B$ mit $I_1 < ... < I_l$ s.d.
    \begin{align*}
        x = \bigcup_{i=1}^l \langle I_i \rangle
    \end{align*}
    ist.\\ \ \\
    Wir definiere dann für $i \in \{1, ..., l\}$
    \begin{align*}
        \treewidth(x) &= l\\
        (x)_i &= I_i.
    \end{align*}
\end{satzdef}

\begin{satzdef}
    Für $x \in \univ^2$ gibt es
    \begin{enumerate}
        \item ein eindeutig bestimmtes $l \in \N$ und $l$ eindeutig bestimmte gerichtete Zahlen $p_1, ..., p_l \in \R^\ud$ mit $p_1 < ... < p_l$
        \item für jedes $i \in \{1, ..., l\}$ eine eindeutig bestimmte Zahl $m_i$ und $m_i$ eindeutig bestimmte beschränkte Basisintervalle $I_{i1}, ..., I_{im_i} \in \B_{[0^\upa, 1^\downa]}$ mit $I_{i1} < ... < I_{im_i}$
    \end{enumerate}
    so dass
    \begin{align*}
        x = \bigcup_{i=1}^l \bigcup_{j=1}^{m_i} \langle p_i, I_{ij} \rangle
    \end{align*}
    ist.\\ \ \\
    Wir definiere dann für $i \in \{1, ..., l\}$, $j \in \{1, ..., m_i\}$
    \begin{align*}
        \treewidth(x) &= l\\
        \treewidth_i(x) &= m_i\\
        (x)_i &= p_i\\
        (x)_{ij} &= I_{ij}.
    \end{align*}
\end{satzdef}

\begin{satzdef}
    Für $x \in \univ^1$ gibt es
    \begin{enumerate}
        \item ein eindeutig bestimmtes $l \in \N$ und $l$ eindeutig bestimmte gerichtete Zahlen $p_1, ..., p_l \in \R^\ud$ mit $p_1 < ... < p_l$
        \item für jedes $i \in \{1, ..., l\}$ eine eindeutig bestimmte Zahl $m_i$ und $m_i$ eindeutig bestimmte gerichtete Zahlen $q_{i1}, ..., q_{im_i} \in [0^\upa, 1^\downa]$ mit $q_{i1} < ... < q_{im_i}$
        \item für jedes $i \in \{1, ..., l\}$, $j \in \{1, ..., m_i\}$ eine eindeutig bestimmte Zahl $n_{ij}$ und $n_{ij}$ eindeutig bestimmte beschränkte Basisintervalle $I_{ij1}, ..., I_{ijm_{ij}} \in \B_{[0^\upa, 1^\downa]}$ mit $I_{ij1} < ... < I_{ijm_{ij}}$
    \end{enumerate}
    so dass
    \begin{align*}
        x = \bigcup_{i=1}^l \bigcup_{j=1}^{m_i} \bigcup_{k=1}^{n_{ij}} \langle p_i, q_{ij}, I_{ijk} \rangle
    \end{align*}
    ist.\\ \ \\
    Wir definiere dann für $i \in \{1, ..., l\}$, $j \in \{1, ..., m_i\}$ und $k \in \{1, ..., n_{ij}\}$
    \begin{align*}
        \treewidth(x) &= l\\
        \treewidth_i(x) &= m_i\\
        \treewidth_{ij}(x) &= n_{ij}\\
        (x)_i &= p_i\\
        (x)_{ij} &= q_{ij}\\
        (x)_{ijk} &= I_{ijk}.
    \end{align*}
\end{satzdef}


\begin{satzdef}
    Für $x \in \univ^0$ gibt es
    \begin{enumerate}
        \item ein eindeutig bestimmtes $l \in \N$ und $l$ eindeutig bestimmte gerichtete Zahlen $p_1, ..., p_l \in \R^\ud$ mit $p_1 < ... < p_l$
        \item für jedes $i \in \{1, ..., l\}$ eine eindeutig bestimmte Zahl $m_i$ und $m_i$ eindeutig bestimmte gerichtete Zahlen $q_{i1}, ..., q_{im_i} \in [0^\upa, 1^\downa]$ mit $q_{i1} < ... < q_{im_i}$
        \item für jedes $i \in \{1, ..., l\}$, $j \in \{1, ..., m_i\}$ eine eindeutig bestimmte Zahl $n_{ij}$ und $n_{ij}$ eindeutig bestimmte gerichtete Zahlen $r_{ij1}, ..., r_{ijm_{ij}} \in [0^\upa, 1^\downa]$ mit $r_{ij1} < ... < r_{ijm_{ij}}$
    \end{enumerate}
    so dass
    \begin{align*}
        x = \bigcup_{i=1}^l \bigcup_{j=1}^{m_i} \bigcup_{k=1}^{n_{ij}} \langle p_i, q_{ij}, r_{ijk} \rangle
    \end{align*}
    ist.\\ \ \\
    Wir definiere dann für $i \in \{1, ..., l\}$, $j \in \{1, ..., m_i\}$ und $k \in \{1, ..., n_{ij}\}$
    \begin{align*}
        \treewidth(x) &= l\\
        \treewidth_i(x) &= m_i\\
        \treewidth_{ij}(x) &= n_{ij}\\
        (x)_i &= p_i\\
        (x)_{ij} &= q_{ij}\\
        (x)_{ijk} &= r_{ijk}.
    \end{align*}
\end{satzdef}



\begin{dfn}[Grenzentität]\
    \begin{enumerate}
        \item Sei $x \in \univ^2$ mit $\can(x) = A_1 ... A_n$ und $A_i = \langle u_i^{d_i},I_i \rangle$ für $i \{1, ..., n\}$. Dann ist $x$ Grenzentität, falls für alle $i,j \in \{1, ..., n\}$ gilt
            \begin{align*}
                u_i = u_j \to d_i = d_j
            \end{align*}
        \item Sei $x \in \univ^1$ mit $\can(x) = A_1 ... A_n$ und $A_i = \langle p_i,u_i^{d_i},I_i \rangle$ für $i \{1, ..., n\}$. Dann ist $x$ Grenzentität, falls für alle $i,j \in \{1, ..., n\}$ gilt
            \begin{align*}
                p_i = p_j \land u_i = u_j \to d_i = d_j
            \end{align*}
        \item Sei $x \in \univ^0$ mit $\can(x) = A_1 ... A_n$ und $A_i = \langle p_i,q_i,u_i^{d_i} \rangle$ für $i \{1, ..., n\}$. Dann ist $x$ Grenzentität, falls für alle $i,j \in \{1, ..., n\}$ gilt
            \begin{align*}
                (p_i, q_i) = (p_j,q_j) \land u_i = u_j \to d_i = d_j
            \end{align*}
    \end{enumerate}
\end{dfn}
\todo[inline]{untersuchen ob $\GSB(x)$ gdw $x$ Grenzentität ist}

\begin{satz}\label{satz:einfachgrenze}
    Alle niederdimensionale einfache Entitäten sind Grenzentitäten.
\end{satz}
Beweis trivial.


%%%% nicht drin (oder anders) %%%%

% \begin{dfn}[Untere und obere Grenzen]\ \\
%     Wir definieren die Funktionen $\lb, \ub: \B \to \R^\ud$ folgendermaßen:\\
%     Für $[a,b) \in \B$ sind
%     \begin{align*}
%         \lb([a,b)) = a^\upa,\ \ub([a,b)) = b^\downa
%     \end{align*}
%     Das Prädikat $\bnd \subseteq \R^\ud \times \B$ wird definiert durch
%     \begin{align*}
%         \bnd(x,y) \quad \Leftrightarrow \quad \lb(y)=x \lor \ub(y)=x
%     \end{align*}
% \end{dfn}
% 
% \begin{dfn}[Einfache Entitäten]\ %\\
%     \begin{enumerate}
%         \item Ein \emph{einfacher Körper} ist ein Basisintervall.\\ 
%             Die Menge der einfachen Körper bezeichnen wir auch mit $\einf^3$.
%         \item Eine \emph{einfache Fläche} ist ein Paar $(A,B) \in \R^\ud \times \B_{[0,1]}$.\\
%             Die Menge der einfachen Flächen bezeichnen wir mit $\einf^2$.
%         \item Eine \emph{einfache Linie} ist ein Tripel $(A,B,C) \in \R^\ud \times [0,1]^\ud \times \B_{[0,1]}$.\\
%             Die Menge der einfachen Linien bezeichnen wir mit $\einf^1$.
%         \item Ein \emph{einfacher Punkt} ist ein Tripel $(A,B,C) \in \R^\ud \times [0,1]^\ud \times [0,1]^\ud$.\\
%             Die Menge der einfachen Punkte bezeichnen wir mit $\einf^0$.
%         \item $\einf : \einf^3 \cup \einf^2 \cup \einf^1 \cup \einf^0$ ist die \emph{Menge der einfachen Entitäten}
%     \end{enumerate}
% \end{dfn}
% 
% \begin{dfn}[Base, Core, euklidische Projektion]\ \\
%     Wir definieren die Funktionen 
%     $$\base: \einf \to (\R^\ud)^*,\ \core: \einf \to \B \cup {\varnothing},\ \eukl: \einf \to 2^{\R^3}$$
%     folgendermaßen:
%     \begin{enumerate}
%         \item Für $x \in \einf^3$ sind
%             \begin{align*}
%                 \eukl(x) &= x \times [0,1]^2\\
%                 \base(x) &= \varepsilon\\
%                 \core(x) &= x
%             \end{align*}
%         \item Für $x = (r^d, I) \in \einf^2$ sind
%             \begin{align*}
%                 \eukl(x) &= \{r\} \times I \times [0,1]\\
%                 \base(x) &= r^d\\
%                 \core(x) &= I
%             \end{align*}
%         \item Für $x = (r^d, s^e, I) \in \einf^1$ sind
%             \begin{align*}
%                 \eukl(x) &= \{r\} \times \{s\} \times I\\
%                 \base(x) &= r^d s^e\\
%                 \core(x) &= I
%             \end{align*}
%         \item Für $x = (r^d, s^e, t^f) \in \einf^0$ sind
%             \begin{align*}
%                 \eukl(x) &= \{(r,s,t)\}\\
%                 \base(x) &= r^d s^e t^f\\
%                 \core(x) &= \varnothing
%             \end{align*}
%     \end{enumerate}
% \end{dfn}
% 
% \begin{dfn}[Ausweitung der euklidischen Projektion auf $\univ$]
%     Für $x = \bigcup_{i=1}^n A_i$ ist
%     \begin{align*}
%         \eukl(x) = \bigcup_{i=1}^n \eukl(A_i)
%     \end{align*}
% \end{dfn}
% 
% \begin{dfn}[Die $\vor$-Relation]\ 
%     \begin{itemize}
%         \item Für $d_1,d_2 \in \udset$ gilt $d_1 \vor d_2$ falls $d_1 = \downa$ und $d_2 = \upa$ ist. 
%         \item Für $r_1^{d_1},r_2^{d_2} \in \R^\ud$ gilt $r_1^{d_1} \vor r_2^{d_2}$ falls $r_1 < r_2$ ist oder $r_1 = r_2$ und $d_1 \vor d_2$.
%         \item Für $[a_1,b_1), [a_2,b_2) \in \einf^3$ gilt $[a_1,b_1) \vor [a_2,b_2)$ falls $b_1 < a_2$ ist.
%         \item Für $(A_1,B_1), (A_2,B_2) \in \einf^2$ gilt $(A_1,B_1) \vor (A_2,B_2)$, falls $A_1 \vor A_2$ gilt oder $A_1 = A_2$ und $B_1 \vor B_2$.
%         \item Für $(A_1,B_1,C_1), (A_2,B_2,C_2) \in \einf^1 \cup \einf^0$ gilt $(A_1,B_1) \vor (A_2,B_2)$, falls $A_1 \vor A_2$ gilt oder $A_1 = A_2 \land B_1 \vor B_2$ oder $A_1 = A_2 \land B_1 = B_2 \land C_1 \vor C_2$.
%     \end{itemize}
% \end{dfn}


\section{Die Blockstruktur}


\begin{dfn}[$\GSReg$]
    $\GSReg^B := \univ^3$
\end{dfn}


\begin{dfn}[$\Gsb^\einf$]\ 
    \begin{itemize}
        \item Für $x = \langle p_1, I_1 \rangle \in \einf^2$, $y = \langle I_2 \rangle \einf^3$ gilt
            \begin{align*}
                \Gsb^\einf(x,y) 
                \quad \Leftrightarrow \quad 
                \bnd(p_1,I_2).
            \end{align*}
        \item Für $x = \langle p_1,q_1,I_1 \alpha \rangle \in \einf^1$, $y = \langle p_2,I_2 \rangle \in \einf^2$ gilt
            \begin{align*}
                \Gsb^\einf(x,y) 
                \quad \Leftrightarrow \quad 
                p_1 = p_2 \land \bnd(q_1,I_2).
            \end{align*}
        \item Für $x = \langle p_1,q_1,r_1 \rangle \in \einf^0$, $y = \langle p_2,q_2,I_2 \rangle \in \einf^1$ gilt
            \begin{align*}
                \Gsb^\einf(x,y) 
                \quad \Leftrightarrow \quad 
                p_1 = p_2 \land q_1 = q_2 \land \bnd(r_1,I_2).
            \end{align*}
    \end{itemize}
\end{dfn}


\begin{dfn}[$\Gsb$]\ \\
    Für $d \in \{0,1,2\}$, $x \in \univ^d$, $y \in \univ^{d+1}$ mit $\can(x) = A_1 ... A_n \in \univ^d$, $\can(y) = B_1 ... B_m \in \univ^{d+1}$ gilt:
    \begin{align*}
        \Gsb(x,y) \quad \Leftrightarrow \quad \forall i \in \{1, ..., n\}\ \exists j \in \{1, .., m\}: \Gsb^\einf(A_i,B_j).
    \end{align*}
\end{dfn}


\begin{dfn}[$\Gscoinc$]\ \\
    Für $x,y \in \univ^d$ gilt $\Gscoinc(x,y)$ falls
    \begin{enumerate}
        \item $d \neq 3$ ist
        \item $x$ und $y$ Grenzentitäten sind und
        \item $\eukl(x) = \eukl(y)$.
    \end{enumerate}
\end{dfn}


\begin{dfn}[$\Gspart^\einf$]\ \\ 
    Für $x,y \in \univ$ gilt $\Gspart^\einf(x,y)$, gdw.
    \begin{enumerate}
        \item $\base(x) = \base(y)$ und
        \item $\core(x) \subseteq \core(y)$
    \end{enumerate}
    sind.
\end{dfn}


\begin{dfn}[$\Gspart$]\ \\
    Für $d \in \{0,1,2,3\}$, $m,n \in \N$, $x,y \in \univ^d$ mit $\can(x) = A_1 ... A_n$, $\can(y) = B_1 ... B_m$ gilt:
    $$\Gspart(x,y) \quad \Leftrightarrow \quad \forall i \in \{1, ..., n\}\ \exists j \in \{1, .., m\}: \Gspart^\einf(A_i,B_j). $$
\end{dfn}


\subsection{Defined Relations and Concepts}
In this section we investigate the intensions of the defined $\strukt$ relations (including unary relations i.e. concepts)

\begin{erin}[Two dimensional boundary]\ \\
    $\Gtwodb(x,y) := \GSReg(y) \wedge \Gsb(x,y)$
\end{erin}

\begin{hyp}[Two dimensional boundary]
    \begin{align*}
        &\Gtwodb(x,y)\\
        \Leftrightarrow \quad &x \in \univ^2 \land \Gsb(x,y)\\
        \Leftrightarrow \quad &x \in \univ^2 \land y \in \univ^3 \land \Gsb(x,y)\\
        \Leftrightarrow \quad &x = \langle p, I \rangle \in \univ^2 \land y = \langle J \rangle \in \univ^3 \land \bnd(p, J)\\
        \Leftrightarrow \quad &x = \langle p, I \rangle \in \univ^2 \land y = \langle [q, r] \rangle \in \univ^3 \land p \in \{q,r\}
    \end{align*}
\end{hyp}

%%%%%%%%%%%%%%%%%%%%%%%%%%%%%%%%%%%%%%%%%%%%%%%%%%%

\begin{erin}[One dimensional boundary]\ \\
    $\Gonedb(x,y) := \GtwoDB(y) \wedge \Gsb(x,y)$
\end{erin}

\begin{hyp}[One dimensional boundary]
    \begin{align*}
        &\Gonedb(x,y)\\
        \Leftrightarrow \quad &x \in \univ^1 \land \Gsb(x,y)\\
        \Leftrightarrow \quad &x \in \univ^1 \land y \in \univ^2 \land \Gsb(x,y)\\
        \Leftrightarrow \quad &x = \langle r^d, s^e, I \rangle \in \univ^1 \land y \in \univ^2 \land r^d = \base(y) \land \bnd(s^e, \core(y))\\
        \Leftrightarrow \quad &x = \langle r^d, s^e, I \rangle \in \univ^1 \land y = \langle r^d,J \rangle \in \univ^2 \land \bnd(s^e, J)\\
        \Leftrightarrow \quad &x = \langle r^d, s^e, I \rangle \in \univ^1 \land y = \langle r^d,[a^\upa,b^\downa] \rangle \in \univ^2 \land (e = \upa \ \to\ s = a) \land (e = \downa \ \to\ s = b)
    \end{align*}
\end{hyp}


\begin{erin}[Zero dimensional boundary]\ \\
    $\Gonedb(x,y) := \GtwoDB(y) \wedge \Gsb(x,y)$
\end{erin}

\begin{hyp}[One dimensional boundary]
    \begin{align*}
        &\Gzerodb(x,y)\\
        \Leftrightarrow \quad &x \in \univ^0 \land \Gsb(x,y)\\
        \Leftrightarrow \quad &x \in \univ^0 \land y \in \univ^1 \land \Gsb(x,y)\\
        \Leftrightarrow \quad &x = \langle r^d, s^e, t^f \rangle \in \univ^0 \land y \in \univ^1 \land (r^d,s^e) = \base(y) \land \bnd(t^f, \core(y))\\
        \Leftrightarrow \quad &x = \langle r^d, s^e, t^f \rangle \in \univ^1 \land y = \langle r^d, s^e, J \rangle \in \univ^2 \land \bnd(t^f,J)\\
        \Leftrightarrow \quad &x = \langle r^d, s^e, t^f \rangle \in \univ^1 \land y = \langle r^d, s^e, [a^\upa,b^\downa] \rangle \in \univ^2 \land (f = \upa \ \to\ t = a) \land (f = \downa \ \to\ t = b)
    \end{align*}
\end{hyp}

%%%%%%%%%%%%%%%%%%%%%%%%%%%%%%%%%%%%%%%%%%%%%%%%%%%%%%%%%%%%%%%%

\begin{erin}[Spatial boundary]\ \\
    $\GSB(x) := \exists y\ \Gsb(x,y)$
\end{erin}

\begin{satz}[Spatial boundary]\ \\
    Es gilt $\GSB(x)$ gdw. $x$ eine Grenzentität ist.
\end{satz}

\begin{bew}
    Klar ist: wenn $x \in \univ^3$ ist, gibt es kein $y \in \univ$ mit $\Gsb(x,y)$ und $x$ ist auch keine Grenzentität.\\
    %
    Wir zeigen den Satz nun für $x \in \univ^2$\\
    Sei $\can(x) =: A_1 ... A_m$ mit $A_i =: \langle u_i^{d_i}, I_i \rangle$ für $i \in \{1, ..., m\}$.\\ \ \\
    $\boldsymbol{\Rightarrow}$: 
    Es gelte $\Gsb(x,y)$.
    Dann ist $y \in \univ^3$.
    Sei $\can(y) =: B_1 ... B_n$ mit $B_k =: \langle [a_k^\upa,b_k^\downa] \rangle$ für $k \in \{1, ..., n\}$.
    Für $i \in \{1, ..., m\}$ sei $k(i) \in \{1, ..., n\}$ s.d. $\Gsb^\einf(A_i,B_{k(i)})$ gilt.
    Angenommen $x$ ist keine Grenzentität.
    Dann gibt es $i,j \in \{1, ..., m\}$ mit $u_i = u_j$ und $d_i \neq d_j$. O.B.d.A: sei $i<j$. Dann ist $A_i \vor A_j$ und somit $d_i = \downa$, $d_j = \upa$.
    Also ist $b_{k(i)} = u_i = u_j = a_{k(j)}$ und somit $B_{k(i)} \neq B_{k(j)}$. Gleichzeitig gilt weder $B_{k(i)} \vor B_{k(j)}$ noch $B_{k(j)} \vor B_{k(i)}$. 
    Somit ist $B_1 ... B_n$ keine kanonische Darstellung. $\lightning$\\ \ \\
    $\boldsymbol{\Leftarrow}$:
    Mit $u_0 := u_1-1$ und $u_{m+1} := u_m+1$ definieren wir für $i \in \{1, ..., m\}$
    %
    \begin{align*}
        a_i := 
            \begin{cases}
                u_i & \text{falls $d_i = \upa$}\\
                \frac{2u_i + u_{i-1}}{3} & \text{falls $d_i = \downa$}
            \end{cases}\\
        b_i := 
            \begin{cases}
                \frac{2u_i + u_{i+1}}{3} & \text{falls $d_i = \upa$}\\
                u_i & \text{falls $d_i = \downa$}
            \end{cases}
    \end{align*}
    Setze nun $B_i := \langle [a_i^\upa, b_i^\downa] \rangle$ und $y := \bigcup_{i=1}^m B_i$.
    Nach Konstruktion gilt $\Gsb^\einf(A_i, B_i)$.
    Da $x$ Grenzentität ist, ist $u_{i}^{d_i} = u_{i+1}^{d_{i+1}}$ oder $u_i < u_{i+1}$ und somit   $B_i \vor B_{i+1} \lor B_i = B_{i+1}$.
    Also ist $B_1 ... B_m$ bis auf eventuelle Dopplungen kanonische Darstellung von $y$ und es gilt $\Gsb(x,y)$.\\ \ \\
    %
    Für $x \in \univ^1 \cup \univ^0$ funktionieren die Hinrichtungen analog. Für die Rückrichtungen ist die Baumdarstellung von $x$ hilfreich. Wir führen hier lediglich beispielhaft die Konstruktion von $y$ für $x \in \univ^1$ an:\\
    Sei $\treewidth(x) =: l$.
    Für $i \in \{1, ..., l\}$ seien $\treewidth_i(x) =: m_i$, $(x)_i =: p_i$, $u_{i0} := 0$ und $u_{im_i+1} := 1$.
    Für $j \in \{1, ..., m_i\}$ sei $(x)_{ij} =: u_{ij}^{d_{ij}}$.
    Definiere nun
    \begin{align*}
        a_{ij} := 
            \begin{cases}
                u_{ij} & \text{falls $d_{ij} = \upa$}\\
                \frac{2u_{ij} + u_{ij-1}}{3} & \text{falls $d_{ij} = \downa$}
            \end{cases}\\
        b_{ij} := 
            \begin{cases}
                \frac{2u_{ij} + u_{ij+1}}{3} & \text{falls $d_{ij} = \upa$}\\
                u_{ij} & \text{falls $d_i = \downa$}
            \end{cases}
    \end{align*}
    Setze nun $J_{ij} := [a_{ij}^\upa, b_{ij}^\downa]$ und $y := \bigcup_{i=1}^{l} \bigcup_{j=1}^{m_i} \langle p_i,J_{ij} \rangle$.
\end{bew}


%%%%%%%%%%%%%%%%%%%%%%%%%%%%%%%%%%%%%%%%%%%%%%%%%%

\begin{erin}[$\Gd$-dimensional boundary]\ \\
    $\GdDB(x) := \exists y\ \Gddb(x,y)$
\end{erin}

\begin{hyp}[$\Gd$-dimensional boundary]\ \\
    Es gilt $\GdDB(x)$ gdw. $x \in \univ^{\Gd}$ und eine Grenzentität ist.
    %Es gilt $\GdDB(x)$ gdw. $x \in \univ^d$ und eine Grenzentität ist.
\end{hyp}

%%%%%%%%%%%%%%%%%%%%%%%%%%%%%%%%%%%%%%%%%%%%%%%%%%%%%%%%%%%%%%%%

\begin{erin}[Greatest spatial boundary]
    \begin{align*}
        &\Ggrsb(x,y) := \Gsb(x,y) \wedge \forall z\ (\Gsb(z,y) \to \Gspart(z,x))\\
        &\Ggrsb(y) = x \quad \Leftrightarrow \quad \Ggrsb(x,y)
    \end{align*}
\end{erin}

\begin{hyp}[Greatest spatial boundaries of simple entities]\ 
    \begin{itemize}
        \item Für $x = [a,b) \in \einf^3$ gilt
            \begin{align*}
                \Ggrsb(x) = (a^\upa, [0,1)) \cup (b^\downa, [0,1))
            \end{align*}
        \item Für $x = (r^d,[a,b)) \in \einf^2$ gilt
            \begin{align*}
                \Ggrsb(x) = (r^d, a^\upa, [0,1)) \cup (r^d, b^\downa, [0,1))
            \end{align*}
        \item Für $x = (r^d,e^f,[a,b)) \in \einf^1$ gilt
            \begin{align*}
                \Ggrsb(x) = (r^d,e^f, a^\upa) \cup (r^d, e^f, b^\downa)
            \end{align*}
    \end{itemize}
\end{hyp}

\begin{hyp}[Greatest spatial boundary]\ \\
    Für $x \in \univ$ mit $\can(x) = A_1 ... A_n$ ist
    \begin{align*}
        \Ggrsb(x) = \bigcup_{i=1}^n \Ggrsb(A_i)
    \end{align*}
\end{hyp}




%         \itemT[D7.]{\GLDE(x) := \exists y\ (\GSB(y) \wedge \Gspart(y, x))}
%                 {$x$ ist eine niederdimensionale Raumentität}
% 
%         \itemT[D8.]{\GdDE(x) := \exists y\ (\GdDB(y) \wedge \Gspart(y, x))}
%                 {$x$ ist ein Flächen-/Linien-/Punktregion}
%                                                                                                 
%         \itemTP[D9.]{$\Geqdim(x,y) := (\GSReg(x) \wedge \GSReg(y)) \vee \mbox{}$\hfill ($x$ und $y$ haben die selbe Dimension)\\
%                 \mbox{}\hphantom{$\Geqdim(x,y) :=$}
%                 $(\GtwoDE(x) \wedge \GtwoDE(y)) \vee
%                             (\GoneDE(x) \wedge \GoneDE(y)) \vee
%                             (\GzeroDE(x) \wedge \GzeroDE(y))$}{}
% 
%         \itemT[D10.]{\Gsppart(x, y) := \Gspart(x, y) \land  x \neq y}
%                 {$x$ ist ein echter Teil von $y$}
% 
%         
%         \itemT[D11.]{\Gsov(x, y) := \exists z\  (\Gspart(z, x) \wedge \Gspart(z, y))}
%                 {$x$ und $y$ überlappen}
% 
%         
%         \itemTP[D12.]{$\Gsumn(x_{1},\ldots,x_{n},x) := \forall y\ (\Gsov(y,x) \leftrightarrow \bigvee^{n}_{i=1} \Gsov(y,x_{i}))$\\ \mbox{}}
%                 {$x$ ist die mereologische Summe von $x_{1}$,\ldots,$x_{n}, n \ge 2$}
%                 
%         \itemTP[D13.]{$\Gintersectn(x_{1},\ldots,x_{n},x)  := \forall y\ (\Gspart(y,x) \leftrightarrow \bigwedge^{n}_{i=1} \Gspart(y,x_{i}))$\\ \mbox{}}
%                 {$x$ ist der mereologische Schnitt von $x_{1}$,\ldots,$x_{n}, n \ge 2$}
%                 
%         \itemTP[D14.]{$\Grelcompln(x_{1},\ldots,x_{n},x) := \mbox{}$\hfill
%                             ($x$ ist das rel. Kompl. von $x_{n}$ und $x_{1}$,\ldots,$x_{n-1}, n \ge 2$)\\
%                 $\hspace*{1em}\bigwedge_{1\leq i < j \leq n} \Geqdim(x_i, x) \wedge
%                 \forall y\ (\Gspart(y,x) \leftrightarrow \bigwedge^{n-1}_{i=1} \neg\Gsov(y,x_{i}) \wedge \Gspart(y, x_{n}))$}
%                 {}  
% 
%                         
%         \itemTP[D15.]{$\Gpartitionn(x_{1},\ldots,x_{n},x) := \Gsumn(x_{1},\ldots,x_{n},x) \wedge \bigwedge_{1 \leq i < j \leq n} \neg\Gsov(x_i,x_j)$\\ \mbox{} \label{D_partitionn}}
%         {$x_{1}$,\ldots,$x_{n}$ zerlegen $x$, $n \ge 2$}			
% 
%         \itemT[D16.]{\Gtwodhypp(x,y) := \exists z\ (\Gspart(z,y) \wedge \Gtwodb(x,z))}
%                 {$x$ ist ein 2-dim.\ Hyperteil von $y$}
%         
%         \itemTP[D17.]{$\Gonedhypp(x,y) := \exists z\ ((\Gspart(z,y) \vee \Gtwodhypp(z,y)) \wedge  \Gonedb(x,z)) $\\ \mbox{}}
%                 {$x$ ist ein 1-dim.\ Hyperteil von $y$} 
%         
%         \itemTP[D18.]{$\Gzerodhypp(x,y) := \exists z\ ((\Gspart(z,y) \vee \Gonedhypp(z,y)) \wedge  \Gzerodb(x,z)) $\\ \mbox{}}
%                 {$x$ ist ein 0-dim.\ Hyperteil von $y$}                  
%                 
%         \itemT[D19.]{\Ghypp(x,y) := \Gtwodhypp(x,y) \vee \Gonedhypp(x,y) \vee \Gzerodhypp(x,y)}
%                 {$x$ ist ein Hyperteil von $y$}    
% 
%         \itemTP[D20.]{$\Gtangpart(x,y) := (\Gspart(x,y) \vee \Ghypp(x,y))\ \ \wedge\ \ \mbox{}$
%                                         \hfill ($x$ ist ein tangentialer Teil von $y$)\\
%                         \hspace*{1em}$\exists x'zz'\ ((\Gspart(x',x) \vee \Ghypp(x',x)) \wedge \mbox{}$\\
%                                 \hspace*{5.2em}$\Gsb(z,y) \wedge (\Gspart(z',z) \vee \Ghypp(z',z)) \wedge \Gscoinc(x',z'))$}
%                 {}  
% 
%         \itemT[D21.]{\Ginpart(x, y) := (\Gspart(x,y) \vee \Ghypp(x,y)) \land \neg \Gtangpart(x, y)}
%                 {\mbox{$x$ ist ein innerer Teil von $y$}}
% 
%         \itemTP[D22.]{$\GExOrd(x) := 
%                         \exists yz\ (\Gspart(y,x) \wedge \Gspart(z,x) \wedge
%                                                 \neg\Gsov(y,z) \wedge \Gscoinc(y,z))$\\ \mbox{}}
%                 {$x$ ist eine extraordinäre Raumentität}
%                 
%         \itemT[D23.]{\GOrd(x) := \neg\GExOrd(x)}
%                     {$x$ ist eine ordinäre Raumentität}	
% 
%         \itemTP[D24.]{$\GdDC(x) := \exists uv\ (\Gpartition(u,v,x)) \wedge  \mbox{}$
%                     \hfill($x$ ist $\Gd$-dim.\ zusammenhängend)
%                 \hspace*{1em}$\forall yz\ (\Gpartition(y,z,x) \to 
%                         \exists y'z'(\Gddhypp(y',y) \wedge \Gddhypp(z',z) \wedge \Gscoinc(y',z')))$
%                         \label{D:dDC}}
%                     {}
% 
%         \itemT[D25.]{\GC(x) := \GtwoDC(x) \vee \GoneDC(x) \vee \GzeroDC(x)}
%                     {$x$ ist zusammenhängend} 
%                     
%         \itemT[D26.]{\Gc(x,y) := \exists z\ (\Gsum(x,y,z) \wedge \GC(z))}
%                     {$x$ und $y$ sind zusammenhängend} 
% 
%         \itemT[D27.]{\Gexc(x,y) := \Gc(x,y) \wedge \neg\Gsov(x,y)}
%                     {$x$ und $y$ sind extern zusammenhängend}
% 
%         \itemT[D28.]{\GoneCCDd(x) := \GdDC(x)}
%                 {$x$ hat eine $\Gd$-dim.\ zusammenhängende Komponente}
% 
%         \itemTP[D29.]{$\GnCCDd(x) := \bigwedge_{i=1}^{n-1} \neg\GiCCDd(x) \wedge
%                     \exists x_{1}...x_{n}(\Gpartitionn(x_{1},...,x_{n},x) \wedge \bigwedge_{i=1}^{n} \GoneCCDd(x_{i}))$\\ \mbox{}
%                                 \label{D:nCCDd}}
%                 {$x$ hat $n$ $\Gd$-dim.\ zusammenhängende Komponenten}
% 
%         \itemT[D30.]{\GTop(x) := \GSReg(x) \wedge \GOrd(x) \wedge \GtwoDC(x)}
%                     {$x$ ist ein Topoid} 
%                     
%         \itemT[D31.]{\GtwoD(x) := \GtwoDE(x) \wedge \GOrd(x) \wedge \GoneDC(x)}
%                     {$x$ ist eine Fläche}  
%                                 
%         \itemT[D32.]{\GoneD(x) := \GoneDE(x) \wedge \GOrd(x) \wedge \GzeroDC(x)}
%                     {$x$ ist eine Linie}
%                                 
%         \itemT[D33.]{\GzeroD(x) := \GzeroDE(x) \wedge \GOrd(x) \wedge \neg\exists y\ \Gsppart(y,x)}
%             {$x$ ist ein Punkt} 
%             
%         \itemTP[D34.]{$\Gstrictsb(x,y) \Ldef 
%            \Gsb(x,y) \wedge 
% 					 \forall x' (\Ghypp(x',y) \wedge \Gscoinc(x,x') \to x=x')$
% 					\\ \mbox{} \label{def:strictboundary}}
%        {$x$ is a strict spatial boundary}
%        
%        \itemT[D35.]{\Gweaksb(x,y) \Ldef \Gsb(x,y) \wedge \neg\Gstrictsb(x,y) }
%             {$x$ is a weak spatial boundary}
%     \end{enumAx}
%     
%     &\Gonedircomp(x,y) := \GzeroD(x) \wedge \GzeroD(y) \wedge \Gscoinc(x,y) \wedge \mbox{}
%         \\ &\hspace{1em}
%         \exists x',y',x'',y''\ (\Gsb(x,x') \wedge \Gsb(x',x'') \wedge \Gsb(y,y') \wedge \Gsb(y',y'') \wedge
%         \\ &\hspace{2em}
%         \neg \Gsov(x',y') \wedge \neg \Gsov(x'', y'') \wedge \mbox{}
%         \\ &\hspace{2em}
%         \forall z\ (\Gscoinc(x,z) \wedge \Ghypp(z, \Ggrsb_f(\Gsum_f(x'',y''))) \to z=x \vee z=y))
%     
%     \begin{enumAx}[D]
% \itemTP{$\Gonedircomp(x,y) \Ldef \Gscoinc(x, y) \wedge \mbox{}$\\
% \hspace*{1em}
% 	$\exists x' y' z'\: 
% 	 (\:\GOrd(x') \wedge
% 	    \GOrd(y') \wedge
% 		  \Gsb(x, x') \wedge
% 			\Gsb(y, y') \wedge
%       \neg \Gsov(x', y') \wedge	\mbox{}$\\
%   \hspace*{2.5em}
%     $ ( \GSB(x') \wedge \GSB(y') \Limp \mbox{}$\\
% 	\hspace*{4em}
% 			 $\exists x'' y'' z'' g' \:
% 			  (\:\Gsb(x', x'') \wedge
% 					 \Gsb(y', y'') \wedge
%            \neg \Gsov(x'', y'') \wedge \mbox{}$\\
% 	\hspace*{6em}
% 	     $	 \Gsum(x'', y'', z'') \wedge
% 					 \Ggrsb(g', z'') \wedge
% 					 \Gzerodhypp(x, g') \wedge 
% 					 \Gzerodhypp(y, g') \wedge \mbox{}$\\
% 	\hspace*{6em}
% 				 $ \forall z (  \Gzerodhypp(z, g') \Limp \mbox{}$\\
% 	\hspace*{8em}		
% 					  $((\Gscoinc(x, z) \wedge \neg\Gsov(x,z) \Limp y=z ) \wedge \mbox{}$\\
% 	\hspace*{8em}\hphantom{$($}		
% 						$(\Gscoinc(y, z) \wedge \neg\Gsov(y,z) \Limp x=z ) 
% 					\ )))))$\\ \mbox{} \label{def:npodircomp}}
%   {$x$ and $y$ are 1-directionally compatible}
% 						
% \itemTP{$\Gonecont(x) \Ldef \GoneDE(x) \wedge \mbox{}$\\
% \hspace*{1em}
%   $\forall y\:(\:(\Gzerodhypp(y,x) \wedge
% 	                \forall y' ( \Gspart(y',y) \to \neg\Gstrictsb(y',x))) \Limp \mbox{} $\\
% 	\hspace*{2.5em}	
% 	  $\exists z ( \Gzerodhypp(z,x) \wedge
% 		             \neg\Gsov(z,y) \wedge 
%                  \Gonedircomp(y,z)))$\\ \mbox{} \label{def:Gdcont}}
%              {$x$ is 1-continuous}	
% \end{enumAx}



\subsection{Untersuchung ausgewählter definierter Relationen}

\subsubsection{$\GSB$}

\begin{satz}
    Für $x \in \univ$ gilt $\GSB(x)$ gdw. $x$ eine Grenzentität ist.
\end{satz}

\begin{bew} (für $x \in \univ^2$)\\
    Sei $\can(x) =: A_1 ... A_m$ mit $A_i =: (r_i^{d_i}, I_i)$ für $i \in \{1, ..., m\}$.\\ \ \\
    $\boldsymbol{\Rightarrow}$: 
    Es gelte $\Gsb(x,y)$.
    Dann ist $y \in \univ^3$.
    Sei $\can(y) =: B_1 ... B_n$ mit $B_k =: [a_k,b_k)$ für $k \in \{1, ..., n\}$.
    Für $i \in \{1, ..., m\}$ sei $k(i) \in \{1, ..., n\}$ s.d. $\Gsb^\einf(A_i,B_{k(i)})$ gilt.
    Angenommen $x$ ist keine Grenzentität.
    Dann gibt es $i,j \in \{1, ..., m\}$ mit $r_i = r_j$ und $d_i \neq d_j$. O.B.d.A: sei $i<j$. Dann ist $A_i \vor A_j$ und somit $d_i = \downa$, $d_j = \upa$.
    Also ist $b_{k(i)} = r_i = r_j = a_{k(j)}$ und somit $B_{k(i)} \neq B_{k(j)}$. Gleichzeitig gilt weder $B_{k(i)} \vor B_{k(j)}$ noch $B_{k(j)} \vor B_{k(i)}$. 
    Somit ist $B_1 ... B_n$ keine kanonische Darstellung. $\lightning$\\ \ \\
    $\boldsymbol{\Leftarrow}$:
    Mit $r_0 := r_1-1$ und $r_{m+1} := r_m+1$ definieren wir für $i \in \{1, ..., m\}$
    %
    \begin{align*}
        a_i := 
            \begin{cases}
                r_i & \text{falls $d_i = \upa$}\\
                \frac{2r_i + r_{i-1}}{3} & \text{falls $d_i = \downa$}
            \end{cases}\\
        b_i := 
            \begin{cases}
                \frac{2r_i + r_{i+1}}{3} & \text{falls $d_i = \upa$}\\
                r_i & \text{falls $d_i = \downa$}
            \end{cases}
    \end{align*}
    Setze nun $B_i := [a_i, b_i)$ und $y := \bigcup_{i=1}^m B_i$.
    Nach Konstruktion gilt $\Gsb^\einf(A_i, B_i)$.
    Da $x$ Grenzentität ist, ist $r_{i}^{d_i} = r_{i+1}^{d_{i+1}}$ oder $r_i < r_{i+1}$ und somit   $B_i \vor B_{i+1} \lor B_i = B_{i+1}$.
    Also ist $B_1 ... B_m$ bis auf eventuelle Dopplungen kanonische Darstellung von $y$ und es gilt $\Gsb(x,y)$.
\end{bew}

\subsubsection{$\GLDE$}

\begin{erin}[$GLDE$]
    $\GLDE(x) := \exists y ( \GSB(y) \land \Gspart(y,x) )$
\end{erin}

\begin{satz}[$\GLDE$]\label{satz:lde}
    Es gilt $\GLDE(x)$ gdw. $x$ niederdimensional ist.
\end{satz}

\begin{bew}
    $\boldsymbol{\Rightarrow}$:
    Sei $y \in \univ$ mit $\GSB(y)$ und $\Gspart(y,x)$. Dann gelten $\dim(y) < 3$ und $\dim(x) = \dim(y)$ also ist $x$ niederdimensional.\\
    $\boldsymbol{\Leftarrow}$: Sei $x \in \univ^d$ mit $d \in \{0,1,2\}$.
    Sein $\can(x) = A_1 ... A_n$. Sei $y := A_1$. Dann ist nach Satz \ref{satz:einfachgrenze} $y$ eine Grenzentität und somit gilt $\GSB(y)$.
    Außerdem ist nach Definition $\Gspart(y,x)$.
\end{bew}


\subsection{Axiome}
A1. $\GSReg([0,1))$\\
A2. $\GLDE \leftrightarrow \neg \GSReg(x)(x)$ folgt aus Satz \ref{satz:lde} und der Definition von $\GSReg$.\\
A3.\\
A4. Folgt aus Definition von $\Gspart$.

\subsubsection{A5.}
Seien $x,y \in \univ$ mit $\Gspart(x,y)$ und $\Gspart(y,x)$.
Seien $\can(x) = A_1 ... A_m$, $\can(y) = B_1 ... B_n$.
Angenommen $x \neq y$. Dann gibt es entweder ein $u \in \{1, ..., m\}$ mit $A_u \notin \{B_1, ..., B_n\}$ oder ein $v \in \{1, ..., n\}$ mit $B_v \notin \{A_1, ..., A_m\}$.
O.B.d.A.: sei ersteres der Fall.
Sei $u \in \{1, ..., m\}$ mit $A_u \notin \{B_1, ..., B_n\}$.
Wegen $\Gspart(x,y)$ gibt es ein $v$ in $\{1, ..., n\}$ mit $\Gspart^\einf(A_u,B_v)$ und wegen $\Gspart(y,x)$ ein $w$ in $\{1, ..., m\}$ mit $\Gspart^\einf(B_v,A_w)$.
Dann gelten
\begin{align*}
    \base(A_u) = \base(B_v) = \base(A_w)\\
    \core(A_u) \subsetneq \core(B_v) \subseteq \core(A_w)
\end{align*}
Also ist weder $A_u = A_w$ noch $A_u \vor A_w$ noch $A_w \vor A_u$. 
Somit ist $A_1 ... A_m$ \textit{keine} kanonische Darstellung.


\section{Future Work}
Die Block-Struktur hat viele Eigenschaften, die wir intuitiv im Brentanoraum erwarten. 
Es gibt jedoch einen Satz, der nicht von ihr erfüllt wird, den wir intuitiv haben wollen.
\begin{align}\label{satz:neu}
    \GSB(x) \land \neg \Gspart(x,y) \to \exists u (\Gsb(x,u) \land \neg \Gsb(y,u))
\end{align}
Das liegt in der Blockstruktur daran, dass Grenzentitäten mit der selben Basis exakt die gleichen Raumentitäten begrenzen.
Man nehme z.B. $x= \langle 0^\upa, [0,\frac{1}{2}) \rangle , y = \langle 0^\upa, [\frac{1}{2},1) \rangle$. Dann ist $\neg \Gspart(x,y)$ aber jeder komplexe Körper, der von $x$ begrenzt wird, wird auch von $y$ begrenzt.
Es ist anzunehmen, dass \ref{satz:neu} unabhängig von $\theoryBS$ ist. Man könnte überlegen, diesen Satz als Axiom zur Theorie dazuzunehmen und müsste dann erneut ihre Konsistenz beweisen.

Mögliche zusätzliche Axiome:
\begin{align*}
    \Ggrsb(x,y) \land \Ggrsb(x,z) \to y=z
\end{align*}
