\chapter{New Ideas}


\section{Block structure}


\subsection{Struktur}

\begin{nota}[(Beschränkte) Basisintervalle]\ \\
    \begin{align*}
      \B := \{[p,q) \subseteq \R \mid p,q \in \R\}
    \end{align*}
    ist die Menge der Basisintervalle in $\R$.
    %Wir bezeichnen die Menge der abgeschlossenen beschränkten Intervalle in $\R$ mit $I$.\\
    Für $\alpha \subseteq \R$ ist
    \begin{align*}
        \B_{\alpha} := \{ \beta \in \B \mid \beta \subseteq \alpha\}
    \end{align*}
    die Menge der und durch $\alpha$ beschränkten Basisintervalle.
\end{nota}



\begin{bsp}
 \begin{align*}
  [-5.78,14\pi) &\in \B\\
  [6,4) &\notin \B\\
  [1,1) &\notin \B\\
  [-5.78,14\pi) &\in \B_{[-10,100]}\\
  [-5.78,14\pi) &\notin \B_{[0,1]}
 \end{align*}
\end{bsp}



% \begin{dfn}
%     Für $a,b \in I$ bezeichnet
%     \begin{align*}
%         I_{[a,b]} := \{[r,s] \mid r,s \in \R, a \leq r < s \leq b\}\\
%     \end{align*}
% \end{dfn}

\begin{dfn}[Gerichtete Zahl]\ \\
    Eine gerichtete Zahl ist ein Paar $(r,d)$ mit $r \in \R$, $d \in \udset$.\\
    Statt $(r,d)$ schreiben wir auch $r^d$.\\
    $\R^\ud$ bezeichnet die Menge der gerichteten Zahlen.
%     \\Für $[a,b] \subseteq \R$ definieren wir die Menge
%         \begin{align*}
%             [a,b]^\ud := \{r^d \in \R^\ud \mid r \in [a,b] \land (r=a \to d =\upa) \land (r=b \to d=\downa)\}
%         \end{align*}
\end{dfn}


\begin{dfn}[Beschränkte gerichtete Zahl]\ \\
    Für $[a,b] \subseteq \R$ definieren wir die Menge $[a,b]^\ud \subseteq \R^\ud$ folgendermaßen: $r^d \in [a,b]^\ud$ falls:
    $$r \in [a,b] \land (r = a \to d=\upa) \land (r=b \to d=\downa)$$.
\end{dfn}


\begin{bsp}
    \begin{align*}
        7\pi^\upa &\in \R^\ud \\
        2^\downa &\in [0,7]^\ud \\
        2^\downa &\in [0,2]^\ud \\
        2^\upa &\notin [0,2]^\ud
    \end{align*}
\end{bsp}


\begin{dfn}[Untere und obere Grenzen]\ \\
    Wir definieren die Funktionen $\lb, \ub: \B \to \R^\ud$ folgendermaßen:\\
    Für $[a,b) \in \B$ sind
    \begin{align*}
        \lb([a,b)) = a^\upa,\ \ub([a,b)) = b^\downa
    \end{align*}
    Das Prädikat $\bnd \subseteq \R^\ud \times \B$ wird definiert durch
    \begin{align*}
        \bnd(x,y) \quad \Leftrightarrow \quad \lb(y)=x \lor \ub(y)=x
    \end{align*}
\end{dfn}


\begin{dfn}[Einfache Entitäten]\ %\\
    \begin{enumerate}
        \item Ein \emph{einfacher Körper} ist ein Basisintervall.\\ 
            Die Menge der einfachen Körper bezeichnen wir auch mit $\einf^3$.
        \item Eine \emph{einfache Fläche} ist ein Paar $(A,B) \in \R^\ud \times \B_{[0,1]}$.\\
            Die Menge der einfachen Flächen bezeichnen wir mit $\einf^2$.
        \item Eine \emph{einfache Linie} ist ein Tripel $(A,B,C) \in \R^\ud \times [0,1]^\ud \times \B_{[0,1]}$.\\
            Die Menge der einfachen Linien bezeichnen wir mit $\einf^1$.
        \item Ein \emph{einfacher Punkt} ist ein Tripel $(A,B,C) \in \R^\ud \times [0,1]^\ud \times [0,1]^\ud$.\\
            Die Menge der einfachen Punkte bezeichnen wir mit $\einf^0$.
        \item $\einf : \einf^3 \cup \einf^2 \cup \einf^1 \cup \einf^0$ ist die \emph{Menge der einfachen Entitäten}
    \end{enumerate}
\end{dfn}


\begin{bsp}
    \begin{align*}
        [3,5) &\in \einf^3\\
        (2^\upa, [0.2,1)) &\in \einf^2\\
        (2^\upa, [0,2)) &\notin \einf^2\\
        (2^\upa,0.7\downa, [0.2,1)) &\in \einf^1\\
        (2^\upa,1\downa, [0.2,1)) &\in \einf^1\\
        (2^\upa,0\downa, [0.2,1)) &\notin \einf^1\\
        (2^\upa,0.7\downa, 0.5\upa) &\in \einf^0\\
        (2^\upa,0.7\downa, 2\upa) &\notin \einf^0\\
        (2^\upa,0.7\downa, 1\upa) &\notin \einf^0
    \end{align*}
\end{bsp}


\begin{dfn}[Base, Core, euklidische Projektion]\ \\
    Wir definieren die Funktionen 
    $$\base: \einf \to (\R^\ud)^*,\ \intv: \einf \to \B \cup {\varnothing},\ \eukl: \einf \to 2^{\R^3}$$
    folgendermaßen:
    \begin{enumerate}
        \item Für $x \in \einf^3$ sind
            \begin{align*}
                \eukl(x) &= x \times [0,1]^2\\
                \base(x) &= \varepsilon\\
                \intv(x) &= x
            \end{align*}
        \item Für $x = (r^d, I) \in \einf^2$ sind
            \begin{align*}
                \eukl(x) &= \{r\} \times I \times [0,1]\\
                \base(x) &= r^d\\
                \intv(x) &= I
            \end{align*}
        \item Für $x = (r^d, s^e, I) \in \einf^1$ sind
            \begin{align*}
                \eukl(x) &= \{r\} \times \{s\} \times I\\
                \base(x) &= r^d s^e\\
                \intv(x) &= I
            \end{align*}
        \item Für $x = (r^d, s^e, t^f) \in \einf^0$ sind
            \begin{align*}
                \eukl(x) &= \{(r,s,t)\}\\
                \base(x) &= r^d s^e t^f\\
                \intv(x) &= \varnothing
            \end{align*}
    \end{enumerate}
\end{dfn}


\begin{dfn}[Komplexe Körper, Flächen, Linien, Punktmengen]\
    \begin{enumerate}
        \item Ein komplexer Körper ist eine endliche Vereinigung einfacher Körper.\\
            Die \emph{Menge der komplexen Körper} bezeichnen wir mit $\univ^3$.
        \item Eine komplexe Fläche ist eine endliche Vereinigung einfacher Flächen.\\
            Die \emph{Menge der komplexen Flächen} bezeichnen wir mit $\univ^2$.
        \item Eine komplexe Linie ist eine endliche Vereinigung einfacher Linien.\\
            Die \emph{Menge der komplexen Linien} bezeichnen wir mit $\univ^1$.
        \item Eine komplexe Punktmenge ist eine endliche Vereinigung einfacher Punkte.\\
            Die \emph{Menge der komplexen Punktmengen} bezeichnen wir mit $\univ^0$.
    \end{enumerate}
\end{dfn}


\begin{dfn}[Universum]\ 
    \begin{align*}
        \univ := \univ^3 \cup \univ^2 \cup \univ^1 \cup \univ^0
    \end{align*}
    ist das Universum der \strukt.
\end{dfn}


\begin{dfn}[Ausweitung der euklidischen Projektion auf $\univ$]
    Für $x = \bigcup_{i=1}^n A_i$ ist
    \begin{align*}
        \eukl(x) = \bigcup_{i=1}^n \eukl(A_i)
    \end{align*}
\end{dfn}


\begin{dfn}[Die Dimensionsfunktion]\ \\
    Wir definieren die Funktion $\Gdim : \univ \to \{0,1,2,3\}$ durch
    $$\Gdim(x) = d \quad \Leftrightarrow \quad x \in \univ^d.$$
\end{dfn}



\begin{dfn}[Die $\vor$-Relation]\ 
    \begin{itemize}
        \item Für $d_1,d_2 \in \udset$ gilt $d_1 \vor d_2$ falls $d_1 = \downa$ und $d_2 = \upa$ ist. 
        \item Für $r_1^{d_1},r_2^{d_2} \in \R^\ud$ gilt $r_1^{d_1} \vor r_2^{d_2}$ falls $r_1 < r_2$ ist oder $r_1 = r_2$ und $d_1 \vor d_2$.
        \item Für $[a_1,b_1), [a_2,b_2) \in \einf^3$ gilt $[a_1,b_1) \vor [a_2,b_2)$ falls $b_1 < a_2$ ist.
        \item Für $(A_1,B_1), (A_2,B_2) \in \einf^2$ gilt $(A_1,B_1) \vor (A_2,B_2)$, falls $A_1 \vor A_2$ gilt oder $A_1 = A_2$ und $B_1 \vor B_2$.
        \item Für $(A_1,B_1,C_1), (A_2,B_2,C_2) \in \einf^1 \cup \einf^0$ gilt $(A_1,B_1) \vor (A_2,B_2)$, falls $A_1 \vor A_2$ gilt oder $A_1 = A_2 \land B_1 \vor B_2$ oder $A_1 = A_2 \land B_1 = B_2 \land C_1 \vor C_2$.
    \end{itemize}
\end{dfn}

\begin{bsp}
    \begin{align*}
        3^\upa &\vor 5^\downa\\
        3^\downa &\vor 3^\upa\\
        [-1,2] &\vor [5,7]\\
        [-1,2] &\nvor [2,3]\\
        [-1,2] &\nvor [0,100]\\
        (3^\upa, [0,0.5]) &\vor (5^\downa, [0,0.2])\\
        (3^\downa, [0,0.5]) &\vor (3^\upa, [\frac{1}{3}, 0.7]\\
        (3^\upa, [0,0.5]) &\vor (3^\upa, [0.6,0.7])\\
        (3^\downa,0.5^\upa,[0.2, 0.7]) &\vor (4^\downa,0.5^\upa,[0, 0.7])\\
        (3^\downa,0.5^\upa,[0.2, 0.7]) &\vor (3^\upa,0^\upa,[0.2, 0.7])\\
        (3^\downa,0.5^\upa,[0.2, 0.7]) &\vor (3^\downa,0.6^\downa,[0, 0.7])\\
        (3^\downa,0.5^\downa,[0.2, 1]) &\vor (3^\downa,0.5^\upa,[0, 0.7])\\
        (3^\downa,0.5^\upa,[0.2, 0.7]) &\vor (3^\downa,0.5^\upa,[0.8, 1])\\
        (3^\downa,0.5^\upa,1^\downa) &\vor (5^\downa,0^\upa,1^\downa)\\
        (3^\downa,0.5^\upa,1^\downa) &\vor (3^\upa,0^\upa,1^\downa)\\
        (3^\downa,0.5^\upa,1^\downa) &\vor (3^\downa,1^\downa,1^\downa)\\
        (3^\downa,0.5^\upa,1^\downa) &\vor (3^\downa,0.5^\upa,1^\upa)
    \end{align*}
\end{bsp}


Für jede komplexe Entität $x$ gibt es eindeutig bestimmte einfache Entitäten $y_1, ..., y_n \in \einf^d$ mit
\begin{enumerate}
    \item $x = \bigcup_{i=1}^n y_i$.
    \item $y_1 \vor ... \vor y_n$
\end{enumerate}
\todo[inline]{beweisen}
Damit können wir die kanonische Darstellung definieren.

\begin{dfn}[Kanonische Darstellung]\ \\
    Wir definieren eine Abbildung $\can: \univ \to (\einf^3)^* \cup (\einf^2)^* \cup (\einf^1)^* \cup (\einf^0)^*$ durch $\can(x) = y_1 ... y_n$ falls
    \begin{enumerate}
        %\item $y_1, ..., y_n \in \B^d$
        \item $x = \bigcup_{i=1}^n y_i$.
        \item $y_1 \vor ... \vor y_n$
    \end{enumerate}
\end{dfn}


% \begin{dfn}[Ordinärität]\ 
%     \begin{itemize}
%         \item Ein komplexer Körper ist immer ordinär.
%         \item Eine komplexe Fläche mit kanonischer Darstellung $((p_1^{d_1}, \alpha_1), ..., (p_n^{d_n}, \alpha_n))$ heißt ordinär, wenn für alle $i,j \in \{1, ..., n\}$ mit $i \neq j$ gilt: $p_i = p_j \to \alpha_i \cap \alpha_j = \varnothing$.
%         \item Eine komplexe Linie mit kanonischer Darstellung $((p_1^{d_1}, q_1^{e_1}, \alpha_1), ... , (p_n^{d_n}, q_n^{e_n}, \alpha_n))$ ist ordinär, wenn für alle $i,j \in \{1, ..., n\}$ mit $i \neq j$ gilt: $(p_i,q_i) = (p_j,q_j) \to \alpha_i \cap \alpha_j = \varnothing$.
%         \item Ein komplexer Punkt mit kanonischer Darstellung $((p_1^{d_1}, q_1^{e_1}, r_1^{f_1}), ... , (p_n^{d_n}, q_n^{e_n}, r_n^{f_n}))$ ist ordinär, wenn für alle $i,j \in \{1, ..., n\}$ mit $i \neq j$ gilt: $(p_i, q_i,r_i) \neq (p_j,q_j,r_j)$.
%     \end{itemize}
% \end{dfn}


\begin{dfn}[$\GSReg$]
    $\GSReg^B := \univ^3$
\end{dfn}


% \begin{dfn}[$\Gsb^\einf$]\ 
%     \begin{itemize}
%         \item Für $x = (r^d,\alpha) \in \einf^2$, $y = [s,t) \in \einf^3$ gilt
%             \begin{align*}
%                 \Gsb^\einf(x,y) 
%                 \quad \Leftrightarrow \quad 
%                 r=s \land d=\upa \lor r=t \land d=\downa.
%             \end{align*}
%         \item Für $x = (r^d,s^e,\alpha) \in \einf^1$, $y = (t^f, [u,v)) \in \einf^3$ gilt
%             \begin{align*}
%                 \Gsb^\einf(x,y) 
%                 \quad \Leftrightarrow \quad 
%                 r^d = t^f \land ( s=u \land e=\upa \lor s=v \land e=\downa ).
%             \end{align*}
%         \item Für $x = (p^d,q^e,r^f) \in \einf^0$, $y = (s^g, t^h, [u,v)) \in \einf^3$ gilt
%             \begin{align*}
%                 \Gsb^\einf(x,y) 
%                 \quad \Leftrightarrow \quad 
%                 p^d = s^g \land q^e = t^h \land ( r=u \land f=\upa \lor r=v \land f=\downa ).
%             \end{align*}
%     \end{itemize}
% \end{dfn}


\begin{dfn}[$\Gsb^\einf$]\ 
    \begin{itemize}
        \item Für $x = (r^d, \alpha) \in \einf^2$, $y = \beta \in \einf^3$ gilt
            \begin{align*}
                \Gsb^\einf(x,y) 
                \quad \Leftrightarrow \quad 
                \bnd(r^d,\beta).
            \end{align*}
        \item Für $x = (r^d,s^e,\alpha) \in \einf^1$, $y = (t^f, \beta) \in \einf^3$ gilt
            \begin{align*}
                \Gsb^\einf(x,y) 
                \quad \Leftrightarrow \quad 
                r^d = t^f \land ( \bnd(s^e,\beta) ).
            \end{align*}
        \item Für $x = (p^d,q^e,r^f) \in \einf^0$, $y = (s^g, t^h, \beta) \in \einf^3$ gilt
            \begin{align*}
                \Gsb^\einf(x,y) 
                \quad \Leftrightarrow \quad 
                p^d = s^g \land q^e = t^h \land ( \bnd(r^f,\beta)).
            \end{align*}
    \end{itemize}
\end{dfn}


\begin{dfn}[$\Gsb$]\ \\
    Für $d \in \{0,1,2\}$, $m,n \in \N$, $x \in \univ^d$, $y \in \univ^{d+1}$ mit $\can(x) = A_1 ... A_n \in \univ^d$, $\can(y) = B_1 ... B_m \in \univ^{d+1}$ gilt:
    \begin{align*}
        \Gsb(x,y) \quad \Leftrightarrow \quad \forall i \in \{1, ..., n\}\ \exists j \in \{1, .., m\}: \Gsb^\einf(A_i,B_j).
    \end{align*}
\end{dfn}


% \begin{dfn}[$\Gscoinc^\einf$]\ 
%     \begin{itemize}
%         \item Für $x = (p_1^{d_1},\alpha_2), y = (p_2^{d_2},\alpha_2) \in \einf^2$ gilt
%             \begin{align*}
%                 \Gscoinc^\einf(x,y) 
%                 \quad \Leftrightarrow \quad 
%                 p_1 = p_2 \land \alpha_1 = \alpha_2.
%             \end{align*}
%         \item Für $x = (p_1^{d_1},q_1^{e_1},\alpha_1), y = (p_2^{d_2},q_2^{e_2},alpha_2) \in \einf^1$ gilt
%             \begin{align*}
%                 \Gscoinc^\einf(x,y) 
%                 \quad \Leftrightarrow \quad 
%                 (p_1, q_1) = (p_2,q_2) \land \alpha_1 = \alpha_2.
%             \end{align*}
%         \item Für $x = (p_1^{d_1},q_1^{e_1},r_1^{f_1}), y = (p_2^{d_2},q_2^{e_2},r_2^{f_2}) \in \einf^0$ gilt
%             \begin{align*}
%                 \Gscoinc^\einf(x,y) 
%                 \quad \Leftrightarrow \quad 
%                 (p_1,q_1,r_1) = (p_1,q_1,r_1).
%             \end{align*}
%     \end{itemize}
% \end{dfn}


% \begin{dfn}[$\Gscoinc^\einf$]\ 
%     Für $x,y \in \einf^2 \cup \einf^1 \cup \einf^0$ gilt
%     \begin{align*}
%         \Gscoinc^\einf(x,y) 
%         \quad \Leftrightarrow \quad 
%         \eukl(x) = \eukl(y)
%     \end{align*}
% \end{dfn}



\begin{dfn}[Grenzentität]\
    \begin{enumerate}
        \item Sei $x \in \univ^2$ mit $\can(x) = A_1 ... A_n$ und $A_i = (r_i^{d_i},I_i)$ für $i \{1, ..., n\}$. Dann ist $x$ Grenzentität, falls gilt
            \begin{align*}
                \forall i,j \in \{1, ..., n\}: r_i = r_j \to d_i = d_j
            \end{align*}
        \item Sei $x \in \univ^1$ mit $\can(x) = A_1 ... A_n$ und $A_i = (r_i^{d_i},s_i^{e_i},I_i)$ für $i \{1, ..., n\}$. Dann ist $x$ Grenzentität, falls gilt
            \begin{align*}
                \forall i,j \in \{1, ..., n\}: r_i^{d_i} = r_j^{d_j} \land s_i = s_j \to e_i = e_j
            \end{align*}
        \item Sei $x \in \univ^0$ mit $\can(x) = A_1 ... A_n$ und $A_i = (r_i^{d_i},s_i^{e_i},t_i^{f_i})$ für $i \{1, ..., n\}$. Dann ist $x$ Grenzentität, falls gilt
            \begin{align*}
                \forall i,j \in \{1, ..., n\}: (r_i^{d_i}, s_i^{e_i}) = (r_j^{d_j},s_j^{e_j}) \land t_i = t_j \to f_i = f_j
            \end{align*}
    \end{enumerate}
\end{dfn}
\todo[inline]{untersuchen ob $\GSB(x)$ gdw $x$ Grenzentität ist}


\begin{dfn}[$\Gscoinc$]\ \\
    Für $x,y \in \univ^d$ gilt $\Gscoinc(x,y)$ falls
    \begin{enumerate}
        \item $d \neq 3$ ist
        \item $x$ und $y$ Grenzentitäten sind und
        \item $\eukl(x) = \eukl(y)$.
    \end{enumerate}
\end{dfn}


\begin{dfn}[$\Gspart^\einf$]\ 
    \begin{itemize}
        \item Für $x,y \in \univ^3$ gilt
            \begin{align*}
                \Gspart^\einf(x,y) 
                \quad \Leftrightarrow \quad 
                x \subseteq y.
            \end{align*}
        \item Für $x = (A_1,\alpha_2), y = (A_2,\alpha_2) \in \einf^2$ gilt
            \begin{align*}
                \Gspart^\einf(x,y) 
                \quad \Leftrightarrow \quad 
                A_1 = A_2 \land \alpha_1 \subseteq \alpha_2.
            \end{align*}
        \item Für $x = (A_1,B_1,\alpha_1), y = (A_2,B_2,\alpha_2) \in \einf^1$ gilt
            \begin{align*}
                \Gspart^\einf(x,y) 
                \quad \Leftrightarrow \quad 
                (A_1, B_1) = (A_2,B_2) \land \alpha_1 \subseteq \alpha_2.
            \end{align*}
        \item Für $x, y \in \einf^0$ gilt
            \begin{align*}
                \Gspart^\einf(x,y) 
                \quad \Leftrightarrow \quad 
                x=y.
            \end{align*}
    \end{itemize}
\end{dfn}


\begin{dfn}[$\Gspart$]\ \\
    Für $d \in \{0,1,2,2\}$, $m,n \in \N$, $x,y \in \univ^d$ mit $\can(x) = A_1 ... A_n$, $\can(y) = B_1 ... B_m$ gilt:
    $$\Gspart(x,y) \quad \Leftrightarrow \quad \forall i \in \{1, ..., n\}\ \exists j \in \{1, .., m\}: \Gspart^\einf(A_i,B_j). $$
\end{dfn}


\subsection{Untersuchung ausgewählter definierter Relationen}

\begin{satz}
    Für $x \in \univ^2$ gilt $\GSB(x)$ gdw. $x$ eine Grenzentität ist.
\end{satz}

\begin{bew}
    Sei $\can(x) = A_1 ... A_m$ mit $A_i = (r_i^{d_i}, I_i)$ für $i \in \{1, ..., m\}$.\\
    $\boldsymbol{\Rightarrow}$: 
    Es gelte $\Gsb(x,y)$.
    Dann ist $y \in \univ^3$.
    Sei $\can(y) = B_1 ... B_n$ mit $B_k = [a_k,b_k)$ für $k \in \{1, ..., n\}$.
    Für $i \in \{1, ..., m\}$ sei $k(i) \in \{1, ..., n\}$ s.d. $\Gsb^\einf(A_i,B_{k(i)})$ gilt.
    Angenommen $x$ ist keine Grenzentität.
    Dann gibt es $i,j \in \{1, ..., m\}$ mit $r_i = r_j$ und $d_i \neq d_j$. O.B.d.A: sei $i<j$. Dann ist $A_i \vor A_j$ und somit $d_i = \downa$, $d_j = \upa$.
    Also ist $b_{k(i)} = r_i = r_j = a_{k(j)}$ und somit $B_{k(i)} \neq B_{k(j)}$. Gleichzeitig gilt weder $B_{k(i)} \vor B_{k(j)}$ noch $B_{k(j)} \vor B_{k(i)}$. 
    Somit ist $B_1 ... B_n$ keine kanonische Darstellung. $\lightning$\\
    $\boldsymbol{\Leftarrow}$:
    Falls $m=1$ seien
    \begin{align*}
        a_1 := 
            \begin{cases}
                r_1 & \text{falls $d_1 = \upa$}\\
                r_1-1 & \text{falls $d_1 = \downa$}
            \end{cases}\\
        b_1 := 
            \begin{cases}
                r_1+1 & \text{falls $d_1 = \upa$}\\
                r_1 & \text{falls $d_1 = \downa$}
            \end{cases}\\
    \end{align*}
    %
    Falls $m \geq 2$ seien
    \begin{align*}
        a_1 := 
            \begin{cases}
                r_1 & \text{falls $d_1 = \upa$}\\
                r_1-1 & \text{falls $d_1 = \downa$}
            \end{cases}\\
        b_1 := 
            \begin{cases}
                \frac{2r_2 + r_2}{3} & \text{falls $d_1 = \upa$}\\
                r_1 & \text{falls $d_1 = \downa$}
            \end{cases}\\
        a_m := 
            \begin{cases}
                r_m & \text{falls $d_m = \upa$}\\
                \frac{2r_m + r_{m-1}}{3} & \text{falls $d_m = \downa$}
            \end{cases}\\
        b_m := 
            \begin{cases}
                r_m+1 & \text{falls $d_m = \upa$}\\
                r_m & \text{falls $d_m = \downa$}
            \end{cases}
    \end{align*}
    %
    Und falls $m \geq 3$ seien für $i \in \{2, ..., m-1\}$
    \begin{align*}
        a_i := 
            \begin{cases}
                r_i & \text{falls $d_i = \upa$}\\
                \frac{2r_i + r_{i-1}}{3} & \text{falls $d_i = \downa$}
            \end{cases}\\
        b_i := 
            \begin{cases}
                \frac{2r_i + r_{i+1}}{3} & \text{falls $d_i = \upa$}\\
                r_i & \text{falls $d_i = \downa$}
            \end{cases}
    \end{align*}
    Für $i \in \{1, ..., m\}$ setze nun $B_i = [a_i, b_i)$.
    Sei $y := \bigcup_{i=1}^m B_i$.
    Dann gilt $\Gsb(x,y)$.
\end{bew}








