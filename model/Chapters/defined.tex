\section{Definierte Relationen}

% \begin{erin}[D1. $x$ ist eine 2-dim.\ Grenze von $y$]
%     \begin{align*}
%         \Gtwodb(x,y) := \GSReg(y) \wedge \Gsb(x,y)
%     \end{align*}
% \end{erin}
% 
% \begin{erin}[D2. $x$ ist eine 1-dim.\ Grenze von $y$]
%     \begin{align*}
%         \Gonedb(x,y) := \GtwoDB(y) \wedge \Gsb(x,y)
%     \end{align*}
% \end{erin}
% 
% \begin{erin}[D3. $x$ ist eine 0-dim.\ Grenze von $y$]
%     \begin{align*}
%         \Gzerodb(x,y) := \GoneDB(y) \wedge \Gsb(x,y)
%     \end{align*}
% \end{erin}

\begin{erin}[D4. $x$ ist eine räumliche Grenze]
    \begin{align*}
        \GSB(x) := \exists y\ \Gsb(x,y)
    \end{align*}
\end{erin}

\begin{hyp}\ 
    \begin{itemize}
        \item Für eine komplexe Fläche $A \in \univ^2$ gilt $\GSB(x)$ gdw. für alle $(p,q,r) \in A$ gilt $(p,q,\ol{r}) \notin A$.
        \item Für eine komplexe Linie $A \in \univ^1$ gilt $\GSB(x)$ gdw. für alle $(p,q,r) \in A$ gilt $(p,\ol{q},r) \notin A$.
        \item Für eine komplexe Punktmenge $A \in \univ^0$ gilt $\GSB(x)$ gdw. für alle $(p,q,r) \in A$ gilt $(\ol{p},q,r) \notin A$.
    \end{itemize}
\end{hyp}


% \begin{erin}[D5. $x$ ist eine $\Gd$-dim.\ Grenze]
%     \begin{align*}
%         \GdDB(x) := \exists y\ \Gddb(x, y)
%     \end{align*}
% \end{erin}

\begin{erin}[D6. $x$ ist die größte Grenze von $y$]
    \begin{align*}
        \Ggrsb(x,y) := \Gsb(x,y) \wedge \forall z\ (\Gsb(z,y) \to \Gspart(z,x))
    \end{align*}
\end{erin}

\begin{erin}[D6. Die $\Ggrsb$-Funktion]
    \begin{align*}
        \Ggrsb(x) = y \quad \Leftrightarrow \quad \Ggrsb(y,x)
    \end{align*}
\end{erin}

\begin{hyp}
    Für eine höherdimensionale komplexe Entität $x$ ist $\Ggrsb(x) = \partial x$.
\end{hyp}

 
\begin{erin}[D7. $x$ ist eine niederdimensionale Raumentität]
    \begin{align*}
        \GLDE(x) := \exists y\ (\GSB(y) \wedge \Gspart(y, x))
    \end{align*}
\end{erin}

\begin{satz}\ \\
    Für eine komplexe Entität $x$ gilt $\GLDE(x)$ gdw. $\dim(x) \neq 3$ (d.h. $x$ ist niederdimensional).
\end{satz}
Beweis trivial.


\begin{erin}[D8. $x$ ist ein Flächen-/Linien-/Punktregion]
    \begin{align*}
        \GdDE(x) := \exists y\ (\GdDB(y) \wedge \Gspart(y, x))
    \end{align*}
\end{erin}

\begin{satz}\ \\
    Für eine komplexe Entität $x$ gilt $\GdDE(x)$ gdw. $\dim(x) = \Gd$ ist.
\end{satz}
Beweis trivial.


\begin{erin}[D9. $x$ und $y$ haben die selbe Dimension]
    \begin{align*}
        \Geqdim(x,y) := (&\GSReg(x) \wedge \GSReg(y)) \vee \GtwoDE(x) \wedge \GtwoDE(y))\\ \vee (&\GoneDE(x) \wedge \GoneDE(y)) \vee (\GzeroDE(x) \wedge \GzeroDE(y))
    \end{align*}
\end{erin}

\begin{satz}\ \\
    Für komplexe Entitäten $x,y$ gilt $\Geqdim(x,y)$ gdw. $\dim(x) = \dim(y)$ ist.
\end{satz}
Beweis trivial.


\begin{erin}[D10. $x$ ist ein echter Teil von $y$]
    \begin{align*}
        \Gsppart(x, y) := \Gspart(x, y) \land  x \neq y
    \end{align*}
\end{erin}

\begin{satz}\ \\
    Für komplexe Entitäten $x,y$ gilt $\Gsppart(x,y)$ gdw. $x \subsetneq y$ ist.
\end{satz}
Beweis trivial.


\begin{erin}[D11. $x$ und $y$ überlappen]
    \begin{align*}
        \Gsov(x, y) := \exists z\  (\Gspart(z, x) \wedge \Gspart(z, y))
    \end{align*}
\end{erin}

\begin{satz}\ \\
    Für komplexe Entitäten $x,y$ gilt $\Gsov(x,y)$ gdw. $\dim(x) = \dim(y)$ und $x \cap y \neq \varnothing$ ist.
\end{satz}
Beweis trivial.


\begin{erin}[D12. $x$ ist die mereologische Summe von $x_{1}$,\ldots,$x_{n}$]
    \begin{align*}
        \Gsumn(x_{1},\ldots,x_{n},x) := \forall y\ (\Gsov(y,x) \leftrightarrow \bigvee^{n}_{i=1} \Gsov(y,x_{i}))
    \end{align*}
\end{erin}

\begin{erin}[Die $\Gsumn^f$-Funktion]
    \begin{align*}
        \Gsumn^f(x_{1},\ldots,x_{n}) = x \quad \Leftrightarrow \quad \Gsumn(x_{1},\ldots,x_{n},x)
    \end{align*}
\end{erin}

\begin{hyp}\ \\
    Für $x_1, ..., x_n \in \univ^d$ ist $\Gsumn^f(x_1, ..., x_n) = \bigcup_{i=1}^n x_i$
\end{hyp}


\begin{erin}[D13. $x$ ist der mereologische Schnitt von $x_{1}$,\ldots,$x_{n}$]
    \begin{align*}
        \Gintersectn(x_{1},\ldots,x_{n},x)  := \forall y\ (\Gspart(y,x) \leftrightarrow \bigwedge^{n}_{i=1} \Gspart(y,x_{i}))
    \end{align*}
\end{erin}

\begin{erin}[Die $\Gintersectn^f$-Funktion]\ \\
    Wir schreiben $\Gintersectn^f(x_{1},\ldots,x_{n}) = y$ falls $\Gintersectn(x_{1},\ldots,x_{n},y)$ gilt.
\end{erin}

\begin{hyp}\ \\
    Für $x_1, ..., x_n \in \univ^d$ mit $\bigcap_{i=1}^n x_i \neq \varnothing$ ist $\Gintersectn^f(x_1, ..., x_n) = \bigcap_{i=1}^n x_i$
\end{hyp}


\begin{erin}[D14. $x$ ist das rel. Kompl. von $x_{n}$ und $x_{1}$,\ldots,$x_{n-1}$]
    \begin{align*}
        \Grelcompln(x_{1},\ldots,x_{n},x) := \hspace*{1em}\bigwedge_{1\leq i \leq n} \Geqdim(x_i, x) \wedge\\
        \forall y\ (\Gspart(y,x)
        \leftrightarrow \bigwedge^{n-1}_{i=1} \neg\Gsov(y,x_{i}) \wedge \Gspart(y, x_{n}))
    \end{align*}
\end{erin}

\begin{erin}[Die $\Grelcompln^f$-Funktion]\ \\
    Wir schreiben $\Grelcompln^f(x_{1},\ldots,x_{n}) = y$ falls $\Grelcompln(x_{1},\ldots,x_{n},y)$ gilt.
\end{erin}

\begin{hyp}\ \\
    Für $x_1, ..., x_n \in \univ^d$ mit $x_n \setminus \bigcup_{i=1}^{n-1} x_i \neq \varnothing$ ist 
    \begin{align*}
        \Grelcompln^f(x_1, ..., x_n) = x_n \setminus \bigcup_{i=1}^{n-1} x_i
    \end{align*}
\end{hyp}



\begin{erin}[D15. $x_{1}$,\ldots,$x_{n}$ zerlegen $x$]
    \begin{align*}
        \Gpartitionn(x_{1},\ldots,x_{n},x) := &\Gsumn(x_{1},\ldots,x_{n},x) \wedge\\ &\bigwedge_{1 \leq i < j \leq n} \neg\Gsov(x_i,x_j)
    \end{align*}
\end{erin}

\begin{hyp}\ \\
    Für $x, x_1, ..., x_n \in \univ^d$ gilt $\Gpartitionn(x_{1},\ldots,x_{n},x)$ falls
    \begin{enumerate}
        \item $ x = \bigcup_{i=1}^{n} x_i$ und
        \item $x_i \cap x_j = \varnothing$ für alle $i,j \in \{1, ..., n\}$ mit $ i \neq j$
    \end{enumerate}
    gelten.
\end{hyp}



\begin{erin}[D16. $x$ ist ein 2-dim.\ Hyperteil von $y$]
    \begin{align*}
        \Gtwodhypp(x,y) := \exists z\ (\Gspart(z,y) \wedge \Gtwodb(x,z))
    \end{align*}
\end{erin}

\begin{satz}\ \\
    Für $x \in \univ^2$ und $y \in \univ^3$ gilt $\Gtwodhypp(x,y)$ gdw. 
    \begin{enumerate}
        \item $x \subset y$ und
        \item $\GSB(x)$
    \end{enumerate}
    gelten.
\end{satz}

\begin{bew}
    ``$\Rightarrow$'': Sei $z \in \univ$ mit $\Gspart(z,y)$ und $\Gtwodb(x,z)$.
    Also gilt $\Gsb(x,z)$ und somit auch $\GSB(x)$ und nach Satz \ref{satz:sb-teil} $x \subset z \subseteq y$.
    \\
    ``$\Leftarrow$'': Sei $u \in \univ^3$ mit $\Gsb(u,x)$. 
    Setze $z := u \cap y$.
    Wegen $x \subset u$ und $x \subset y$ ist $x \subset z$ und damit $z$ als nichtleerer Schnitt von mengen aus $\univ^3$ wieder in $\univ^3$.
    Klar ist damit $\Gspart(z,y)$.
    Außerdem gilt für alle $(p,q,r) \in x$ dass $(p,q,\ol{r}) \notin u$ ist und somit auch nicht in $z$. 
    Also ist $x \subset \partial z$ und wegen $\dim(x) = 2$ und $\dim(z) = 3$ gilt $\Gtwodb(x,z)$.
\end{bew}


\begin{erin}[D17. $x$ ist ein 1-dim.\ Hyperteil von $y$]
    \begin{align*}
        \Gonedhypp(x,y) := \exists z\ ((&\Gspart(z,y) \vee \Gtwodhypp(z,y))\\
        \wedge  &\Gonedb(x,z))
    \end{align*}
\end{erin}

\begin{satz}\ \\
    Für $x \in \univ^1$ und $y \in \univ^3 \cup \univ^2$ gilt $\Gonedhypp(x,y)$ gdw. 
    \begin{enumerate}
        \item $x \subset y$ und
        \item $\GSB(x)$
    \end{enumerate}
    gelten.
\end{satz}

\begin{erin}[D18. $x$ ist ein 0-dim.\ Hyperteil von $y$]
    \begin{align*}
        \Gzerodhypp(x,y) := \exists z\ ((&\Gspart(z,y) \vee \Gonedhypp(z,y))\\ 
        \wedge  &\Gzerodb(x,z))
    \end{align*}
\end{erin}

\begin{satz}\ \\
    Für $x \in \univ^0$ und $y \in \univ^3 \cup \univ^2 \cup \univ^1$ gilt $\Gzerodhypp(x,y)$ gdw. 
    \begin{enumerate}
        \item $x \subset y$ und
        \item $\GSB(x)$
    \end{enumerate}
    gelten.
\end{satz}
               
\begin{erin}[D19. $x$ ist ein Hyperteil von $y$]
    \begin{align*}
        \Ghypp(x,y) := \Gtwodhypp(x,y) \vee \Gonedhypp(x,y) \vee \Gzerodhypp(x,y)
    \end{align*}
\end{erin}

\begin{satz}\ \\
    Für $x,y \in \univ$ gilt $\Ghypp(x,y)$ gdw. 
    \begin{enumerate}
        \item $\dim(x) < \dim(y)$,
        \item $x \subset y$ und
        \item $\GSB(x)$
    \end{enumerate}
    gelten.
\end{satz}
  
%%%%%%%%%%%%%%%%%%%%%%%%%%%%%%%%%%%%%%%%%%%%%%%%%%%%%%%%%%%%%%%

\begin{erin}[D20. $x$ ist ein tangentialer Teil von $y$]
    \begin{align*}
        \Gtangpart(x,y) := (\Gspart(x,y) \vee \Ghypp(x,y))\ \ \wedge\\
        \exists x'zz'\ ((\Gspart(x',x) \vee \Ghypp(x',x)) \wedge 
        \Gsb(z,y) \\
        \wedge (\Gspart(z',z) \vee \Ghypp(z',z)) \wedge \Gscoinc(x',z'))
    \end{align*}
\end{erin}

\begin{hyp}\label{satz:tangpart}
    \begin{itemize}\ 
        \item Für $x,y \in \univ$ mit $\dim(y) = 3$ und $\Gtangpart(x,y)$ gilt
            \begin{align*}
                x \subseteq y \land \exists (p,q,r) \in x : (p,q,\ol{r}) \notin y.
            \end{align*}
        \item Für $x,y \in \univ$ mit $\dim(y) = 2$ und $\Gtangpart(x,y)$ gilt
            \begin{align*}
                x \subseteq y \land \exists (p,q,r) \in x : (p,\ol{q},r) \notin y.
            \end{align*}
        \item Für $x,y \in \univ$ mit $\dim(y) = 1$ und $\Gtangpart(x,y)$ gilt
            \begin{align*}
                x \subseteq y \land \exists (p,q,r) \in x : (\ol{p},q,r) \notin y.
            \end{align*}
        \item Für $x,y \in \univ$ mit $\dim(y) = 0$ gilt nie $\Gtangpart(x,y)$.
    \end{itemize}
\end{hyp}


%%%%%%%%%%%%%%%%%%%%%%%%%%%%%%%%%%%%%%%%%%%%%%%%%%%%%%%%%%%

\begin{erin}[D21. $x$ ist ein innerer Teil von $y$]
    \begin{align*}
        \Ginpart(x, y) := (\Gspart(x,y) \vee \Ghypp(x,y)) \land \neg \Gtangpart(x, y)
    \end{align*}
\end{erin}

\begin{hyp}\label{satz:inpart}
    \begin{itemize}\ 
        \item Für $x,y \in \univ$ mit $\dim(y) = 3$ und $\Ginpart(x,y)$ gilt
            \begin{align*}
                x \subseteq y \land \forall (p,q,r) \in x : (p,q,\ol{r}) \in y.
            \end{align*}
        \item Für $x,y \in \univ$ mit $\dim(y) = 2$ und $\Ginpart(x,y)$ gilt
            \begin{align*}
                x \subseteq y \land \forall (p,q,r) \in x : (p,\ol{q},r) \in y.
            \end{align*}
        \item Für $x,y \in \univ$ mit $\dim(y) = 1$ und $\Ginpart(x,y)$ gilt
            \begin{align*}
                x \subseteq y \land \forall (p,q,r) \in x : (\ol{p},q,r) \in y.
            \end{align*}
        \item Für $x,y \in \univ$ mit $\dim(y) = 0$ und $\Ginpart(x,y)$ gilt
            \begin{align*}
                x \subseteq y.
            \end{align*}
    \end{itemize}
\end{hyp}

%%%%%%%%%%%%%%%%%%%%%%%%%%%%%%%%%%%%%%%%%%%%%%%%%%%%%%

\begin{erin}[D22. $x$ ist eine extraordinäre Raumentität]
    \begin{align*}
        \GExOrd(x) := \exists yz\ (&\Gspart(y,x) \wedge \Gspart(z,x) \\
        \wedge \neg&\Gsov(y,z) \wedge \Gscoinc(y,z))
    \end{align*}
\end{erin}

\begin{hyp}\
    \begin{enumerate}
        \item Für $x \in \univ^3$ gilt nie $\GExOrd(x)$.
        \item Für $x \in \univ^2$ gilt $\GExOrd(x)$ gdw.
            \begin{align*}
                \exists (p,q,r) \in x : (p,q,\ol{r})
            \end{align*}
        \item Für $x \in \univ^1$ gilt $\GExOrd(x)$ gdw.
            \begin{align*}
                \exists (p,q,r) \in x : ((p,q,\ol{r}) \in x \lor (p,\ol{q},r) \in x \lor (p,\ol{q},\ol{r}) \in x)
            \end{align*}
        \item Für $x \in \univ^0$ gilt $\GExOrd(x)$ gdw.
            \begin{align*}
                \exists (p,q,r) \in x : (&(p,q,\ol{r}) \in x \lor (p,\ol{q},r) \in x \lor (p,\ol{q},\ol{r}) \in x\\
                \lor &(\ol{p},q,r) \in x \lor (\ol{p},q,\ol{r}) \in x \lor (\ol{p},\ol{q},r) \lor (\ol{p},\ol{q},\ol{r}))
            \end{align*}   
    \end{enumerate}

\end{hyp}


\begin{erin}[D23. $x$ ist eine ordinäre Raumentität]
    \begin{align*}
        \GOrd(x) := \neg\GExOrd(x)
    \end{align*}
\end{erin}

\begin{erin}[D24. $x$ ist $\Gd$-dim.\ zusammenhängend]
    \begin{align*}
        \GdDC(x) := \exists uv\ (\Gpartition(u,v,x)) \wedge \forall yz\ (\Gpartition(y,z,x)\\ 
        \to 
        \exists y'z'(\Gddhypp(y',y) \wedge \Gddhypp(z',z) \wedge \Gscoinc(y',z')))
    \end{align*}
\end{erin}

\begin{erin}[D25. $x$ ist zusammenhängend]
    \begin{align*}
        \GC(x) := \GtwoDC(x) \vee \GoneDC(x) \vee \GzeroDC(x)
    \end{align*}
\end{erin}

\begin{erin}[D26. $x$ und $y$ sind zusammenhängend]
    \begin{align*}
        \Gc(x,y) := \exists z\ (\Gsum(x,y,z) \wedge \GC(z))
    \end{align*}
\end{erin}

\begin{erin}[D27. $x$ und $y$ sind extern zusammenhängend]
    \begin{align*}
        \Gexc(x,y) := \Gc(x,y) \wedge \neg\Gsov(x,y)
    \end{align*}
\end{erin}

\begin{erin}[D28. $x$ hat eine $\Gd$-dim.\ zusammenhängende Komponente]
    \begin{align*}
        \GoneCCDd(x) := \GdDC(x)
    \end{align*}
\end{erin}

\begin{erin}[D29. $x$ hat $n$ $\Gd$-dim.\ zusammenhängende Komp.]
    \begin{align*}
        \GnCCDd(x) := &\bigwedge_{i=1}^{n-1} \neg\GiCCDd(x) \wedge\\
                    &\exists x_{1}...x_{n}(\Gpartitionn(x_{1},...,x_{n},x)
                    \wedge \bigwedge_{i=1}^{n} \GoneCCDd(x_{i}))
    \end{align*}
\end{erin}

\begin{erin}[D30. $x$ ist ein Topoid]
    \begin{align*}
        \GTop(x) := \GSReg(x) \wedge \GOrd(x) \wedge \GtwoDC(x)
    \end{align*}
\end{erin}

\begin{erin}[D31. $x$ ist eine Fläche]
    \begin{align*}
        \GtwoD(x) := \GtwoDE(x) \wedge \GOrd(x) \wedge \GoneDC(x)
    \end{align*}
\end{erin}

\begin{erin}[D32. $x$ ist eine Linie]
    \begin{align*}
        \GoneD(x) := \GoneDE(x) \wedge \GOrd(x) \wedge \GzeroDC(x)
    \end{align*}
\end{erin}

\begin{erin}[D33. $x$ ist ein Punkt]
    \begin{align*}
        \GzeroD(x) := \GzeroDE(x) \wedge \GOrd(x) \wedge \neg\exists y\ \Gsppart(y,x)
    \end{align*}
\end{erin}

\begin{hyp}\label{satz:zerod}
    \begin{align*}
        \GzeroD(x) \quad \Leftrightarrow \quad \exists p \in (\R^\ud)^3 : x = \{p\}
    \end{align*}
\end{hyp}


\begin{erin}[D34. $x$ ist eine strikte räumliche Grenze]
    \begin{align*}
        \Gstrictsb(x,y) :=
        &\Gsb(x,y) \wedge \\
        &\forall x' (\Ghypp(x',y) \wedge \Gscoinc(x,x') \to x=x')
    \end{align*}
\end{erin}

\begin{erin}[D35. $x$ ist eine schwache räumliche Grenze]
    \begin{align*}
        \Gweaksb(x,y) := \Gsb(x,y) \wedge \neg\Gstrictsb(x,y)
    \end{align*}
\end{erin}
% 
% \begin{erin}
%     \begin{align*}
%         
%     \end{align*}
% \end{erin}
% 
%     &\Gonedircomp(x,y) := \GzeroD(x) \wedge \GzeroD(y) \wedge \Gscoinc(x,y) \wedge \mbox{}
%         \\ &\hspace{1em}
%         \exists x',y',x'',y''\ (\Gsb(x,x') \wedge \Gsb(x',x'') \wedge \Gsb(y,y') \wedge \Gsb(y',y'') \wedge
%         \\ &\hspace{2em}
%         \neg \Gsov(x',y') \wedge \neg \Gsov(x'', y'') \wedge \mbox{}
%         \\ &\hspace{2em}
%         \forall z\ (\Gscoinc(x,z) \wedge \Ghypp(z, \Ggrsb_f(\Gsum_f(x'',y''))) \to z=x \vee z=y))
%  
% \begin{erin}
%     \begin{align*}
%         
%     \end{align*}
% \end{erin}
% 
%     \begin{enumAx}[D]
% \itemTP{$\Gonedircomp(x,y) \Ldef \Gscoinc(x, y) \wedge \mbox{}$\\
% \hspace*{1em}
% 	$\exists x' y' z'\: 
% 	 (\:\GOrd(x') \wedge
% 	    \GOrd(y') \wedge
% 		  \Gsb(x, x') \wedge
% 			\Gsb(y, y') \wedge
%       \neg \Gsov(x', y') \wedge	\mbox{}$\\
%   \hspace*{2.5em}
%     $ ( \GSB(x') \wedge \GSB(y') \Limp \mbox{}$\\
% 	\hspace*{4em}
% 			 $\exists x'' y'' z'' g' \:
% 			  (\:\Gsb(x', x'') \wedge
% 					 \Gsb(y', y'') \wedge
%            \neg \Gsov(x'', y'') \wedge \mbox{}$\\
% 	\hspace*{6em}
% 	     $	 \Gsum(x'', y'', z'') \wedge
% 					 \Ggrsb(g', z'') \wedge
% 					 \Gzerodhypp(x, g') \wedge 
% 					 \Gzerodhypp(y, g') \wedge \mbox{}$\\
% 	\hspace*{6em}
% 				 $ \forall z (  \Gzerodhypp(z, g') \Limp \mbox{}$\\
% 	\hspace*{8em}		
% 					  $((\Gscoinc(x, z) \wedge \neg\Gsov(x,z) \Limp y=z ) \wedge \mbox{}$\\
% 	\hspace*{8em}\hphantom{$($}		
% 						$(\Gscoinc(y, z) \wedge \neg\Gsov(y,z) \Limp x=z ) 
% 					\ )))))$\\ \mbox{} \label{def:npodircomp}}
%   {$x$ and $y$ are 1-directionally compatible}
% 
% \begin{erin}
%     \begin{align*}
%         
%     \end{align*}
% \end{erin}
% 
% \itemTP{$\Gonecont(x) \Ldef \GoneDE(x) \wedge \mbox{}$\\
% \hspace*{1em}
%   $\forall y\:(\:(\Gzerodhypp(y,x) \wedge
% 	                \forall y' ( \Gspart(y',y) \to \neg\Gstrictsb(y',x))) \Limp \mbox{} $\\
% 	\hspace*{2.5em}	
% 	  $\exists z ( \Gzerodhypp(z,x) \wedge
% 		             \neg\Gsov(z,y) \wedge 
%                  \Gonedircomp(y,z)))$\\ \mbox{} \label{def:Gdcont}}
%              {$x$ is 1-continuous}	
% \end{enumAx}
